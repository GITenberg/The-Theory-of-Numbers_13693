% %%%%%%%%%%%%%%%%%%%%%%%%%%%%%%%%%%%%%%%%%%%%%%%%%%%%%%%%%%%%%%%%%%%%%%% %
%                                                                         %
% The Project Gutenberg EBook of The Theory of Numbers, by Robert D. Carmichael
%                                                                         %
% This eBook is for the use of anyone anywhere at no cost and with        %
% almost no restrictions whatsoever.  You may copy it, give it away or    %
% re-use it under the terms of the Project Gutenberg License included     %
% with this eBook or online at www.gutenberg.org                          %
%                                                                         %
%                                                                         %
% Title: The Theory of Numbers                                            %
%                                                                         %
% Author: Robert D. Carmichael                                            %
%                                                                         %
% Release Date: April 8, 2013 [EBook #13693]                              %
%                                                                         %
% Language: English                                                       %
%                                                                         %
% Character set encoding: TeX                                             %
%                                                                         %
% *** START OF THIS PROJECT GUTENBERG EBOOK THE THEORY OF NUMBERS ***     %
%                                                                         %
% %%%%%%%%%%%%%%%%%%%%%%%%%%%%%%%%%%%%%%%%%%%%%%%%%%%%%%%%%%%%%%%%%%%%%%% %

\def\ebook{13693}
%%%%%%%%%%%%%%%%%%%%%%%%%%%%%%%%%%%%%%%%%%%%%%%%%%%%%%%%%%%%%%%%%%%%%%
%%                                                                  %%
%% Packages and substitutions:                                      %%
%%                                                                  %%
%% book:     Required.                                              %%
%% inputenc: Standard DP encoding. Required.                        %%
%%                                                                  %%
%% amsmath:  AMS mathematics enhancements. Required.                %%
%% amssymb:  Additional mathematical symbols. Required.             %%
%%                                                                  %%
%% makeidx:  Indexing. Required.                                    %%
%%                                                                  %%
%% PDF pages: 88                                                    %%
%% PDF page size: US Letter (8.5 x 11in)                            %%
%%                                                                  %%
%% Summary of log file:                                             %%
%% * Two overfull hboxes (~2.6pt and ~15.2pt too wide).             %%
%%                                                                  %%
%% Command block:                                                   %%
%%                                                                  %%
%%     pdflatex x2                                                  %%
%%     makeindex                                                    %%
%%     pdflatex                                                     %%
%%                                                                  %%
%%                                                                  %%
%% April 2013: pglatex.                                             %%
%%   Compile this project with:                                     %%
%%   pdflatex 13693-t.tex ..... TWO times                           %%
%%   makeindex 13693-t.idx                                          %%
%%   pdflatex 13693-t.tex                                           %%
%%                                                                  %%
%%   pdfTeX, Version 3.1415926-1.40.10 (TeX Live 2009/Debian)       %%
%%                                                                  %%
%%%%%%%%%%%%%%%%%%%%%%%%%%%%%%%%%%%%%%%%%%%%%%%%%%%%%%%%%%%%%%%%%%%%%%
\listfiles
\documentclass[oneside]{book}
\usepackage[latin1]{inputenc}
\usepackage[reqno]{amsmath}
\usepackage{amssymb}
\usepackage{makeidx}
\makeindex
\begin{document}

\thispagestyle{empty}
\small
\begin{verbatim}
The Project Gutenberg EBook of The Theory of Numbers, by Robert D. Carmichael

This eBook is for the use of anyone anywhere at no cost and with
almost no restrictions whatsoever.  You may copy it, give it away or
re-use it under the terms of the Project Gutenberg License included
with this eBook or online at www.gutenberg.org


Title: The Theory of Numbers

Author: Robert D. Carmichael

Release Date: April 8, 2013 [EBook #13693]

Language: English

Character set encoding: TeX

*** START OF THIS PROJECT GUTENBERG EBOOK THE THEORY OF NUMBERS ***

Produced by David Starner, Joshua Hutchinson, John Hagerson,
\end{verbatim}
\normalsize
\newpage

\frontmatter

\begin{center}
\noindent \Large MATHEMATICAL MONOGRAPHS \\

\bigskip \footnotesize{EDITED BY} \\
\normalsize \textsc{MANSFIELD MERRIMAN and ROBERT S. WOODWARD.} \\

\bigskip\bigskip \huge
No. 13.

\bigskip\bigskip \huge THE THEORY \\
\bigskip\small \textsc{of}  \\
\bigskip\huge NUMBERS \\

\bigskip\bigskip\footnotesize\textsc{by} \\
\bigskip\large ROBERT D. CARMICHAEL, \\
\footnotesize\textsc{Associate Professor of Mathematics in Indiana
University}

\bigskip\bigskip\normalsize NEW YORK: \\
\medskip JOHN WILEY \& SONS. \\
\medskip \textsc{London: CHAPMAN \& HALL, Limited.} \\
\medskip 1914.

\bigskip\bigskip
\tiny \textsc{Copyright 1914} \\
\textsc{by} \\
ROBERT D. CARMICHAEL. \\
\medskip \textsc{the scientific press} \\
\textsc{robert drummond and company} \\
\textsc{brooklyn, n.~y.}
\end{center}

\bigskip\bigskip
\scriptsize \noindent \textsc{Transcriber's Note:} \emph{I did my
best to recreate the index.} \normalsize

\newpage

\fbox{\parbox{11cm}{
\begin{center}
\textbf{MATHEMATICAL MONOGRAPHS.} \\
\small\textsc{edited by}\normalsize \\
\textbf{Mansfield Merriman and Robert S. Woodward.} \\
\footnotesize \textbf{Octavo. Cloth. \$1.00 each.} \\

\bigskip \textbf{No. 1. History of Modern Mathematics.} \\
By \textsc{David Eugene Smith.}

\smallskip \textbf{No. 2. Synthetic Projective Geometry.} \\
By \textsc{George Bruce Halsted.}

\smallskip \textbf{No. 3. Determinants.} \\
By \textsc{Laenas Gifford Weld.}

\smallskip \textbf{No. 4. Hyperbolic Functions.} \\
By \textsc{James McMahon.}

\smallskip \textbf{No. 5. Harmonic Functions.} \\
By \textsc{William E. Byerly.}

\smallskip \textbf{No. 6. Grassmann's Space Analysis.} \\
By \textsc{Edward W. Hyde.}

\smallskip \textbf{No. 7. Probability and Theory of Errors.} \\
By \textsc{Robert S. Woodward.}

\smallskip \textbf{No. 8. Vector Analysis and Quaternions.} \\
By \textsc{Alexander Macfarlane.}

\smallskip \textbf{No. 9. Differential Equations.} \\
By \textsc{William Woolsey Johnson.}

\smallskip \textbf{No. 10. The Solution of Equations.} \\
By \textsc{Mansfield Merriman.}

\smallskip \textbf{No. 11. Functions of a Complex Variable.} \\
By \textsc{Thomas S. Fiske.}

\smallskip \textbf{No. 12. The Theory of Relativity.} \\
By \textsc{Robert D. Carmichael.}

\smallskip \textbf{No. 13. The Theory of Numbers.} \\
By \textsc{Robert D. Carmichael.} \normalsize

\bigskip \small PUBLISHED BY \\
\smallskip \textbf{JOHN WILEY \& SONS, Inc., NEW YORK. \\
CHAPMAN \& HALL, Limited, LONDON.}
\end{center}}}

\chapter{Editors' Preface.}

The volume called Higher Mathematics, the third edition of which was
published in 1900, contained eleven chapters by eleven authors, each
chapter being independent of the others, but all supposing the
reader to have at least a mathematical training equivalent to that
given in classical and engineering colleges. The publication of that
volume was discontinued in 1906, and the chapters have since been
issued in separate Monographs, they being generally enlarged by
additional articles or appendices which either amplify the former
presentation or record recent advances. This plan of publication was
arranged in order to meet the demand of teachers and the convenience
of classes, and it was also thought that it would prove advantageous
to readers in special lines of mathematical literature.

It is the intention of the publishers and editors to add other
monographs to the series from time to time, if the demand seems to
warrant it. Among the topics which are under consideration are those
of elliptic functions, the theory of quantics, the group theory, the
calculus of variations, and non-Euclidean geometry; possibly also
monographs on branches of astronomy, mechanics, and mathematical
physics may be included. It is the hope of the editors that this
Series of Monographs may tend to promote mathematical study and
research over a wider field than that which the former volume has
occupied.

\chapter{Preface}

The purpose of this little book is to give the reader a convenient
introduction to the theory of numbers, one of the most extensive and
most elegant disciplines in the whole body of mathematics. The
arrangement of the material is as follows: The first five chapters
are devoted to the development of those elements which are essential
to any study of the subject. The sixth and last chapter is intended
to give the reader some indication of the direction of further study
with a brief account of the nature of the material in each of the
topics suggested. The treatment throughout is made as brief as is
possible consistent with clearness and is confined entirely to
fundamental matters. This is done because it is believed that in
this way the book may best be made to serve its purpose as an
introduction to the theory of numbers.

Numerous problems are supplied throughout the text. These have been
selected with great care so as to serve as excellent exercises for
the student's introductory training in the methods of number theory
and to afford at the same time a further collection of useful
results. The exercises marked with a star are more difficult than
the others; they will doubtless appeal to the best students.

Finally, I should add that this book is made up from the material
used by me in lectures in Indiana University during the past two
years; and the selection of matter, especially of exercises, has
been based on the experience gained in this way.

\hfill \textsc{R.~D.\ Carmichael.}

\tableofcontents

%% CHAPTER I. ELEMENTARY PROPERTIES OF INTEGERS
%%  1. Fundamental Notions and Laws
%%  2. Definition of Divisibility. The Unit
%%  3. Prime Numbers. The Sieve of Eratosthenes
%%  4. The Number of Primes is Infinite
%%  5. The Fundamental Theorem of Euclid
%%  6. Divisibility by a Prime Number
%%  7. The Unique Factorization Theorem
%%  8. The Divisors of an Integer
%%  9. The Greatest Common Factor of Two or More Integers
%% 10. The Least Common Multiple of Two or More Integers
%% 11. Scales of Notation
%% 12. Highest Power of a Prime $p$ Contained in $n!$
%% 13. Remarks Concerning Prime Numbers
%%
%% CHAPTER II. ON THE INDICATOR OF AN INTEGER
%% 14. Definition. Indicator of a Prime Power
%% 15. The Indicator of a Product
%% 16. The Indicator of Any Positive Integer
%% 17. Sum of the Indicators of the Divisors of a Number
%%
%% CHAPTER III. ELEMENTARY PROPERTIES OF CONGRUENCES
%% 18. Congruences Modulo $m$
%% 19. Solutions of Congruences by Trial
%% 20. Properties of Congruences Relative to Division
%% 21. Congruences with a Prime Modulus
%% 22. Linear Congruences
%%
%% CHAPTER IV. THE THEOREMS OF FERMAT AND WILSON
%% 23. Fermat's General Theorem
%% 24. Euler's Proof of the Simple Fermat Theorem
%% 25. Wilson's Theorem
%% 26. The Converse of Wilson's Theorem
%% 27. Impossibility of $1\cdot 2\cdot 3\cdot \ldots \cdot
%%       \overline{n-1}+1=n^k, n>5$
%% 28. Extension of Fermat's Theorem
%% 29. On the Converse of Fermat's Simple Theorem
%% 30. Application of Previous Results to Linear Congruences
%% 31. Application of the Preceding Results to the Theory of
%%       Quadratic Residues
%%
%% CHAPTER V. PRIMITIVE ROOTS MODULO $m$
%% 32. Exponent of an Integer Modulo $m$
%% 33. Another Proof of Fermat's General Theorem
%% 34. Definition of Primitive Roots
%% 35. Primitive Roots Modulo $p$
%% 36. Primitive Roots Modulo $p^\alpha$, $p$ an Odd Prime
%% 37. Primitive Roots Modulo $2p^\alpha$, $p$ an Odd Prime
%% 38. Recapitulation
%% 39. Primitive $\lambda$-Roots
%%
%% CHAPTER VI. OTHER TOPICS
%% 40. Introduction
%% 41. Theory of Quadratic Residues
%% 42. Galois Imaginaries
%% 43. Arithmetic Forms
%% 44. Analytical Theory of Numbers
%% 45. Diophantine Equations
%% 46. Pythagorean Triangles
%% 47. The Equation $x^n+y^n = z^n$

\mainmatter

\chapter{ELEMENTARY PROPERTIES OF INTEGERS}
\section{Fundamental Notions and Laws}\label{s1}%
\index{Fundamental notions}

In the present chapter we are concerned primarily with certain
elementary properties of the positive integers 1, 2, 3, 4, \ldots It
will sometimes be convenient, when no confusion can arise, to employ
the word \emph{integer} or the word \emph{number} in the sense of
positive integer.

We shall suppose that the integers are already defined, either by
the process of counting or otherwise. We assume further that the
meaning of the terms \emph{greater, less, equal, sum, difference,
product} is known.

From the ideas and definitions thus assumed to be known follow
immediately the theorems:
\begin{table}[h]
\begin{tabular}{rl}
    I.\ & The sum of any two integers is an integer. \\
   II.\ & The difference of any two integers is an integer. \\
  III.\ & The product of any two integers is an integer.
\end{tabular}
\end{table}

Other fundamental theorems, which we take without proof, are
embodied in the following formulas:
\begin{table}[h]
\begin{tabular}{rrcl}
  IV.\ &                 $a + b$ & = & $b + a$.                  \\
   V.\ &            $a \times b$ & = & $b \times a$.             \\
  VI.\ &           $(a + b) + c$ & = & $a + (b + c)$.            \\
 VII.\ & $(a \times b) \times c$ & = & $a \times (b \times c)$.  \\
VIII.\ &      $a \times (b + c)$ & = & $a \times b + a \times c$.
\end{tabular}
\end{table}
Here $a$, $b$, $c$ denote any positive integers.

\newpage
These formulas are equivalent in order to the following five
theorems: addition is commutative; multiplication is commutative;
addition is associative; multiplication is associative;
multiplication is distributive with respect to addition.

\begin{center}
EXERCISES
\end{center}

\small \begin{enumerate}
\item[1.] Prove the following relations:
\begin{align*}
          1 + 2 + 3 \ldots + n &= \frac{n(n+1)}{2} \\
 1 + 3 + 5 + \ldots + (2n - 1) &= n^2,             \\
1^3 + 2^3 + 3^3 + \ldots + n^3 &= \left(\frac{n(n+1)}{2}\right)^2
  = (1+2+\ldots+n)^2.
\end{align*}

\item[2.] Find the sum of each of the following series:
\begin{align*}
1^2 + 2^2 + 3^2 + &\ldots + n^2,        \\
1^2 + 3^2 + 5^2 + &\ldots + (2n - 1)^2, \\
1^3 + 3^3 + 5^3 + &\ldots + (2n - 1)^3.
\end{align*}

\item[3.] Discover and establish the law suggested by the equations
$1^2 = 0 + 1$, $2^2 = 1 + 3$, $3^2 = 3 + 6$, $4^2 = 6 + 10$,
$\ldots$; by the equations $1 = 1^3$, $3 + 5 = 2^3$, $7 + 9 + 11 =
3^3$, $13 + 15 + 17 + 19 = 4^3$, $\ldots$.
\end{enumerate} \normalsize

\section{Definition of Divisibility. The Unit}\label{s2}%
\index{Divisibility}\index{Unit}

\textsc{Definitions.} An integer $a$ is said to be divisible by an
integer $b$ if there exists an integer $c$ such that $a = bc$. It is
clear from this definition that $a$ is also divisible by $c$. The
integers $b$ and $c$ are said to be divisors or factors of $a$; and
$a$ is said to be a multiple of $b$ or of $c$. The process of
finding two integers $b$ and $c$ such that $bc$ is equal to a given
integer $a$ is called the process of resolving $a$ into factors or
of factoring $a$; and $a$ is said to be resolved into factors or to
be factored.

We have the following fundamental theorems:

\smallskip I.~\emph{If $b$ is a divisor of $a$ and $c$ is a divisor
of $b$, then $c$ is a divisor of $a$.}

Since $b$ is a divisor of a there exists an integer $\beta$ such
that $a = b\beta$. Since $c$ is a divisor of $b$ there exists an
integer $\gamma$ such that $b = c\gamma$. Substituting this value of
$b$ in the equation $a = b\gamma$ we have $a = c\gamma\beta$. But
from theorem III of \S~\ref{s1} it follows that $\gamma\beta$ is an
integer; hence, $c$ is a divisor of $a$, as was to be proved.

\smallskip II.~\emph{If $c$ is a divisor of both $a$ and $b$, then
$c$ is a divisor of the sum of $a$ and $b$.}

From the hypothesis of the theorem it follows that integers $\alpha$
and $\beta$ exist such that
\begin{gather*}
a = c\alpha,\quad b = c\beta. \\
\intertext{Adding, we have}
a + b = c\alpha + c\beta = c(\alpha + \beta) = c\delta,
\end{gather*}
where $\delta$ is an integer. Hence, $c$ is a divisor of $a+b$.

\smallskip III.~\emph{If $c$ is a divisor of both $a$ and $b$, then
$c$ is a divisor of the difference of $a$ and $b$.}

The proof is analogous to that of the preceding theorem.

\smallskip \textsc{Definitions.} If $a$ and $b$ are both divisible
by $c$, then $c$ is said to be a common divisor or a common factor
of $a$ and $b$. Every two integers have the common factor 1. The
greatest integer which divides both $a$ and $b$ is called the
greatest common divisor of $a$ and $b$. More generally, we define in
a similar way a common divisor and the greatest common divisor of
$n$ integers $a_1$, $a_2$, $\ldots$, $a_n$.\index{Common!divisors}

\smallskip \textsc{Definitions.} If an integer $a$ is a multiple of
each of two or more integers it is called a common multiple of these
integers. The product of any set of integers is a common multiple of
the set. The least integer which is a multiple of each of two or
more integers is called their least common multiple.%
\index{Common!multiples}

It is evident that the integer $1$ is a divisor of every integer and
that it is the only integer which has this property. It is called
the unit.

\smallskip \textsc{Definition.} Two or more integers which have no
common factor except $1$ are said to be prime to each other or to be
relatively prime.\index{Relatively prime}

\smallskip \textsc{Definition.} If a set of integers is such that no
two of them have a common divisor besides $1$ they are said to be
prime each to each.\index{Prime each to each}

\begin{center}
EXERCISES
\end{center}

\small \begin{enumerate}
\item[1.] Prove that $n^3 - n$ is divisible by $6$ for every
positive integer $n$.

\item[2.] If the product of four consecutive integers is increased by
$1$ the result is a square number.

\item[3.] Show that $2^{4n + 2} + 1$ has a factor different from itself
and $1$ when $n$ is a positive integer.
\end{enumerate} \normalsize

\section{Prime Numbers. The Sieve of Eratosthenes}\label{s3}%
\index{Eratosthenes}\index{Sieve of Eratosthenes}

\textsc{Definition.} If an integer $p$ is different from 1 and has
no divisor except itself and 1 it is said to be a prime number or to
be a prime.

\smallskip \textsc{Definition.} An integer which has at least one
divisor other than itself and 1 is said to be a composite number or
to be composite.

All integers are thus divided into three classes:
\begin{table}[h]
\begin{tabular}{rl}
1.\ & The unit; \\
2.\ & Prime numbers; \\
3.\ & Composite numbers.
\end{tabular}
\end{table}\index{Composite numbers}\index{Prime numbers}

We have seen that the first class contains only a single number. The
third class evidently contains an infinitude of numbers; for, it
contains all the numbers $2^2, 2^3, 2^4, \ldots$ In the next section
we shall show that the second class also contains an infinitude of
numbers. We shall now show that every number of the third class
contains one of the second class as a factor, by proving the
following theorem:

\smallskip I.~\emph{Every integer greater than 1 has a prime factor.}

Let $m$ be any integer which is greater than 1. We have to show that
it has a prime factor. If $m$ is prime there is the prime factor $m$
itself. If $m$ is not prime we have
\begin{equation*}
m = m_1 m_2
\end{equation*}
where $m_1$ and $m_2$ are positive integers both of which are less
than $m$. If either $m_1$ or $m_2$ is prime we have thus obtained a
prime factor of $m$. If neither of these numbers is prime, then
write
\begin{equation*}
m_1 = m'_1 m'_2,\quad m'_1 > 1, m'_2 > 1.
\end{equation*}
Both $m'_1$ and $m'_2$ are factors of $m$ and each of them is less
than $m_1$. Either we have not found in $m'_1$ or $m'_2$ a prime
factor of $m$ or the process can be continued by separating one of
these numbers into factors. Since for any given $m$ there is
evidently only a finite number of such steps possible, it is clear
that we must finally arrive at a prime factor of $m$. From this
conclusion, the theorem follows immediately.

Eratosthenes has given a useful means of finding the prime numbers
which are less than any given integer $m$. It may be described as
follows:

Every prime except 2 is odd. Hence if we write down every odd number
from 3 up to $m$ we shall have it the list every prime less than $m$
except 2. Now 3 is prime. Leave it in the list; but beginning to
count from 3 strike out every third number in the list. Thus every
number divisible by 3, except 3 itself, is cancelled. Then begin
from 5 and cancel every fifth number. Then begin from from the next
uncancelled number, namely 7, and strike out every seventh number.
Then begin from the next uncancelled number, namely 11, and strike
out every eleventh number. Proceed in this way up to $m$. The
uncancelled numbers remaining will be the odd primes not greater
than $m$.

It is obvious that this process of cancellation need not be carried
altogether so far as indicated; for if $p$ is a prime greater than
$\sqrt{m}$, the cancellation of any $p^\text{th}$ number from $p$
will be merely a repetition of cancellations effected by means of
another factor smaller than $p$, as one my see by the use of the
following theorem.

\smallskip II.~\emph{An integer $m$ is prime if it has no prime
factor equal or less than $I$, where $I$ is the greatest integer
whose square is equal to or less than $m$.}

Since $m$ has no prime factor less than $I$, it follows from theorem
I that is has no factor but unity less than $I$. Hence, if $m$ is
not prime it must be the product of two numbers each greater than
$I$; and hence it must be equal to or greater than $(I+1)^2$. This
contradicts the hypothesis on $I$; and hence we conclude that $m$ is
prime.

\begin{center}
EXERCISE
\end{center}

\small \begin{enumerate}
\item[ ] By means of the method of Eratosthenes determine the primes
less than 200.
\end{enumerate}
\normalsize

\section{The Number of Primes is Infinite}\label{s4}%
\index{Prime numbers}

I.~\emph{The number of primes is infinite.}

We shall prove this theorem by supposing that the number of primes
is not infinite and showing that this leads to a contradiction. If
the number of primes is not infinite there is a greatest prime
number, which we shall denote by $p$. Then form the number
\begin{equation*}
N = 1 \cdot 2 \cdot 3 \cdot \ldots \cdot p + 1.
\end{equation*}
Now by theorem 1 of \S~\ref{s3} $N$ has a prime divisor $q$. But
every non-unit divisor of $N$ is obviously greater than $p$. Hence
$q$ is greater than $p$, in contradiction to the conclusion that $p$
is the greatest prime. Thus the proof of the theorem is complete.

In a similar way we may prove the following theorem:

\smallskip II.~\emph{Among the integers of the arithmetic
progression $5$, $11$, $17$, $23$, $\ldots$, there is an infinite
number of primes.}

If the number of primes in this sequence is not infinite there is a
greatest prime number in the sequence; supposing that this greatest
prime number exists we shall denote it by $p$. Then the number $N$,
\begin{equation*}
N = 1 \cdot 2 \cdot 3 \cdot \ldots \cdot p-1,
\end{equation*}
is not divisible by any number less than or equal to $p$. This
number $N$, which is of the form $6n - 1$, has a prime factor. If
this factor is of the form $6k - 1$ we have already reached a
contradiction, and our theorem is proved. If the prime is of the
form $6k_1 + 1$ the complementary factor is of the form $6k_2 - 1$.
Every prime factor of $6k_2 - 1$ is greater than $p$. Hence we may
treat $6k_2 - 1$ as we did $6n - 1$, and with a like result. Hence
we must ultimately reach a prime factor of the form $6k_3 - 1$; for,
otherwise, we should have $6n - 1$ expressed as a product of prime
factors all of the form $6t + 1$---a result which is clearly
impossible. Hence we must in any case reach a contradiction of the
hypothesis. Thus the theorem is proved.

The preceding results are special cases of the following more
general theorem:

\smallskip III.~\emph{Among the integers of the arithmetic
progression $a$, $a + d$, $a + 2d$, $a + 3d$, $\ldots$, there is an
infinite number of
primes, provided that $a$ and $b$ are relatively prime.}%
\index{Arithmetic progression}

For the special case given in theorem II we have an elementary
proof; but for the general theorem the proof is difficult. We shall
not give it here.

\begin{center}
EXERCISES
\end{center}

\small \begin{enumerate}

\item[1.] Prove that there is an infinite number of primes of the
form $4n - 1$.

\item[2.] Show that an odd prime number can be represented as the
difference of two squares in one and in only one way.

\item[3.] The expression $m^p - n^p$, in which $m$ and $n$ are integers
and $p$ is a prime, is either prime to $p$ or is divisible by $p^2$.

\item[4.] Prove that any prime number except $2$ and $3$ is of one of
the forms $6n + 1$, $6n - 1$.
\end{enumerate}\normalsize

\section{The Fundamental Theorem of Euclid}\label{s5}%
\index{Euclid, Theorem of}

\emph{If $a$ and $b$ are any two positive integers there exist
integers $q$ and $r$, $q\stackrel{=}{>} 0, 0 \leqq r < b$, such
that}
\begin{equation*}
a = qb + r.
\end{equation*}

If $a$ is a multiple of $b$ the theorem is at once verified, $r$
being in this case $0$. If $a$ is not a multiple of $b$ it must lie
between two consecutive multiples of $b$; that is, there exists a
$q$ such that
\begin{equation*}
qb < a < (q + 1)b.
\end{equation*}
Hence there is an integer $r$, $0 < r < b$, such that $a = qb + r$.
In case $b$ is greater than $a$ it is evident that $q = 0$ and $r =
a$. Thus the proof of the theorem is complete.

\section{Divisibility by a Prime Number}\label{s6}\index{Prime numbers}

I.~\emph{If $p$ is a prime number and $m$ is any integer, then $m$
either is divisible by $p$ or is prime to $p$.}

This theorem follows at once from the fact that the only divisors of
$p$ are $1$ and $p$.

\smallskip II.~\emph{The product of two integers each less than a
given prime number $p$ is not divisible by $p$.}

Let $a$ be a number which is less than $p$ and suppose that $b$ is a
number less than $p$ such that $ab$ is divisible by $p$, and let $b$
be the least number for which $ab$ is so divisible. Evidently there
exists an integer $m$ such that
\begin{equation*}
mb < p < (m + 1)b.
\end{equation*}
Then $p - mb < b$. Since $ab$ is divisible by $p$ it is clear that
$mab$ is divisible by $p$; so is $ap$ also; and hence their
difference $ap - mab$, $=a(p - mb)$, is divisible by $p$. That is,
the product of $a$ by an integer less than $b$ is divisible by $p$,
contrary to the assumption that $b$ is the least integer such that
$ab$ is divisible by $p$. The assumption that the theorem is not
true has thus led to a contradiction; and thus the theorem is
proved.

\smallskip III.~\emph{If neither of two integers is divisible by a
given prime number $p$ their product is not divisible by $p$.}

Let $a$ and $b$ be two integers neither of which is divisible by the
prime $p$. According to the fundamental theorem of Euclid there
exist integers $m$, $n$, $\alpha$, $\beta$ such that
\begin{align*}
a &= mp + \alpha,& 0 &< \alpha < p, \\
b &= np + \beta, & 0 &< \beta < p.
\end{align*}
Then
\begin{equation*}
ab = (mp  + \alpha)(np + \beta)
   = (mnp + \alpha + \beta)p + \alpha\beta.
\end{equation*}
If now we suppose $ab$ to be divisible by $p$ we have $\alpha\beta$
divisible by $p$. This contradicts II, since $\alpha$ and $\beta$
are less than $p$. Hence $ab$ is not divisible by $p$.

By an application of this theorem to the continued product of
several factors, the following result is readily obtained:

\smallskip IV.~\emph{If no one of several integers is divisible by a
given prime $p$ their product is not divisible by $p$.}

\section{The Unique Factorization Theorem}\label{s7}%
\index{Factorization theorem}\index{Factors}

I.~\emph{Every integer greater than unity can be represented in one
and in only one way as a product of prime numbers.}

In the first place we shall show that it is always possible to
resolve a given integer $m$ greater than unity into prime factors by
a finite number of operations. In the proof of theorem I,
\S~\ref{s3}, we showed how to find a prime factor $p_1$ of $m$ by a
finite number of operations. Let us write
\begin{equation*}
m = p_1 m_1.
\end{equation*}
If $m_1$ is not unity we may now find a prime factor $p_2$ of $m_1$.
Then we may write
\begin{equation*}
m = p_1 m_1 = p_1 p_2 m_2.
\end{equation*}
If $m_2$ is not unity we may apply to it the same process as that
applied to $m_1$ and thus obtain a third prime factor of $m$. Since
$m_1 > m_2 > m_3 > \ldots$ it is clear that after a finite number of
operations we shall arrive at a decomposition of $m$ into prime
factors. Thus we shall have
\begin{equation*}
m = p_1 p_2 \ldots p_r
\end{equation*}
where $p_1$, $p_2$, $\ldots$, $p_r$ are prime numbers. We have thus
proved the first part of our theorem, which says that the
decomposition of an integer (greater than unity) into prime factors
is always possible.

Let us now suppose that we have also a decomposition of $m$ into
prime factors as follows:
\begin{gather*}
m = q_1 q_2 \ldots q_s. \\
\intertext{Then we have}
p_1 p_2 \ldots p_r = q_1 q_2 \ldots q_s.
\end{gather*}
Now $p_1$ divides the first member of this equation. Hence it also
divides the second member of the equation. But $p_1$ is prime; and
therefore by theorem IV of the preceding section we see that $p_1$
divides some one of the factors $q$; we suppose that $p_1$ is a
factor of $q_1$. It must then be equal to $q_1$. Hence we have
\begin{equation*}
p_2 p_3 \ldots p_r = q_2 q_3 \ldots q_s.
\end{equation*}
By the same argument we prove that $p_2$ is equal to some $q$, say
$q_2$. Then we have
\begin{equation*}
p_3 p_4 \ldots p_r = q_3 q_4 \ldots q_s.
\end{equation*}
Evidently the process may be continued until one side of the
equation is reduced to $1$. The other side must also be reduced to
$1$ at the same time. Hence it follows that the two decompositions
of $m$ are in fact identical.

This completes the proof of the theorem.

\smallskip The result which we have thus demonstrated is easily the
most important theorem in the theory of integers. It can also be
stated in a different form more convenient for some purposes:

\smallskip II.~\emph{Every non-unit positive integer $m$ can be
represented in one and in only one way in the form
\begin{equation*}
m = p_1^{\alpha_1} p_2^{\alpha_2} \ldots p_n^{\alpha_n}
\end{equation*}
where $p_1$, $p_2$, $\ldots$, $p_n$ are different primes and
$\alpha_1$, $\alpha_2$, $\ldots$, $\alpha_n$ are positive integers.}%
\index{Factors}

This comes immediately from the preceding representation of $m$ in
the form $m = p_1 p_2 \ldots p_r$ by combining into a power of $p_1$
all the primes which are equal to $p_1$.

\smallskip \textsc{Corollary 1.}~\emph{If $a$ and $b$ are relatively
prime integers and $c$ is divisible by both $a$ and $b$, then $c$ is
divisible by $ab$.}

\smallskip \textsc{Corollary 2.}~\emph{If $a$ and $b$ are each prime
to $c$ then $ab$ is prime to $c$.}

\smallskip \textsc{Corollary 3.}~\emph{If $a$ is prime to $c$ and
$ab$ is divisible by $c$, then $b$ is divisible by $c$.}

\section{The Divisors of an Integer}\label{s8}%
\index{Divisors of a number|(}\index{Factors}

The following theorem is an immediate corollary of the results in
the preceding section:

I.~\emph{All the divisors of $m$,
\begin{gather*}
m = p_1^{\alpha_1} p_2^{\alpha_2} \ldots p_n^{\alpha_n}, \\
\intertext{are of the form}
p_1^{\beta_1} p_2^{\beta_2} \ldots p_n^{\beta_n},\
   0 \leqq \beta_i \leqq \alpha_i;
\end{gather*}
and every such number is a divisor of $m$.}

From this it is clear that every divisor of $m$ is included once and
only once among the terms of the product
\begin{multline*}
(1 + p_1 + p_1^2 + \ldots + p_1^{\alpha_1})(1 + p_2 + p_2^2 + \ldots
  + p_2^{\alpha_2}) \ldots \\
(1 + p_n + p_n^2 + \ldots + p_n^{\alpha_n}),
\end{multline*}
when this product is expanded by multiplication. It is obvious that
the number of terms in the expansion is $(\alpha_1 + 1)(\alpha_2 +
1) \ldots (\alpha_n+1)$. Hence we have the theorem:

\smallskip II.~\emph{The number of divisors of $m$ is}
$(\alpha_1 + 1) (\alpha_2 + 1) \ldots (\alpha_n+1)$.

Again we have
\begin{equation*}
\prod_i(1 + p_i + p_i^2 + \ldots + p_i^{\alpha_i}) =
  \prod_i\frac{p_i^{\alpha_i+1} - 1}{p_i - 1}.
\end{equation*}
Hence,

\smallskip III.~\emph{The sum of the divisors of $m$ is}
\begin{equation*}
\frac{p_1^{\alpha_1 + 1} - 1}{p_1 - 1} \cdot
    \frac{p_2^{\alpha_2 + 1} - 1}{p_2 - 1} \cdot
    \ldots \cdot
    \frac{p_i^{\alpha_i + 1} - 1}{p_i - 1}.
\end{equation*}

In a similar manner we may prove the following theorem:

\smallskip IV.~\emph{The sum of the $h^{th}$ powers of the divisors
of $m$ is}
\begin{equation*}
\frac{p_1^{h(\alpha_1 + 1)} - 1}{p_1^h - 1} \cdot
    \ldots \cdot
    \frac{p_n^{h(\alpha_n + 1)} - 1}{p_n^h - 1}.
\end{equation*}

\begin{center}
EXERCISES
\end{center}

\small \begin{enumerate}
\item[1.] Find numbers $x$ such that the sum of the divisors of $x$
is a perfect square.

\item[2.] Show that the sum of the divisors of each of the following
integers is twice the integer itself: 6, 28, 496, 8128, 33550336.
Find other integers $x$ such that the sum of the divisors of $x$ is
a multiple of $x$.

\item[3.] Prove that the sum of two odd squares cannot be a square.

\item[4.] Prove that the cube of any integer is the difference of the
squares of two integers.

\item[5.] In order that a number shall be the sum of consecutive
integers, it is necessary and sufficient that it shall not be a
power of 2.

\item[6.] Show that there exist no integers $x$ and $y$ (zero excluded)
such that $y^2 = 2x^2$. Hence, show that there does not exist a
rational fraction whose square is 2.

\item[7.] The number $m = p_1^{\alpha_1} p_2^{\alpha_2} \cdots
p_n^{\alpha_n}$, where the $p$'s are different primes and the
$\alpha$'s are positive integers, may be separated into relatively
prime factors in $2^{n-1}$ different ways.

\item[8.] The product of the divisors of $m$ is $\sqrt{m^v}$ where $v$
is the number of divisors of $m$.
\end{enumerate} \normalsize\index{Divisors of a number|)}

\section{The Greatest Common Factor of Two or More
Integers}\label{s9}%
\index{Common!divisors|(}\index{Factors}%
\index{Greatest common factor|(}

Let $m$ and $n$ be two positive integers such that $m$ is greater
than $n$. Then, according to the fundamental theorem of Euclid, we
can form the set of equations
\begin{align*}
m         &= qn + n_1,                & 0 &< n_1 < n,   \\
n         &= q_1 n_1 + n_2,           & 0 &< n_2 < n_1, \\
n_1       &= q_2 n_2 + n_3,           & 0 &< n_3 < n_2, \\
&\vdots \qquad \vdots &&\vdots \qquad \vdots \\
n_{k - 2} &= q_{k - 1} n_{k-1} + n_k, & 0 &< n_k < n_{k - 1}, \\
n_{k - 1} &= q _k n_k.                &   &
\end{align*}
If $m$ is a multiple of $n$ we write $n = n_0$, $k = 0$, in the
above equations.

\smallskip \textsc{Definition.} The process of reckoning involved in
determining the above set of equations is called the Euclidian
Algorithm.\index{Euclidian algorithm}

\smallskip I.~\emph{The number $n_k$ to which the Euclidian
algorithm leads is the greatest common divisor of $m$ and $n$.}

In order to prove this theorem we have to show two things:

1)~That $n_k$ is a divisor of both $m$ and $n$;

2)~That the greatest common divisor $d$ of $m$ and $n$ is a divisor
of $n_k$.

To prove the first statement we examine the above set of equations,
working from the last to the first. From the last equation we see
that $n_k$ is a divisor of $n_{k-1}$. Using this result we see that
the second member of next to the last equation is divisible by $n_k$
Hence its first member $n_{k-2}$ must be divisible by $n_k$.
Proceeding in this way step by step we show that $n_2$ and $n_1$,
and finally that $n$ and $m$, are divisible by $n_k$.

For the second part of the proof we employ the same set of equations
and work from the first one to the last one. Let $d$ be any common
divisor of $m$ and $n$. From the first equation we see that $d$ is a
divisor of $n_1$. Then from the second equation it follows that $d$
is a divisor of $n_2$. Proceeding in this way we show finally that
$d$ is a divisor of $n_k$. Hence any common divisor, and in
particular the greatest common divisor, of $m$ and $n$ is a factor
of $n_k$.

This completes the proof of the theorem.

\smallskip \textsc{Corollary.} \emph{Every common divisor of $m$ and
$n$ is a factor of their greatest common divisor.}

\smallskip II.~\emph{Any number $n_i$ in the above set of equations
is the difference of multiples of $m$ and $n$.}

From the first equation we have
\begin{equation*}
n_i = m - qn
\end{equation*}
so that the theorem is true for $i = 1$. We shall suppose that the
theorem is true for every subscript up to $i - 1$ and prove it true
for the subscript $i$. Thus by hypothesis we have\footnote{If $i =
2$ we must replace $n_{i-2}$ by $n$.}
\begin{align*}
n_{i-2} &= \pm(\alpha_{i-2}m - \beta_{i-2}n ),  \\
n_{i-1} &= \mp(\alpha_{i-1}m - \beta_{i-1}n).
\intertext{Substituting in the equation}
n_i &= -q_{i-1}n_{n-1} + n_{i-2} \\
\intertext{we have a result of the form}
n_i &= \pm (\alpha_i m - \beta_i n).
\end{align*}
From this we conclude at once to the truth of the theorem.

Since $n_k$ is the greatest common divisor of $m$ and $n$, we have
as a corollary the following important theorem:

\smallskip III.~\emph{If $d$ is the greatest common divisor of the
positive integers $m$ and $n$, then there exist positive integers
$\alpha$ and $\beta$ such that}
\begin{equation*}
\alpha m - \beta n = \pm d.
\end{equation*}

If we consider the particular case in which $m$ and $n$ are
relatively prime, so that $d = 1$, we see that there exist positive
integers $\alpha$ and $\beta$ such that $\alpha m - \beta n = \pm
1$. Obviously, if $m$ and $n$ have a common divisor $d$, greater
than $1$, there do not exist integers $\alpha$ and $\beta$
satisfying this relation; for, if so, $d$ would be a divisor of the
first member of the equation and not of the second. Thus we have the
following theorem:

\smallskip IV.~\emph{A necessary and sufficient condition that $m$
and $n$ are relatively prime is that there exist integers $\alpha$
and $\beta$ such that $\alpha m - \beta n = \pm 1$.}

The theory of the greatest common divisor of three or more numbers
is based directly on that of the greatest common divisor of two
numbers; consequently it does not require to be developed in detail.

\begin{center}
EXERCISES
\end{center}

\small \begin{enumerate}
\item[1.] If $d$ is the greatest common divisor of $m$ and $n$,
then $m / d$ and $n / d$ are relatively prime.

\item[2.] If $d$ is the greatest common divisor of $m$ and $n$ and
$k$ is prime to $n$, then $d$ is the greatest common divisor of $km$
and $n$.

\item[3.] The number of multiplies of $6$ in the sequence $a, 2a, 3a,
\cdots, ba$ is equal to the greatest common divisor of $a$ and $b$.

\item[4.] If the sum or the difference of two irreducible fractions is
an integer, the denominators of the fractions are equal.

\item[5.] The algebraic sum of any number of irreducible fractions,
whose denominators are prime each to each, cannot be an integer.

\item[6*.] The number of divisions to be effected in finding the
greatest common divisor of two numbers by the Euclidian algorithm
does not exceed five times the number of digits in the smaller
number (when this number is written in the usual scale of 10).
\end{enumerate}\normalsize%
\index{Common!divisors|)}\index{Greatest common factor|)}

\section{The Least Common Multiple of Two or More
Integers}\label{s10}%
\index{Common!multiples|(}\index{Least common multiple|(}

I.~\emph{The common multiples of two or more numbers are the
multiples of their least common multiple.}

This may be readily proved by means of the unique factorization
theorem. The method is obvious. We shall, however, give a proof
independent of this theorem.

Consider first the case of two numbers; denote them by $m$ and $n$
and their greatest common divisor by $d$. Then we have
\begin{equation*}
m = d\mu, \quad n = d\nu,
\end{equation*}
where $\mu$ and $\nu$ are relatively prime
integers.\index{Common!divisors}\index{Greatest common factor} The
common multiples sought are multiples of $m$ and are all comprised
in the numbers $am=ad\mu$, where $a$ is any integer whatever. In
order that these numbers shall be multiples of $n$ it is necessary
and sufficient that $ad\mu$ shall be a multiple of $d\nu$; that is,
that $a\mu$ shall be a multiple of $\nu$; that is, that $a$ shall be
a multiple of $\nu$, since $\mu$ and $\nu$ are relatively prime.
Writing $a = \delta\nu$ we have as the multiples in question the set
$\delta d\mu\nu$ where $\delta$ is an arbitrary integer. This proves
the theorem for the case of two numbers; for $d\mu\nu$ is evidently
the least common multiple of $m$ and $n$.

We shall now extend the proposition to any number of integers $m, n,
p, q,\ldots$. The multiples in question must be common multiples of
$m$ and $n$ and hence of their least common multiple $\mu$. Then the
multiples must be multiples of $\mu$ and $p$ and hence of their
least common multiple $\mu_1$. But $\mu_1$ is evidently the least
common multiple of $m, n, p$. Continuing in a similar manner we may
show that every multiple in question is a multiple of $\mu$, the
least common multiple of $m, n, p, q, \ldots$. And evidently every
such number is a multiple of each of the numbers $m, n, p, q,
\ldots$.

Thus the proof of the theorem is complete.

When the two integers $m$ and $n$ are relatively prime their
greatest common divisor is $1$ and their least common multiple is
their product. Again if $p$ is prime to both $m$ and $n$ it is prime
to their product $mn$; and hence the least common multiple of $m, n,
p$ is in this case $mnp$. Continuing in a similar manner we have the
theorem:

\smallskip II.~\emph{The least common multiple of several integers,
prime each to each, is equal to their product.}

\newpage
\begin{center}
EXERCISES
\end{center}

\small \begin{enumerate}
\item[1.] In order that a common multiple of $n$ numbers shall be
the least, it is necessary and sufficient that the quotients
obtained by dividing it successively by the numbers shall be
relatively prime.

\item[2.] The product of $n$ numbers is equal to the product of
their least common multiple by the greatest common divisor of their
products $n - 1$ at a time.

\item[3.] The least common multiple of $n$ numbers is equal to any
common multiple $M$ divided by the greatest common divisor of the
quotients obtained on dividing this common multiple by each of the
numbers.

\item[4.] The product of $n$ numbers is equal to the product of their
greatest common divisor by the least common multiple of the products
of the numbers taken $n - 1$ at a time.
\end{enumerate} \normalsize%
\index{Common!multiples|)}\index{Least common multiple|)}

\section{Scales of Notation}\label{s11}\index{Scales of notation|(}

I.~\emph{If $m$ and $n$ are positive integers and $n > 1$, then $m$
can be represented in terms of $n$ in one and in only one way in the
form}
\begin{gather*}
m = a_0 n^h + a_1 n^{h-1} + \ldots + a_{h-1} n + a_h, \\
\intertext{where}
a_0 \ne 0,\ 0 \leqq a_i < n, \quad i = 0, 1, 2, \ldots, h.
\end{gather*}

That such a representation of $m$ exists is readily proved by means
of the fundamental theorem of Euclid. For we have
\begin{align*}
m       &= n_0 n + a_h,     & 0 &\leqq a_h < n,     \\
n_0     &= n_1n + a_{h-1},  & 0 &\leqq a_{h-1} < n, \\
n_1     &= n_2 n + a_{h-2}, & 0 &\leqq a_{h-2} < n, \\
\hdots\hdots & \hdots\hdots\hdots\hdots\hdots\hdots &
  \hdots\hdots & \hdots\hdots\hdots\hdots\hdots\hdots \\
n_{h-3} &= n_{h-2} n + a_2, & 0 &\leqq a_2 < n, \\
n_{h-2} &= n_{h-1} n + a_1, & 0 &\leqq a_1 < n, \\
n_{h-1} &= a_0,             & 0 &<   a_0 < n.
\end{align*}
If the value of $n_{h-1}$ given in the last of these equations is
substituted in the second last we have
\begin{equation*}
n_{h-2} = a_0n + a_1.
\end{equation*}
This with the preceding gives
\begin{equation*}
n_{h-3} = a_0 n^2 + a_1n + a_2.
\end{equation*}
Substituting from this in the preceding and continuing the process
we have finally
\begin{equation*}
m = a_0 n^h + a_1 n^{h-1} + \ldots + a_{h-1}n + a_h,
\end{equation*}
a representation of $m$ in the form specified in the theorem.

To prove that this representation is unique, we shall suppose that
$m$ has the representation
\begin{gather*}
m = b_0 n^k + b_1 n^{k-1} + \ldots + b_{k-1}n + b_k, \\
\intertext{where}
b_0 \ne 0,\ 0 < b_i < n,\quad i=0, 1, 2, \ldots, k, \\
\intertext{and show that the two representations are identical. We
have}
a_0 n^h + \ldots + a_{h-1} n + a_h =
  b_0 n^k + \ldots + b_{k-1} n + b_k.
\intertext{Then}
a_0 n^h + \ldots + a_{h-1} n -
   (b_0 n^k + \ldots + b_{k-1} n) = b_k - a_h.
\end{gather*}
The first member is divisible by $n$. Hence the second is also. But
the second member is less than $n$ in absolute value; and hence, in
order to be divisible by $n$, it must be zero. That is, $b_k = a_h$.
Dividing the equation through by $n$ and transposing we have
\begin{equation*}
a_0 n^{h-1} + \ldots + a_{h-2} n - (b_0 n^{k-1} + \ldots +
       b_{k-2} n)
   = b_{k-1} - a_{h-1}.
\end{equation*}
It may now be seen that $b_{k-1} = a_{h-1}$. It is evident that this
process may be continued until either the $a$'s are all eliminated
from the equation or the $b$'s are all eliminated. But it is obvious
that when one of these sets is eliminated the other is also. Hence,
$h = k$. Also, every $a$ equals the $b$ which multiplies the same
power of $n$ as the corresponding $a$. That is, the two
representations of $m$ are identical. Hence the representation in
the theorem is unique.

From this theorem it follows as a special case that any positive
integer can be represented in one and in only one way in the scale
of 10; that is, in the familiar Hindoo notation. It can also be
represented in one and in only one way in any other scale. Thus
\begin{equation*}
120759 = 1 \cdot 7^6 + 0 \cdot 7^5 + 1 \cdot 7^4 + 2 \cdot 7^3 +
   0 \cdot 7^2 + 3 \cdot 7^1 + 2.
\end{equation*}
Or, using a subscript to denote the scale of notation, this may be
written
\begin{equation*}
(120759)_{10} = (1012032)_7.
\end{equation*}

For the case in which $n$ (of theorem I) is equal to 2, the only
possible values for the $a$'s are 0 and 1. Hence we have at once the
following theorem:

II.~\emph{Any positive integer can be represented in one and in only
one way as a sum of different powers of 2.}

\newpage
\begin{center}
EXERCISES
\end{center}

\small \begin{enumerate}

\item[1.] Any positive integer can be represented as an aggregate of
different powers of $3$, the terms in the aggregate being combined
by the signs $+$ and $-$ appropriately chosen.

\item[2.] Let $m$ and $n$ be two positive integers of which $n$ is the
smaller and suppose that $2^k \leq n < 2^{k+1}$. By means of the
representation of $m$ and $n$ in the scale of 2 prove that the
number of divisions to be effected in finding the greatest common
divisor of $m$ and $n$ by the Euclidian algorithm does not exceed
$2k$.
\end{enumerate}\normalsize\index{Scales of notation|)}

\section{Highest Power of a Prime $p$ Contained in $n!$.}\label{s12}%
\index{Highest power of \emph{p} in \emph{n}"!|(}

Let $n$ be any positive integer and $p$ any prime number not greater
than $n$. We inquire as to what is the highest power $p^\nu$ of the
prime $p$ contained in $n!$.

In solving this problem we shall find it convenient to employ the
notation
\begin{equation*}
\left [ \frac{r}{s} \right ]
\end{equation*} to denote the greatest integer $\alpha$ such that
$\alpha s \leq r$. With this notation it is evident that we have
\begin{gather}
\left [
        \frac{\left [ \frac{n}{p} \right ]}
             {p}
\right ] = \left [ \frac{n}{p^2} \right ]; \tag{1} \\
\intertext{and more generally}
\left [
        \frac{\left [ \frac{n}{p^i} \right ]}
             {p^j}
\right ] = \left [ \frac{n}{p^{i+j}} \right ]. \notag
\end{gather}

If now we use $H\{x\}$ to denote the index of the highest power of
$p$ contained in an integer $x$, it is clear that we have
\begin{gather*}
H\{n!\} =
   H \left \{ p \cdot 2p \cdot 3p \ldots
          \left [ \frac{n}{p} \right ] p \right \}, \\
\intertext{since only multiples of $p$ contain the factor $p$.
Hence}
H\{n!\} =
  \left [ \frac{n}{p} \right ] +
     H \left \{ 1 \cdot 2 \ldots \left [ \frac{n}{p} \right ]
                                                           \right \}.
\end{gather*}
Applying the same process to the $H$-function in the second member
and remembering relation (1) it is easy to see that we have
\begin{align*}
H\{n!\} &= \left[ \frac{n}{p} \right] +
  H\left\{ p \cdot 2p \cdot \ldots \cdot
        \left[ \frac{n}{p^2} \right]p\right\} \\
   &= \left[\frac{n}{p}\right] + \left[\frac{n}{p^2}\right] +
         H \left\{\cdot 1 \cdot 2 \cdot 3
            \ldots \left[ \frac{n}{p^2} \right] \right\}. \\
\intertext{Continuing the process we have finally}
H\{n1\} &= \left[ \frac{n}{p} \right] +
  \left[ \frac{n}{p^2} \right] + \left[ \frac{n}{p^3} \right] +
                                                              \ldots,
\end{align*}
the series on the right containing evidently only a finite number of
terms different from zero. Thus we have the theorem:

\smallskip I.~\emph{The index of the highest power of a prime $p$
contained in $n!$ is}
\begin{gather*}
\left[ \frac{n}{p} \right] + \left[ \frac{n}{p^2} \right] +
  \left[ \frac{n}{p^3} \right] + \ldots.
\end{gather*}

The theorem just obtained may be written in a different form, more
convenient for certain of its applications. Let $n$ be expressed in
the scale of $p$ in the form
\begin{gather*}
n = a_0p^h + a_1p^{h-1} + \ldots + a_{h-1}p + a_h, \\
\intertext{where}
a_0 \neq 0,\ 0 \leqq a_i < p,\ i = 0, 1, 2, \ldots, h.
\end{gather*}
Then evidently
\begin{align*}
\left[ \frac{n}{p} \right]   &= a_0p^{h-1} + a_1p^{h-2} + \ldots +
  a_{h-2}p + a_{h-1}, \\
\left[ \frac{n}{p^2} \right] &= a_0p^{h-2} + a_1p^{h-3} + \ldots +
  a_{h-2}, \\
.\  \ .\ \ .\ \ .\ \ &.\ \ .\ \ .\ \ .\ \ .\ \ .\ \ .\ \ .\ \ .\ \
.\ \ .\ \ .\ \ .\ \ .\ \ .\ \ .\ \ .\ \ .\ \ .\ \ .\ \ .
\end{align*}
Adding these equations member by member and combining the second
members in columns as written, we have
\begin{align*}
\left[ \frac{n}{p} \right] +
  \left[ \frac{n}{p^2} \right] &+
  \left[ \frac{n}{p^3} \right] + \ldots \\
&= \sum_{i=0}^h \frac{a_i(p^{h-i} - 1)}{p - 1} \\
&= \frac{a_0p^h + a_1p^{h-1} + \ldots + a_h -
          (a_0 + a_1 + \ldots + a_h)}{p-1} \\
&= \frac{n - (a_0 + a_1 + \ldots + a_h)}{p - 1}.
\end{align*}
Comparing this result with theorem I we have the following theorem:

\smallskip II.~\emph{If $n$ is represented in the scale of $p$ in
the form
\begin{gather*}
n = a_0 p^h + a_1 p^{h-1} + \ldots + a_h, \\
\intertext{where $p$ is prime and}
a_0 \neq 0,\ 0 \leqq a_i < p,\ i = 0, 1, 2, \ldots, h, \\
\intertext{then the index of the highest power of $p$ contained in
$n!$ is}
\frac{n - (a_0 + a_1 + \ldots + a_h)}{p - 1}.
\end{gather*}}

Note the simple form of the theorem for the case $p = 2$; in this
case the denominator $p - 1$ is unity.

We shall make a single application of these theorems by proving the
following theorem:

\smallskip III.~\emph{If $n$, $\alpha$, $\beta$, $\ldots$, $\lambda$
are any positive integers such that $n = \alpha + \beta + \ldots +
\lambda$, then
\begin{equation}
\frac{n!}{\alpha! \beta! \ldots \lambda!} \tag{A}
\end{equation}
is an integer.}

Let $p$ be any prime factor of the denominator of the fraction (A).
To prove the theorem it is sufficient to show that the index of the
highest power of $p$ contained in the numerator is at least as great
as the index of the highest power of $p$ contained in the
denominator. This index for the denominator is the sum of the
expressions
\begin{equation}
  \left .
   \begin{gathered}
    \left [ \frac{\alpha}{p} \right ] +
    \left [ \frac{\alpha}{p^2} \right ] +
    \left [ \frac{\alpha}{p^3} \right ] +
    \ldots \\
    \left [ \frac{\beta}{p} \right ] +
    \left [ \frac{\beta}{p^2} \right ] +
    \left [ \frac{\beta}{p^3} \right ] +
    \ldots \\
      \vdots \\
    \left [ \frac{\lambda}{p} \right ] +
    \left [ \frac{\lambda}{p^2} \right ] +
    \left [ \frac{\lambda}{p^3} \right ] +
    \ldots
   \end{gathered}
  \right \} \tag{B}
\end{equation}

The corresponding index for the numerator is
\begin{equation}
\left [ \frac{n}{p} \right ] +
\left [ \frac{n}{p^2} \right ] +
\left [ \frac{n}{p^3} \right ] +
\ldots \tag{C}
\end{equation}
But, since $n = \alpha + \beta + \ldots + \lambda$, it is evident
that
\begin{equation*}
   \left [ \frac{n}{p^r} \right ] \stackrel{=}{>}
   \left [ \frac{\alpha}{p^r} \right ] +
   \left [ \frac{\beta}{p^r} \right ] +
   \ldots +
   \left [ \frac{\lambda}{p^r} \right ].
\end{equation*}
From this and the expressions in (B) and (C) it follows that the
index of the highest power of any prime $p$ in the numerator of (A)
is equal to or greater than the index of the highest power of p
contained in its denominator. The theorem now follows at once.

\smallskip \textsc{Corollary.}~\emph{The product of $n$ consecutive
integers is divisible by $n!$.}

\begin{center}
EXERCISES
\end{center}

\small \begin{enumerate}
\item[1.] Show that the highest power of 2 contained in 1000! is
$2^{994}$; in 1900! is $2^{1893}$. Show that the highest power of 7
contained in 10000! is $7^{1665}$.

\item[2.] Find the highest power of 72 contained in 1000!

\item[3.] Show that 1000! ends with 249 zeros.

\item[4.] Show that there is no number $n$ such that $3^7$ is the
highest power of 3 contained in $n!$.

\item[5.] Find the smallest number $n$ such that the highest power
of 5 contained in $n!$ is $5^{31}$. What other numbers have the same
property?

\item[6.] If $n = rs$, $r$ and $s$ being positive integers, show that
$n!$ is divisible by $(r!)^s$ by $(s!)^r$; by the least common
multiple of $(r!)^s$ and $(s!)^r$.

\item[7.] If $n = \alpha + \beta + pq + rs$, where $\alpha, \beta, p,
q, r, s$, are positive integers, then $n!$ is divisible by
\begin{equation*}
\alpha ! \beta ! (q!)^p (s!)^r.
\end{equation*}

\item[8.] When $m$ and $n$ are two relatively prime positive integers
the quotient
\begin{equation*}
Q = \frac{(m + n + 1)!} {m! n!}
\end{equation*}
as an integer.

\item[9*.] If $m$ and $n$ are positive integers, then each of the
quotients
\begin{equation*}
Q = \frac{(mn)!} {n! (m!)^n},\quad
Q = \frac{(2m)! (2n)!} {m! n! (m+n)!},
\end{equation*}
is an integer. Generalize to $k$ integers $m, n, p, \ldots$.

\item[10*.] If $n = \alpha + \beta + pq + rs$ where $\alpha, \beta,
p, q, r, s$ are positive integers, then $n!$ is divisible by
\begin{equation*}
\alpha! \beta! r! p! (q!)^p (s!)^r.
\end{equation*}

\item[11*.] Show that
\begin{equation*}
\frac{(rst)!} {t! (s!)^t (r!)^{st}},
\end{equation*} is an integer ($r, s, t$ being positive integers).
Generalize to the case of $n$ integers $r, s, t, u, \ldots$.
\end{enumerate}\normalsize%
\index{Highest power of \emph{p} in \emph{n}"!|)}

\section{Remarks Concerning Prime Numbers}\label{s13}%
\index{Prime numbers|(}

We have seen that the number of primes is infinite. But the integers
which have actually been identified as prime are finite in number.
Moreover, the question as to whether a large number, as for instance
$2^{257}-1$, is prime is in general very difficult to answer. Among
the large primes actually identified as such are the following:
\begin{equation*}
2^{61}-1, \quad 2^{75} \cdot 5+1, \quad 2^{89}-1, \quad 2^{127}-1.
\end{equation*}

\emph{No analytical expression for the representation of prime
numbers has yet been discovered.} Fermat believed, though he
confessed that he was unable to prove, that he had found such an
analytical expression in
\begin{equation*}
2^{2^n} + 1.
\end{equation*}
Euler showed the error of this opinion by finding that $641$ is a
factor of this number for the case when $n = 5$.%
\index{Euler}\index{Fermat}

The subject of prime numbers is in general one of exceeding
difficulty. In fact it is an easy matter to propose problems about
prime numbers which no one has been able to solve. Some of the
simplest of these are the following:

\begin{enumerate}
\item Is there an infinite number of pairs of primes differing by
2?
\item Is every even number (other than 2) the sum of two primes or
the sum of a prime and the unit?
\item Is every even number the difference of two primes or the
difference of 1 and a prime number?
\item To find a prime number greater than a given prime.
\item To find the prime number which follows a given prime.
\item To find the number of primes not greater than a given number.
\item To compute directly the $n^\text{th}$ prime number, when $n$
is given.
\end{enumerate}\index{Prime numbers|)}

\chapter{ON THE INDICATOR OF AN INTEGER}%
\index{Indicator|(}

\section{Definition. Indicator of a Prime Power}\label{s14}%
\index{Indicator!of a prime power}

\emph{Definition.} If $m$ is any given positive integer the number
of positive integers not greater than $m$ and prime to it is called
the indicator of $m$. It is usually denoted by $\phi(m)$, and is
sometimes called Euler's $\phi$-function of $m$.%
\index{Euler's!$\phi$-function}\index{$\phi(m)$} More rarely, it has
been given the name of totient of $m$.\index{Totient}

As examples we have
\begin{equation*}
\phi(1) = 1,\ \phi(2) = 1,\ \phi(3) = 2,\ \phi(4) = 2.
\end{equation*}

If $p$ is a prime number it is obvious that
\begin{equation*}
\phi(p) = p - 1;
\end{equation*}
for each of the integers 1, 2, 3, $\ldots$, $p-1$ is prime to $p$.

Instead of taking $m = p$ let us assume that $m = p^\alpha$, where
$\alpha$ is a positive integer, and seek the value of
$\phi(p^\alpha)$. Obviously, every number of the set 1, 2, 3,
$\ldots$, $p^\alpha$ either is divisible by $p$ or is prime to
$p^\alpha$. The number of integers in the set divisible by $p$ is
$p^{\alpha - 1}$. Hence $p^\alpha-p^{\alpha-1}$ of them are prime to
$p$. Hence $\phi(p^\alpha) = p^\alpha-p^{\alpha-1}$. Therefore

\emph{If $p$ is any prime number and $\alpha$ is any positive
integer, then}
\begin{equation*}
\phi(p^\alpha) = p^\alpha \left ( 1 - \frac{1}{p} \right ).
\end{equation*}

\section{The Indicator of a Product}\label{s15}%
\index{Indicator!of a product|(}

I.~\emph{If $\mu$ and $\nu$ are any two relatively prime positive
integers, then}
\begin{equation*}
\phi(\mu\nu) = \phi(\mu) \phi(\nu).
\end{equation*}

In order to prove this theorem let us write all the integers up to
$\mu\nu$ in a rectangular array as follows:
\footnotesize\begin{equation}
 \left .
 \begin{aligned}
                 1 &&                2 &&                3 &&
  \ldots &&                h && \ldots &&  \mu \\
           \mu + 1 &&          \mu + 2 &&          \mu + 3 &&
  \ldots &&          \mu + h && \ldots && 2\mu \\
         2 \mu + 1 &&        2 \mu + 2 &&        2 \mu + 3 &&
  \ldots &&        2 \mu + h && \ldots && 3\mu \\
            \vdots &&           \vdots &&           \vdots &&
         &&           \vdots &&        && \vdots \\
  (\nu - 1)\mu + 1 && (\nu - 1)\mu + 2 && (\nu - 1)\mu + 3 &&
  \ldots && (\nu - 1)\mu + h && \ldots && \nu\mu \\
 \end{aligned}
 \right \} \tag{A}
\end{equation}\normalsize

If a number $h$ in the first line of this array has a factor in
common with $\mu$ then every number in the same column with $h$ has
a factor in common with $\mu$. On the other hand if $h$ is prime to
$\mu$, so is every number in the column with $h$ at the top. But the
number of integers in the first row prime to $\mu$ is $\phi(\mu)$.
Hence the number of columns containing integers prime to $\mu$ is
$\phi(\mu)$ and every integer in these columns is prime to $\mu$.

Let us now consider what numbers in one of these columns are prime
to $\nu$; for instance, the column with $h$ at the top. We wish to
determine how many integers of the set
\begin{gather*}
h,\ \mu + h,\ 2\mu + h,\ \ldots,\ (\nu - 1)\mu + h \\
\intertext{are prime to $\nu$. Write}
s\mu + h = q_s\nu + r_s
\end{gather*} where s ranges over the numbers $s = 0,\ 1,\ 2,\
\ldots,\ \nu - 1$ and $0\leqq r_s < \nu$. Clearly $s\mu + h$ is or
is not prime to $\nu$ according as $r_s$ is or is not prime to
$\nu$. Our problem is then reduced to that of determining how many
of the quantities $r_s$ are prime to $\nu$.

First let us notice that all the numbers $r_s$ are different; for,
if $r_s = r_t$ then from
\begin{equation*}
s\mu + h = q_s\nu + r_s,\quad t\mu + h = q_t\nu + r_t,
\end{equation*}
we have by subtraction that $(s-t)\mu$ is divisible by $\nu$. But
$\mu$ is prime to $\nu$ and $s$ and $t$ are each less than $\nu$.
Hence $(s-t)\mu$ can be a multiple of $\nu$ only by being zero; that
is, $s$ must equal $t$. Hence no two of the remainders $r_s$ can be
equal.

Now the remainders $r_s$ are $\nu$ in number, are all zero or
positive, each is less than $\nu$, and they are all distinct. Hence
they are in some order the numbers 0, 1, 2, $\ldots$, $\nu-1$. The
number of integers in this set prime to $\nu$ is evidently
$\phi(\nu)$.

Hence it follows that in any column of the array (A) in which the
numbers are prime to $\mu$ there are just $\phi(\nu)$ numbers which
are prime to $\nu$. That is, in this column there are just
$\phi(\nu)$ numbers which are prime to $\mu\nu$. But there are
$\phi(\mu)$ such columns. Hence the number of integers in the array
(A) prime to $\mu\nu$ is $\phi(\mu)\phi(\nu)$.

But from the definition of the $\phi$-function it follows that the
number of integers in the array (A) prime to $\mu\nu$ is
$\phi(\mu\nu).$ Hence,
\begin{equation*}
\phi(\mu\nu) = \phi(\mu)\phi(\nu),
\end{equation*} which is the theorem to be proved.

\smallskip \textsc{Corollary.}~\emph{In the series of $n$
consecutive terms of an arithmetical progression the common
difference of which is prime to $n$, the number of terms prime to
$n$ is $\phi(n)$.}

From theorem I we have readily the following more general result:

\smallskip II.~\emph{If $m_1, m_2, \ldots, m_k$ are $k$ positive
integers which are prime each to each, then}
\begin{equation*}
\phi(m_1 m_2 \ldots m_k) = \phi(m_1) \phi(m_2) \ldots \phi(m_k).
\end{equation*}\index{Indicator!of a product|)}

\section{The Indicator of any Positive Integer}\label{s16}%
\index{Indicator!of any integer|(}

From the results of \S\S \ref{s14} and \ref{s15} we have an
immediate proof of the following fundamental theorem:

\emph{If $m = p_1^{\alpha_1} p_2^{\alpha_2} \ldots p_n^{\alpha_n}$
where $p_1, p_2, \ldots, p_n$ are different primes and $\alpha_1,
\alpha_2, \ldots, \alpha_n$ are positive integers, then}
\begin{equation*}
\phi(m) = m \left ( 1-\frac{1}{p_1} \right )
            \left ( 1-\frac{1}{p_2} \right )
            \ldots
            \left ( 1-\frac{1}{p_n} \right ).
\end{equation*}

For,
\begin{align*}
\phi(m) &= \phi(p_1^{\alpha_1}) \phi(p_2^{\alpha_2}) \ldots
             \phi(p_n^{\alpha_n}) \\
        &= p_1^{\alpha_1} \left ( 1-\frac{1}{p_1} \right )
             p_2^{\alpha_2} \left ( 1-\frac{1}{p_2} \right )
             \ldots
             p_n^{\alpha_n} \left ( 1-\frac{1}{p_n} \right ) \\
        &= m \left ( 1-\frac{1}{p_1} \right )
             \left ( 1-\frac{1}{p_2} \right )
             \ldots
             \left ( 1-\frac{1}{p_n} \right ).
\end{align*}

On account of the great importance of this theorem we shall give a
second demonstration of it.

It is clear that the number of integers less than $m$ and divisible
by $p_1$ is
\begin{gather*}
\frac{m}{p_1}. \\
\intertext{The number of integers less than $m$ and divisible by
$p_2$ is}
\frac{m}{p_2}. \\
\intertext{The number of integers less than $m$ and divisible by
$p_1 p_2$ is}
\frac{m}{p_1 p_2}. \\
\intertext{Hence the number of integers less than $m$ and divisible
by either $p_1$ or $p_2$ is}
\frac{m}{p_1} + \frac{m}{p_2} - \frac{m}{p_1 p_2}. \\
\intertext{Hence the number of integers less than $m$ and prime to
$p_1 p_2$ is}
m - \frac{m}{p_1} - \frac{m}{p_2} + \frac{m}{p_1 p_2} =
   m \left ( 1-\frac{1}{p_1} \right ) \left ( 1-\frac{1}{p_2} \right ).
\end{gather*}

We shall now show that if the number of integers less than $m$ and
prime to $p_1 p_2 \ldots p_i$, where $i$ is less than $n$, is
\begin{gather*}
m \left ( 1-\frac{1}{p_1} \right )
  \left ( 1-\frac{1}{p_2} \right )
  \ldots
  \left ( 1-\frac{1}{p_i} \right ), \\
\intertext{then the number of integers less than $m$ and prime to
$p_1 p_2 \ldots p_i p_{i+1}$ is}
   m \left ( 1-\frac{1}{p_1} \right )
     \left ( 1-\frac{1}{p_2} \right )
     \ldots
     \left ( 1-\frac{1}{p_{i+1}} \right ).
\end{gather*}
From this our theorem will follow at once by induction.

From our hypothesis it follows that the number of integers less than
$m$ and divisible by at least one of the primes $p_1$, $p_2$,
$\ldots$, $p_i$ is
\begin{gather}
m -
   m \left (1 - \frac{1}{p_1}\right )
   \ldots
   \left (1 - \frac{1}{p_i}\right ), \notag \\
\intertext{or}
\sum \frac{m}{p_1} - \sum \frac{m}{p_1p_2}
   + \sum \frac{m}{p_1p_2p_3} - \ldots, \tag{A}
\end{gather}
where the summation in each case runs over all numbers of the type
indicated, the subscripts of the $p$'s being equal to or less than
$i$.

Let us consider the integers less than $m$ and having the factor
$p_{i+1}$ but not having any of the factors $p_1$, $p_2$, $\ldots$,
$p_i$. Their number is
\begin{gather}
\frac{m}{p_{i+1}} -
   \frac{1}{p_{i+1}} \left \{
   \sum \frac{m}{p_1} -
   \sum \frac{m}{p_1p_2} +
   \sum \frac{m}{p_1p_2p_3} -
   \ldots
   \right \}, \tag{B}
\end{gather}
where the summation signs have the same significance as before. For
the number in question is evidently $\frac{m}{p_{i+1}}$ \emph{minus}
the number of integers not greater than $\frac{m}{p_{i+1}}$ and
divisible by at least one of the primes $p_1$, $p_2$, $\ldots$,
$p_i$.

If we add (A) and (B) we have the number of integers less than $m$
and divisible by one at least of the numbers $p_1$, $p_2$, $\ldots$,
$p_{i+1}$. Hence the number of integers less than $m$ and prime to
$p_1$, $p_2$, $\ldots$, $p_{i+1}$ is
\begin{gather*}
m -
   \sum \frac{m}{p_1} +
   \sum \frac{m}{p_1 p_2} -
   \sum \frac{m}{p_1 p_2 p_3} +
   \ldots, \\
\intertext{where now in the summations the subscripts run from 1 to
$i+1$. This number is clearly equal to}
m
   \left ( 1 - \frac{1}{p_1} \right )
   \left ( 1 - \frac{1}{p_2} \right )
   \ldots
   \left ( 1 - \frac{1}{p_{i+1}} \right ).
\end{gather*}
From this result, as we have seen above, our theorem follows at once
by induction.\index{Indicator!of any integer|)}

\section{Sum of the Indicators of the Divisors of a Number}%
\label{s17}

We shall first prove the following lemma:

\smallskip \emph{Lemma. If $d$ is any divisor of $m$ and $m = nd$,
the number of integers not greater than $m$ which have with $m$ the
greatest common divisor $d$ is $\phi(n)$.}

Every integer not greater than $m$ and having the divisor $d$ is
contained in the set $d$, $2d$, $3d$, $\ldots$, $nd$. The number of
these integers which have with $m$ the greatest common divisor $d$
is evidently the same as the number of integers of the set 1, 2,
$\ldots$, $n$ which are prime to $\frac{m}{d}$, or $n$; for $\alpha
d$ and $n$ have or have not the greatest common divisor $d$
according as $\alpha$ is or is not prime to $\frac{m}{d}=n$. Hence
the number in question is $\phi(n)$.

From this lemma follows readily the proof of the following theorem:

\smallskip \emph{If $d_1$, $d_2$, $\ldots$, $d_r$ are the different
divisors of $m$, then}
\begin{equation*}
\phi(d_1) + \phi(d_2) + \ldots + \phi(d_r) = m.
\end{equation*}

Let us define integers $m_1$, $m_2$, $\ldots$, $m_r$ by the
relations
\begin{equation*}
m = d_1 m_1 = d_2 m_2 = \ldots = d_r m_r.
\end{equation*}
Now consider the set of $m$ positive integers not greater than $m$,
and classify them as follows into $r$ classes. Place in the first
class those integers of the set which have with $m$ the greatest
common divisor $m_1$; their number is $\phi(d_1)$, as may be seen
from the lemma. Place in the second class those integers of the set
which have with $m$ the greatest common divisor $m_2$; their number
is $\phi(d_2)$. Proceeding in this way throughout, we place finally
in the last class those integers of the set which have with $m$ the
greatest common divisor $m_r$; their number is $\phi(d_r)$. It is
evident that every integer in the set falls into one and into just
one of these $r$ classes. Hence the total number $m$ of integers in
the set is $\phi(d_1) + \phi(d_r) + \ldots + \phi(d_r)$. From this
the theorem follows immediately.

\begin{center}
EXERCISES
\end{center}

\small \begin{enumerate}
\item[1.] Show that the indicator of any integer greater than $2$
is even.

\item[2.] Prove that the number of irreducible fractions not greater
than $1$ and with denominator equal to $n$ is $\phi(n)$.

\item[3.] Prove that the number of irreducible fractions not greater
than $1$ and with denominators not greater than $n$ is
\begin{equation*}
\phi(1) + \phi(2) + \phi(3) + \ldots + \phi(n).
\end{equation*}

\item[4.] Show that the sum of the integers less than $n$ and prime to
$n$ is $\frac{1}{2} n \phi(n)$ if $n > 1$.

\item[5.] Find ten values of $x$ such that $\phi(x) = 24$.

\item[6.] Find seventeen values of $x$ such that $\phi(x) = 72$.

\item[7.] Find three values of $n$ for which there is no $x$ satisfying
the equation $\phi(x) = 2n$.

\item[8.] Show that if the equation
\begin{equation*}
\phi(x) = n
\end{equation*}
has one solution it always has a second solution, $n$ being given
and $x$ being the unknown.

\item[9.] Prove that all the solutions of the equation
\begin{equation*}
\phi(x) = 4n - 2, n > 1,
\end{equation*}
are of the form $p^\alpha$ and $2p^\alpha$, where $p$ is a prime of
the form $4s-1$.

\item[10.] How many integers prime to $n$ are there in the set
\begin{enumerate}
\item $1 \cdot 2, 2 \cdot 3, 3 \cdot 4, \ldots, n(n+1)$?
\item $1 \cdot 2 \cdot 3, 2 \cdot 3 \cdot 4,
   3 \cdot 4 \cdot 5, \ldots, n(n+1)(n+2)$?
\item $\frac{1 \cdot 2}{2}, \frac{2 \cdot 3}{2},
  \frac{3 \cdot 4}{2}, \ldots, \frac{n(n+1)}{2}$?
\item $\frac{1 \cdot 2 \cdot 3}{6},
  \frac{2 \cdot 3 \cdot 4}{6},
  \frac{3 \cdot 4 \cdot 5}{6},
  \ldots,
  \frac{n(n+1)(n+2)}{6}$?
\end{enumerate}

\item[11*.] Find a method for determining all the solutions of the
equation
\begin{equation*}
\phi(x) = n,
\end{equation*}
where $n$ is given and $x$ is to be sought.

\item[12*.] A number theory function $\phi(n)$ is defined for every
positive integer $n$; and for every such number $n$ it satisfies the
relation
\begin{equation*}
\phi(d_1) + \phi(d_2) + \ldots + \phi(d_r) = n,
\end{equation*}
where $d_1, d_2, \ldots, d_r$ are the divisors of $n$. From this
property alone show that
\begin{equation*}
\phi(n) = n \left ( 1 - \frac{1}{p_1} \right )
            \left ( 1 - \frac{1}{p_2} \right )
            \ldots
            \left ( 1 - \frac{1}{p_k} \right ),
\end{equation*}
where $p_1, p_2, \ldots, p_k$ are the different prime factors of
$n$. \end{enumerate} \normalsize\index{Indicator|)}

\chapter{ELEMENTARY PROPERTIES OF CONGRUENCES}%
\index{Congruences|(}

\section{Congruences Modulo $m$}\label{s18}

\textsc{Definitions.} If $a$ and $b$ are any two integers, positive
or zero or negative, whose difference is divisible by $m$, $a$ and
$b$ are said to be congruent modulo $m$, or congruent for the
modulus $m$, or congruent according to the modulus $m$. Each of
the numbers $a$ and $b$ is said to be a residue of the other.%
\index{Residue}

\smallskip To express the relation thus defined we may write
\begin{equation*}
a = b + cm,
\end{equation*}
where $c$ is an integer (positive or zero or negative). It is more
convenient, however, to use a special notation due to Gauss, and to
write
\begin{equation*}
a \equiv b \mod m,
\end{equation*}
an expression which is read $a$ is congruent to $b$ modulo $m$, or
$a$ is congruent to $b$ for the modulus $m$, or $a$ is congruent to
$b$ according to the modulus $m$.\index{Gauss} This notation has the
advantage that it involves only the quantities which are essential
to the idea involved, whereas in the preceding expression we had the
irrelevant integer $c$. The Gaussian notation is of great value and
convenience in the study of the theory of divisibility. In the
present chapter we develop some of the fundamental elementary
properties of congruences. It will be seen that many theorems
concerning equations are likewise true of congruences with fixed
modulus; and it is this analogy with equations which gives
congruences (as such) one of their chief claims to attention.

As immediate consequences of our definitions we have the following
fundamental theorems:

\smallskip I.~\emph{If} $a\equiv c \mod m$, $b\equiv c\mod m$,
\emph{then} $a\equiv b\mod m$; \noindent \emph{that is, for a given
modulus, numbers congruent to the same number are congruent to each
other.}

For, by hypothesis, $a - c = c_1 m$, $b - c = c_2 m$, where $c_1$
and $c_2$ are integers. Then by subtraction we have $a - b = (c_1 -
c_2) m$; whence $a \equiv b \mod m$.

\smallskip II.~\emph{If} $a \equiv b \mod m$, $\alpha \equiv
\beta \mod m$, \emph{then} $a \pm \alpha \equiv b \pm \beta \mod m$;
\emph{that is, congruences with the same modulus may be added or
subtracted member by member.}

For, by hypothesis, $a - b = c_1 m$, $\alpha - \beta = c_2 m$;
whence $(a \pm \alpha) - (b \pm \beta) = (c_1 \pm c_2)m$. Hence $a
\pm \alpha = b \pm \beta \mod m$.

\smallskip III.~\emph{If} $a = b \mod m$, \emph{then}
$ca = cb \mod m$, \emph{$c$ being any integer whatever.}

The proof is obvious and need not be stated.

\smallskip IV.~\emph{If} $a \equiv b \mod m$,
$\alpha \equiv \beta \mod m$, \emph{then} $a \alpha \equiv b \beta
\mod m$; \emph{that is, two congruences with the same modulus may be
multiplied member by member.}

For, we have $a = b + c_1 m$, $\alpha = \beta + c_2 m$. Multiplying
these equations member by member we have $a \alpha = b \beta + m (b
c_2 + \beta c_1 + c_1 c_2 m)$. Hence $a \alpha \equiv b \beta \mod
m$.

\smallskip A repeated use of this theorem gives the following
result:

\smallskip V.~\emph{If} $a \equiv b \mod m$, \emph{then}
$a^n \equiv b^n \mod m$ \emph{where $n$ is any positive integer.}

\smallskip As a corollary of theorems II, III and V we have the
following more general result:

\smallskip VI.~\emph{If $f(x)$ denotes any polynomial in $x$ with
coefficients which are integers (positive or zero or negative) and
if further $a\equiv b \bmod m$, then}
\begin{equation*}
f(a) \equiv f(b) \bmod m.
\end{equation*}

\section{Solutions of Congruences by Trial}\label{s19}%
\index{Congruences!Solution by trial|(}

Let $f(x)$ be any polynomial in $x$ with coefficients which are
integers (positive or negative or zero). Then if $x$ and $c$ are any
two integers it follows from the last theorem of the preceding
section that
\begin{gather*}
f(x) \equiv f(x + cm) \bmod m. \tag{1} \\
\intertext{Hence if $a$ is any value of $x$ for which the
congruence}
f(x)\equiv 0\bmod m. \tag{2}
\end{gather*}
is satisfied, then the congruence is also satisfied for $x = \alpha
+ cm$, where $c$ is any integer whatever. The numbers $\alpha + cm$
are said to form a \emph{solution} (or to be a \emph{root}) of the
congruence, $c$ being a variable integer. Any one of the integers
$\alpha + cm$ may be taken as the representative of the solution. We
shall often speak of one of these numbers as the solution itself.

Among the integers in a solution of the congruence (2) there is
evidently one which is positive and not greater than $m$. Hence all
solutions of a congruence of the type (2) may be found by trial, a
substitution of each of the numbers $1, 2, \ldots, m$ being made for
$x$. It is clear also that $m$ is the maximum number of solutions
which (2) can have whatever be the function $f(x)$. By means of an
example it is easy to show that this maximum number of solutions is
not always possessed by a congruence; in fact, it is not even
necessary that the congruence have a solution at all.

This is illustrated by the example
\begin{equation*}
x^2 - 3 \equiv 0 \bmod 5.
\end{equation*}
In order to show that no solution is possible it is necessary to
make trial only of the values $1, 2, 3, 4, 5$ for $x$. A direct
substitution verifies the conclusion that none of them satisfies the
congruence; and hence that the congruence has no solution at all.

On the other hand the congruence
\begin{equation*}
x^5 - x \equiv 0 \bmod 5
\end{equation*}
has the solutions $x = 1, 2, 3, 4, 5$ as one readily verifies; that
is, this congruence has five solutions---the maximum number possible
in accordance with the results obtained above.

\begin{center}
EXERCISES
\end{center}

\small \begin{enumerate}
\item[1.] Show that $(a + b)^p \equiv  a^p + b^p \bmod p$
where $a$ and $b$ are any integers and $p$ is any prime.

\item[2.] From the preceding result prove that
$\alpha^p \equiv \alpha \bmod p$ for every integer $\alpha$.

\item[3.] Find all the solutions of each of the congruences $x^{11}
\equiv x \bmod 11, x^{10} \equiv 1 \bmod 11, x^{5} \equiv 1 \bmod
11$.
\end{enumerate} \normalsize\index{Congruences!Solution by trial|)}

\section{Properties of Congruences Relative to Division}\label{s20}

The properties of congruences relative to addition, subtraction and
multiplication are entirely analogous to the properties of algebraic
equations. But the properties relative to division are essentially
different. These we shall now give.

\smallskip I.~\emph{If two numbers are congruent modulo $m$ they are
congruent modulo $d$, where $d$ is any divisor of $m$.}

For, from $a \equiv b \bmod m$, we have $a = b + cm = b + c'd$.
Hence $a\equiv b \bmod d$.

\smallskip II.~\emph{If two numbers are congruent for different
moduli they are congruent for a modulus which is the least common
multiple of the given moduli.}

The proof is obvious, since the difference of the given numbers is
divisible by each of the moduli.

\smallskip III.~\emph{When the two members of a congruence are
multiples of an integer $c$ prime to the modulus, each member of the
congruence may be divided by $c$.}

For, if $ca \equiv cb \bmod m$ then $ca - cb$ is divisible by $m$.
Since $c$ is prime to $m$ it follows that $a - b$ is divisible by
$m$. Hence $a\equiv b \bmod m$.

\smallskip IV.~\emph{If the two members of a congruence are
divisible by an integer $c$, having with the modulus the greatest
common divisor $\delta$, one obtains a congruence equivalent to the
given congruence by dividing the two members by $c$ and the modulus
by $\delta$.}

By hypothesis $ac \equiv bc \bmod m,\quad c = \delta c_1,\quad m =
\delta m_1$. Hence $c(a-b)$ is divisible by $m$. A necessary and
sufficient condition for this is evidently that $c_1(a-b)$ is
divisible by $m_1$. This leads at once to the desired result.

\section{Congruences with a Prime Modulus}\label{s21}%
\index{Congruences!with prime modulus|(}

\emph{The congruence\footnote{The sign $\not\equiv$ is read \emph{is
not congruent to}.}}
\begin{equation*}
a_0 x^n + a_1 x^{n-1} + \ldots + a_n \equiv 0 \bmod p,
  \quad a_0 \not\equiv 0 \bmod p
\end{equation*}
\emph{where $p$ is a prime number and the $a$'s are any integers,
has not more than $n$ solutions.}

Denote the first member of this congruence by $f(x)$ so that the
congruence may be written
\begin{gather}
f(x) \equiv 0 \bmod p \tag{1} \\
\intertext{Suppose that $a$ is a root of the congruence, so that}
f(a) \equiv 0 \bmod p. \notag \\
\intertext{Then we have} f(x)
\equiv f(x) - f(a) \bmod p. \notag \\
\intertext{But, from algebra, $f(x) - f(a)$ is divisible by $x - a$.
Let $(x-a)^{\alpha}$ be the highest power of $x - a$ contained in
$f(x) - f(a)$. Then we may write}
f(x) - f(a) = (x - a)^{\alpha} f_1(x), \tag{2} \\
\intertext{where $f_1(x)$ is evidently a polynomial with integral
coefficients. Hence we have}
f(x) \equiv (x - a)^{\alpha} f_1(x) \bmod p. \tag{3}
\end{gather}
We shall say that $a$ occurs $\alpha$ times as a solution of (1); or
that the congruence has $\alpha$ solutions each equal to $a$.

Now suppose that congruence (1) has a root $b$ such that
$b\not\equiv a \bmod p$. Then from (3) we have
\begin{gather*}
f(b) \equiv (b-a)^{\alpha}f_1(b) \bmod p. \\
\intertext{But}
f(b)\equiv 0 \bmod p,\quad (b-a)^{\alpha} \not\equiv 0 \bmod p. \\
\intertext{Hence, since $p$ is a prime number, we must have}
f_1(b)\equiv 0 \bmod p.
\end{gather*}

By an argument similar to that just used above, we may show that
$f_1(x) - f_1(b)$ may be written in the form
\begin{gather*}
f_1(x) - f_1(b) = (x-b)^{\beta}f_2(x), \\
\intertext{where $\beta$ is some positive integer. Then we have}
f(x) \equiv (x-a)^{\alpha}(x-b)^{\beta}f_2(x) \bmod p.
\end{gather*}

Now this process can be continued until either all the solutions of
(1) are exhausted or the second member is a product of linear
factors multiplied by the integer $a_0$. In the former case there
will be fewer than $n$ solutions of (1), so that our theorem is true
for this case. In the other case we have
\begin{equation*}
f(x) \equiv a_0(x-a)^{\alpha}(x-b)^{\beta}
     \ldots (x-l)^{\lambda} \bmod p.
\end{equation*}
We have now $n$ solutions of (1): $a$ counted $\alpha$ times, $b$
counted $\beta$ times, \ldots, $l$ counted $\lambda$ times; $\alpha
+ \beta + \ldots +\lambda = n$.

Now let $\eta$ be any solution of (1). Then
\begin{equation*}
f(\eta) \equiv a_0(\eta-a)^{\alpha}(\eta-b)^{\beta} \ldots
   (\eta-l)^{\lambda} \equiv 0 \bmod p.
\end{equation*}
Since $p$ is prime it follows now that some one of the factors
$\eta-a, \eta-b, \ldots, \eta-l$ is divisible by $p$. Hence $\eta$
coincides with one of the solutions $a, b, c, \ldots, l$. That is,
(1) can have only the $n$ solutions already found.

This completes the proof of the theorem.

\begin{center}
EXERCISES
\end{center}

\small \begin{enumerate}
\item[1.] Construct a congruence of the form
\begin{equation*}
a_0 x^n + a_1 x^{n-1} + \ldots + a_n \equiv 0 \bmod m, \quad
  a_0 \not\equiv 0 \bmod m,
\end{equation*}
having more than $n$ solutions and thus show that the limitation to
a prime modulus in the theorem of this section is essential.

\item[2.] Prove that
\begin{equation*}
x^6-1 \equiv (x-1)(x-2)(x-3)(x-4)(x-5)(x-6) \bmod 7
\end{equation*}
for every integer $x$.

\item[3.] How many solutions has the congruence $x^5 \equiv 1 \bmod
11$? the congruence $x^5\equiv 2 \bmod 11$?
\end{enumerate}\normalsize\index{Congruences!with prime modulus|)}

\section{Linear Congruences}\label{s22}%
\index{Congruences!Linear|(}

From the theorem of the preceding section it follows that the
congruence
\begin{equation*}
ax \equiv c \bmod p,\quad a \not\equiv 0 \bmod p,
\end{equation*}
where $p$ is a prime number, has not more than one solution. In this
section we shall prove that it always has a solution. More
generally, we shall consider the congruence
\begin{equation*}
ax \equiv c \bmod m
\end{equation*}
where $m$ is any integer. The discussion will be broken up into
parts for convenience in the proofs.

\smallskip I.~\emph{The congruence}
\begin{equation}
ax \equiv 1 \bmod m, \tag{1}
\end{equation}
\emph{in which a and m are relatively prime, has one and only one
solution.}

The question as to the existence and number of the solutions of (1)
is equivalent to the question as to the existence and number of
integer pairs $x, y$ satisfying the equation,
\begin{equation}
ax - my = 1, \tag{2}
\end{equation}
the integers $x$ being incongruent modulo $m$. Since $a$ and $m$ are
relatively prime it follows from theorem IV of \S~\ref{s9} that
there exists a solution of equation (2). Let $x = \alpha$ and $y =
\beta$ be a particular solution of (2) and let $x = \bar{\alpha}$
and $y = \bar{\beta}$ be any solution of (2). Then we have
\begin{gather*}
a\alpha-m\beta = 1, \\
a \bar{\alpha} - m\bar{\beta} = 1; \\
\intertext{whence}
a(\alpha - \bar{\alpha}) - m(\beta - \bar{\beta}) = 0.
\end{gather*}
Hence $\alpha-\bar{\alpha}$ is divisible by $m$, since $a$ and $m$
are relatively prime. That is, $a \equiv \bar{\alpha} \mod m$. Hence
$\alpha$ and $\bar{\alpha}$ are representatives of the same solution
of (1). Hence (1) has one and only one solution, as was to be
proved.

\smallskip II.~\emph{The solution $x = \alpha$ of the congruence
$ax \equiv 1 \mod m$, in which $a$ and $m$ are relatively prime, is
prime to $m$.}

For, if $a\alpha - 1$ is divisible by $m$, $\alpha$ is divisible by
no factor of $m$ except $1$.

\smallskip III.~\emph{The congruence}
\begin{equation}
ax \equiv c \mod m \tag{3}
\end{equation}
\emph{in which $a$ and $m$ and also $c$ and $m$ are relatively
prime, has one and only one solution.}

Let $x = \gamma$ be the unique solution of the congruence $cx = 1
\mod m$. Then we have $a\gamma x \equiv c\gamma \equiv 1 \mod m$.
Now, by I we see that there is one and only one solution of the
congruence $a\gamma x \equiv 1 \mod m$; and from this the theorem
follows at once.

Suppose now that $a$ is prime to $m$ but that $c$ and $m$ have the
greatest common divisor $\delta$ which is different from 1. Then it
is easy to see that any solution $x$ of the congruence $ax \equiv c
\mod m$ must be divisible by $\delta$. The question of the existence
of solutions of the congruence $ax \equiv c \bmod m$ is then
equivalent to the question of the existence of solutions of the
congruence
\begin{equation*}
a \frac{x}{\delta} \equiv \frac{c}{\delta} \bmod \frac{m}{\delta},
\end{equation*}
where $\frac{x}{\delta}$ is the unknown integer. From III it follows
that this congruence has a unique solution $\frac{x}{\delta} =
\alpha$. Hence the congruence $ax \equiv c \bmod m$ has the unique
solution $x = \delta\alpha$. Thus we have the following theorem:

\smallskip IV.~\emph{The congruence $ax \equiv c \bmod m$, in which
$a$ and $m$ are relatively prime, has one and only one solution.}


\smallskip\textsc{Corollary.}~\emph{The congruence
$ax \equiv c \bmod p$, $a \not\equiv 0 \bmod p$, where $p$ is a
prime number, has one and only one solution.}

It remains to examine the case of the congruence $ax =c \bmod m$ in
which $a$ and $m$ have the greatest common divisor $d$. It is
evident that there is no solution unless $c$ also contains this
divisor $d$. Then let us suppose that $a = \alpha d$, $c = \gamma
d$, $m = \mu d$. Then for every $x$ such that $ax = c \bmod m$ we
have $\alpha x = \gamma \bmod \mu$; and conversely every $x$
satisfying the latter congruence also satisfies the former. Now
$\alpha x = \gamma \bmod \mu$, has only one solution. Let $\beta$ be
a non-negative number less than $\mu$, which satisfies the
congruence $\alpha x = \gamma \bmod \mu$. All integers which satisfy
this congruence are then of the form $\beta + \mu\nu$, where $\nu$
is an integer. Hence all integers satisfying the congruence $ax = c
\bmod m$ are of the form $\beta + \mu\nu$; and every such integer is
a representative of a solution of this congruence. It is clear that
the numbers
\begin{equation}
\beta,\ \beta + \mu,\ \beta + 2\mu,\ \ldots,\ \beta + (d-1)\mu
\tag{A}
\end{equation}
are incongruent modulo $m$ while every integer of the form $\beta +
\mu\nu$ is congruent modulo $m$ to a number of the set (A). Hence
the congruence $ax = c \bmod m$ has the $d$ solutions (A).

This leads us to an important theorem which includes all the other
theorems of this section as special cases. It may be stated as
follows:

\smallskip V.~\emph{Let}
\begin{equation*}
ax \equiv c \bmod m
\end{equation*}
\emph{be any linear congruence and let $a$ and $m$ have the greatest
common divisor $d (d \geq 1)$. Then a necessary and sufficient
condition for the existence of solutions of the congruence is that
$c$ be divisible by $d$. If this condition is satisfied the
congruence has just $d$ solutions, and all the solutions are
congruent modulo $m / d$.}

\newpage
\begin{center}
EXERCISES
\end{center}

\small \begin{enumerate}
\item[1.] Find the remainder when $2^{40}$ is divided by $31$; when
$2^{43}$ is divided by $31$.

\item[2.] Show that $2^{2^5}+1$ has the factor $641$.

\item[3.] Prove that a number is a multiple of $9$ if and only if the
sum of its digits is a multiple of $9$.

\item[4.] Prove that a number is a multiple of $11$ if and only if the
sum of the digits in the odd numbered places diminished by the sum
of the digits in the even numbered places is a multiple of $11$.
\end{enumerate} \normalsize%
\index{Congruences|)}\index{Congruences!Linear|)}

\chapter{THE THEOREMS OF FERMAT AND WILSON}

\section{Fermat's General Theorem}\label{s23}%
\index{Fermat's!General Theorem}

Let $m$ be any positive integer and let
\begin{equation}
a_1,\ a_2,\ \ldots,\ a_{\phi(m)} \tag{A}
\end{equation}
be the set of $\phi(m)$ positive integers not greater than $m$ and
prime to $m$. Let $a$ be any integer prime to $m$ and form the set
of integers
\begin{equation}
aa_1,\ aa_2,\ \ldots,\ aa_{\phi(m)} \tag{B}
\end{equation}
No number $aa_i$ of the set (B) is congruent to a number $aa_j$,
unless $j = i;$ for, from
\begin{equation*}
aa_i \equiv aa_j \bmod m
\end{equation*}
we have $a_i \equiv a_j \bmod m$; whence $a_i = a_j$ since both
$a_i$ and $a_j$ are positive and not greater than $m$. Therefore $j
= i$. Furthermore, every number of the set (B) is congruent to some
number of the set (A). Hence we have congruences of the form
\begin{align*}
aa_1         & \equiv a_{i_1} \bmod m, \\
aa_2         & \equiv a_{i_2} \bmod m, \\
             & \vdots \\
aa_{\phi(m)} & \equiv a_{i_{\phi(m)}} \bmod m.
\end{align*}
No two numbers in the second members are equal, since $aa_i
\not\equiv aa_j$ unless $i= j$. Hence the numbers $a_{i_1},\
a_{i_2},\ \ldots,\ a_{i_{\phi(m)}}$ are the numbers $a_1,\ a_2,\
\ldots,\ a_{\phi(m)}$ in some order. Therefore, if we multiply the
above system of congruences together member by member and divide
each member of the resulting congruence by $a_1\cdot a_2\ldots
a_{\phi(m)}$ (which is prime to $m$), we have
\begin{equation*}
a^{\phi(m)} \equiv 1 \bmod m.
\end{equation*}
This result is known as Fermat's general theorem.%
\index{Fermat's!general theorem} It may be stated as follows:

\emph{If $m$ is any positive integer and $a$ is any integer prime to
$m$, then}
\begin{equation*}
a^{\phi(m)} \equiv 1 \bmod m.
\end{equation*}

\smallskip \textsc{Corollary 1.}~\emph{If $a$ is any integer
not divisible by a prime number $p$, then}
\begin{equation*}
a^{p-1} \equiv 1 \bmod p.
\end{equation*}

\smallskip \textsc{Corollary 2.}~\emph{If $p$ is any prime number
and $a$ is any integer, then}
\begin{equation*}
a^p \equiv a \bmod p.
\end{equation*}

\section{Euler's Proof of the Simple Fermat Theorem}\label{s24}%
\index{Euler}\index{Fermat}\index{Fermat's!Simple Theorem}

The theorem of Cor.\ 1, \S~\ref{s23}, is often spoken of as the
simple Fermat theorem. It was first announced by Fermat in 1679, but
without proof. The first proof of it was given by Euler in 1736.
This proof may be stated as follows:

From the Binomial Theorem it follows readily that
\begin{gather*}
(a+1)^p \equiv a^p + 1 \bmod p \\
\intertext{since}
\frac{p!}{r!(p-r)!}, \quad 0 < r < p, \\
\intertext{is obviously divisible by $p$. Subtracting $a + 1$ from
each side of the foregoing congruence, we have}
(a+1)^p - (a+1) \equiv a^p - a \bmod p.
\end{gather*}
Hence if $a^p - a$ is divisible by $p$, so is $(a + 1)^p - (a + 1)$.
But $1^p - 1$ is divisible by $p$. Hence $2^p - 2$ is divisible by
$p$; and then $3^p - 3$; and so on. Therefore, in general, we have
\begin{equation*}
a^p \equiv a \mod p.
\end{equation*}
If $a$ is prime to $p$ this gives $a^{p-1} \equiv 1 \mod p$, as was
to be proved.

If instead of the Binomial Theorem one employs the Polynomial
Theorem, an even simpler proof is obtained. For, from the latter
theorem, we have readily
\begin{gather*}
(\alpha_1 + \alpha_2 + \ldots + \alpha_a)^p \equiv
  \alpha_1^p + \alpha_2^p + \ldots + \alpha_a^p \mod p. \\
\intertext{Putting $\alpha_1 = \alpha_2 = \ldots = \alpha_a = 1$ we
have}
a^p\equiv a \mod p,
\end{gather*}
from which the theorem follows as before.

\section{Wilson's Theorem}\label{s25}\index{Wilson's theorem|(}

From the simple Fermat theorem it follows that the congruence
\begin{gather*}
x^{p-1} \equiv 1\mod p \\
\intertext{has the $p-1$ solutions $1$, $2$, $3$, $\ldots$, $p-1$.
Hence from the discussion in \S \ref{s21} it follows that}
x^{p-1} \equiv (x-1)(x-2)\ldots(x-\overline{p-1}) \mod p, \\
\intertext{this relation being satisfied for every value of $x$.
Putting $x = 0$ we have}
(-1) = (-1)^{p - 1}\cdot 1\cdot 2\cdot 3 \ldots
           \overline{p-1}\mod p.  \\
\intertext{If $p$ is an odd prime this leads to the congruence}
1 \cdot 2 \cdot 3 \ldots \overline{p-1} + 1 = 0 \mod p.
\end{gather*}
Now for $p = 2$ this congruence is evidently satisfied. Hence
we have the Wilson theorem:

\smallskip \emph{Every prime number $p$ satisfies the relation}
\begin{equation*}
1 \cdot 2 \cdot 3 \ldots \overline{p+1} + 1 \equiv 0 \mod p.
\end{equation*}

An interesting proof of this theorem on wholly different principles
may be given. Let $p$ points be distributed at equal intervals on
the circumference of a circle. The whole number of $p$-gons which
can be formed by joining up these $p$ points in every possible order
is evidently
\begin{equation*}
\frac{1}{2p} p (p-1) (p-2) \ldots 3 \cdot 2 \cdot 1;
\end{equation*}
for the first vertex can be chosen in $p$ ways, the second in $p -
1$ ways, $\ldots$, the $(p-1)^{\mathrm{th}}$ in two ways, and the
last in one way; and in counting up thus we have evidently counted
each polygon $2p$ times, once for each vertex and for each direction
from the vertex around the polygon. Of the total number of polygons
$\frac{1}{2}(p-1)$ are regular (convex or stellated) so that a
revolution through $\frac{360^\circ}{p}$ brings each of these into
coincidence with its former position. The number of remaining
$p$-gons must be divisible by $p$; for with each such $p$-gon we may
associate the $p-1$ $p$-gons which can be obtained from it by
rotating it through successive angles of $\frac{360^\circ}{p}$. That
is,
\begin{gather*}
\frac{1}{2p} p (p-1) (p-2) \ldots 3 \cdot 2 \cdot 1 -
  \frac 12 (p-1) \equiv 0 \bmod p. \\
\intertext{Hence}
(p-1) (p-2) \ldots 3 \cdot 2 \cdot 1 -  p + 1 \equiv 0 \bmod p; \\
\intertext{and from this it follows that}
1 \cdot 2 \ldots \overline{p-1} + 1 \equiv 0 \bmod p, \\
\end{gather*}
as was to be proved.

\section{The Converse of Wilson's Theorem}\label{s26}

Wilson's theorem is noteworthy in that its converse is also true.
The converse may be stated as follows:

\smallskip \emph{Every integer $n$ such that the congruence}
\begin{equation*}
1 \cdot 2 \cdot 3 \ldots \overline{n-1} + 1 \equiv 0 \bmod n
\end{equation*}
\emph{is satisfied is a prime number.}

For, if $n$ is not prime, there is some divisor $d$ of $n$ different
from $1$ and less than $n$. For such a $d$ we have $1 \cdot 2 \cdot
3 \ldots \overline{n-1} \equiv 0 \bmod d$; so that $1 \cdot 2 \ldots
\overline{n-1}+1 \not\equiv 0 \bmod d$; and hence $1 \cdot 2 \ldots
\overline{n-1}+1 \equiv 0 \bmod n$. Since this contradicts our
hypothesis the truth of the theorem follows.

\smallskip Wilson's theorem and its converse may be combined into
the following elegant theorem:

\smallskip \emph{A necessary and sufficient condition that an
integer $n$ is prime is that}
\begin{equation*}
1 \cdot 2 \cdot 3 \ldots \overline{n-1} + 1 \equiv 0 \bmod n.
\end{equation*}\index{Prime numbers}

Theoretically this furnishes a complete and elegant test as to
whether a given number is prime. But, practically, the labor of
applying it is so great that it is useless for verifying large
primes.

\section{Impossibility of $1 \cdot 2 \cdot 3 \cdots
\overline{n-1} + 1 = n^k$ for $n > 5$.}\label{s27}

In this section we shall prove the following theorem:

\emph{There exists no integer $k$ for which the equation}
\begin{equation*}
1 \cdot 2 \cdot 3 \cdots \overline{n-1} + 1 = n^k
\end{equation*}
is true when $n$ is greater than $5$.

If $n$ contains a divisor $d$ different from $1$ and $n$, the
equation is obviously false; for the second member is divisible by
$d$ while the first is not. Hence we need to prove the theorem only
for primes $n$.

Transposing $1$ to the second member and dividing by $n - 1$ we have
\begin{equation*}
1 \cdot 2 \cdot 3 \cdots \overline{n-2} = n^{k-1} + n^{k-2}
   + \ldots + n+1.
\end{equation*}
If $n>5$ the product on the left contains both the factor $2$ and
the factor $\frac{1}{2} (n-1)$; that is, the first member contains
the factor $n - 1$. But the second member does not contain this
factor, since for $n = 1$ the expression $n^{k-1} + \ldots n + 1$ is
equal to $k \neq 0$. Hence the theorem follows at once.

\section{Extension of Fermat's Theorem}\label{s28}%
\index{Fermat's!theorem extended|(}

The object of this section is to extend Fermat's general theorem and
incidentally to give a new proof of it. We shall base this proof on
the simple Fermat theorem, of which we have already given a simple
independent proof. This theorem asserts that for every prime $p$ and
integer $a$ not divisible by $p$, we have the congruence
\begin{equation*}
a^{p-1} \equiv 1 \bmod p.
\end{equation*}

Then let us write
\begin{gather}
a^{p-1} = 1 + hp. \tag{1} \\
\intertext{Raising each member of this equation to the
$p^{\text{th}}$ power we may write the result in the form}
a^{p(p-1)} = 1 + h_1p^2. \tag{2} \\
\intertext{where $h_1$ is an integer. Hence}
a^{p(p-1)} \equiv 1 \bmod p^2. \notag \\
\intertext{By raising each member of (2) to the $p^{\text{th}}$
power we can readily show that}
a^{p^2(p-1)} \equiv 1 \bmod p^3. \notag \\
\intertext{It is now easy to see that we shall have in general}
a^{p^{\alpha - 1}(p-1)} \equiv 1 \bmod p^{\alpha}. \notag \\
\intertext{where $\alpha$ is a positive integer; that is,}
a^{\phi(p^{\alpha})} \equiv 1 \bmod p^{\alpha}. \notag
\end{gather}

For the special case when $p$ is 2 this result can be extended. For
this case (1) becomes
\begin{gather}
a = 1 + 2h. \notag \\
\intertext{Squaring we have}
a^2 = 1 + 4h(h+1). \notag \\
\intertext{Hence,}
a^2 = 1+8h_1, \tag{3} \\
\intertext{where $h_1$ is an integer. Therefore}
a^2 \equiv 1 \bmod 2^3. \notag \\
\intertext{Squaring (3) we have}
a^{2^2} = 1 + 2^4h_2; \notag \\
\intertext{or}
a^{2^2} \equiv 1 \bmod 2^4. \notag \\
\intertext{It is now easy to see that we shall have in general}
a^{2^{\alpha-2}} \equiv 1 \bmod 2^{\alpha} \notag \\
\intertext{if $\alpha > 2$. That is,}
a^{\frac{1}{2}\phi(2^{\alpha})} \equiv 1 \bmod 2^{\alpha}
  \text{ if } a > 2.
\end{gather}

Now in terms of the $\phi$-function let us define a new function
$\lambda(m)$ as follows:
\begin{align*}
\lambda(2^{\alpha}) &= \phi(2^{\alpha}) \text{ if $a = 0, 1, 2$;} \\
\lambda(2^{\alpha}) &= \frac{1}{2}\phi(2^{\alpha})
                                               \text{ if $a > 2$;} \\
\lambda(p^{\alpha}) &= \phi(p^{\alpha})
                                   \text{ if $p$ is an odd prime;} \\
\lambda(2^{\alpha} p_1^{\alpha_1} p_2^{\alpha_2} \cdots
    p_n^{\alpha_n}) &= M,
\end{align*}
where $M$ is the least common multiple of
\begin{equation*}
  \lambda(2^{\alpha}),
  \lambda(p_1^{\alpha_1}),
  \lambda(p_2^{\alpha_2}), \ldots, \lambda(p_n^{\alpha_n}),
\end{equation*}
$2, p_1, p_2, \ldots, p_n$ being different primes.%
\index{$\lambda(m)$}

Denote by $m$ the number
\begin{equation*}
m = 2^{\alpha}p_1^{\alpha_1}p_2^{\alpha_2} \cdots p_n^{\alpha_n}.
\end{equation*}
Let $a$ be any number prime to $m$. From our preceding results we
have
\begin{align*}
a^{\lambda(2^{\alpha})}     &\equiv 1 \bmod 2^{\alpha}, \\
a^{\lambda(p_1^{\alpha_1})} &\equiv 1 \bmod p_1^{\alpha_1},\\
a^{\lambda(p_2^{\alpha_2})} &\equiv 1 \bmod p_2^{\alpha_2}, \\
\ldots \\
a^{\lambda(p_n^{\alpha_n})} &\equiv 1 \bmod p_2^{\alpha_n}. \\
\end{align*}

Now any one of these congruences remains true if both of its members
are raised to the same positive integral power, whatever that power
may be. Then let us raise both members of the first congruence to
the power $\frac{\lambda(m)}{\lambda(2^\alpha)}$; both members of
the second congruence to the power
$\frac{\lambda(m)}{\lambda(p_1^{\alpha_1})}$; $\ldots$; both members
of the last congruence to the power
$\frac{\lambda(m)}{\lambda(p_n^{\alpha_n})}$. Then we have
\begin{align*}
a^{\lambda(m)} &\equiv 1 \mod 2^\alpha, \\
a^{\lambda(m)} &\equiv 1 \mod p_1^{\alpha_1}, \\
\ldots \ldots \\
a^{\lambda(m)} &\equiv 1 \mod p_n^{\alpha_n}. \\
\intertext{From these congruences we have immediately}
a^{\lambda(m)} &\equiv 1 \mod m.
\end{align*}

We may state this result in full in the following theorem:

\smallskip \emph{If $a$ and $m$ are any two relatively prime positive
integers, the congruence}
\begin{equation*}
a^{\lambda(m)} \equiv 1 \mod m.
\end{equation*}
\emph{is satisfied.}

As an excellent example to show the possible difference between the
exponent $\lambda(m)$ in this theorem and the exponent $\phi(m)$ in
Fermat's general theorem, let us take
\begin{gather*}
m = 2^6 \cdot 3^3 \cdot 5 \cdot 7 \cdot 13 \cdot 17 \cdot 19
        \cdot 37 \cdot 73. \\
\intertext{Here}
\lambda(m) = 2^4 \cdot 3^2, \quad \phi(m) = 2^{31} \cdot 3^{10}.
\end{gather*}

In a later chapter we shall show that there is no exponent $\nu$
less than $\lambda(m)$ for which the congruence
\begin{equation*}
a^\nu = 1 \mod m
\end{equation*}
is verified for every integer $a$ prime to $m$.

From our theorem, as stated above, Fermat's general theorem follows
as a corollary, since $\lambda(m)$ is obviously a factor of
$\phi(m)$,
\begin{equation*}
\phi(m) = \phi(2^\alpha) \phi(p_1^{\alpha_1}) \ldots
               \phi(p_n^{\alpha_n}).
\end{equation*}

\begin{center}
EXERCISES
\end{center}

\small \begin{enumerate}
\item[1.] Show that $a^{16} \equiv 1 \bmod 16320$, for every $a$
which is prime to $16320$.

\item[2.] Show that $a^{12} \equiv 1 \bmod 65520$, for every $a$ which
is prime to $65520$.

\item[3*.] Find one or more composite numbers $P$ such that
\begin{equation*}
a^{P-1} \equiv 1 \bmod P
\end{equation*}
for every a prime to $P$. (Compare this problem with the next
section.) \end{enumerate} \normalsize%
\index{Fermat's!theorem extended|)}

\section{On the Converse of Fermat's Simple Theorem}\label{s29}%
\index{Fermat's!Simple Theorem}

The fact that the converse of Wilson's theorem is a true proposition
leads one naturally to inquire whether the converse of Fermat's
simple theorem is true. Thus, we may ask the question: Does the
existence of the congruence $2^{n-1} \equiv 1 \bmod n$ require that
$n$ be a prime number? The Chinese answered this question in the
affirmative and the answer passed unchallenged among them for many
years. An example is sufficient to show that the theorem is not
true. We shall show that
\begin{equation*}
2^{340} \equiv 1 \bmod 341
\end{equation*}
although $341 = 11 \cdot 31$, is not a prime number. Now $2^{10}-1 =
3 \cdot 11 \cdot 31$. Hence $2^{10} \equiv 1 \bmod 341$. Hence
$2^{340} \equiv 1 \bmod 341$. From this it follows that the direct
converse of Fermat's theorem is not true. The following theorem,
however, which is a converse with an extended hypothesis, is readily
proved.

\smallskip \emph{If there exists an integer $a$ such that}
\begin{equation*}
a^{n-1} \equiv 1 \bmod n
\end{equation*}
\emph{and if further there does not exist an integer $\nu$ less than
$n - 1$ such that}
\begin{equation*}
a^{\nu} \equiv 1 \bmod n,
\end{equation*}
\emph{then the integer $n$ is a prime number.}

For, if $n$ is not prime, $\phi(n) < n - 1$. Then for $\nu =
\phi(n)$ we have $a^{\nu} \equiv 1 \bmod n$, contrary to the
hypothesis of the theorem.

\section{Application of Previous Results to Linear
Congruences}\label{s30}%
\index{Congruences!Linear}

The theorems of the present chapter afford us a ready means of
writing down a solution of the congruence
\begin{equation}
ax \equiv c \bmod m. \tag{1}
\end{equation}
We shall consider only the case in which $a$ and $m$ are relatively
prime, since the general case is easily reducible to this one, as we
saw in the preceding chapter.

Since $a$ and $m$ are relatively prime we have the congruences
\begin{gather*}
a^{\lambda(m)} \equiv 1,\quad a^{\phi(m)} \equiv 1 \bmod m. \\
\intertext{Hence either of the numbers $x$,}
x = ca^{\lambda(m)-1},\quad x = ca^{\phi(m)-1},
\end{gather*}
is a representative of the solution of (1). Hence the following
theorem:

\smallskip \emph{If}
\begin{gather*}
ax \equiv c \bmod m \\
\intertext{\emph{is any linear congruence in which $a$ and $m$ are
relatively prime, then either of the numbers $x$,}}
x = ca^{\lambda(m)-1},\quad x = ca^{\phi(m)-1},
\end{gather*}
\emph{is a representative of the solution of the congruence.}

The former representative of the solution is the more convenient of
the two, since the power of $a$ is in general much less in this case
than in the other.

\begin{center}
EXERCISE
\end{center}

\small \begin{enumerate}
\item[ ] Find a solution of $7x \equiv 1 \bmod 2^6 \cdot 3 \cdot 5 \cdot
17.$ Note the greater facility in applying the first of the above
representatives of the solution rather than the second.
\end{enumerate} \normalsize

\section{Application of the Preceding Results to the Theory
of Quadratic Residues}\label{s31}\index{Quadratic residues|(}

In this section we shall apply the preceding results of this chapter
to the problem of finding the solutions of congruences of the form
\begin{equation*}
\alpha z^2 + \beta z + \gamma \equiv 0 \mod \mu
\end{equation*}
where $\alpha, \beta, \gamma, \mu$ are integers. These are called
quadratic congruences.

The problem of the solution of the quadratic congruence (1) can be
reduced to that of the solution of a simpler form of congruence as
follows: Congruence (1) is evidently equivalent to the congruence
\begin{gather}
4\alpha^2 z^2 + 4\alpha\beta z + 4\alpha\gamma \equiv
   0 \mod 4\alpha\mu. \tag{1} \\
\intertext{But this may be written in the form}
(2\alpha z + \beta)^2 \equiv \beta^2 - 4\alpha\gamma
      \mod 4\alpha\mu. \notag \\
\intertext{Now if we put}
2\alpha z + \beta\equiv x \mod 4\alpha\mu \tag{2} \\
\intertext{and}
\beta^2 - 4\alpha\gamma = a,\quad 4\alpha\mu = m, \notag \\
\intertext{we have}
x^2 \equiv a\mod m. \tag{3}
\end{gather}
We have thus reduced the problem of solving the general congruence
(1) to that of solving the binomial congruence (3) and the linear
congruence (2). The solution of the latter may be effected by means
of the results of \S \ref{s30}. We shall therefore confine ourselves
now to a study of congruence (3). We shall make a further limitation
by assuming that $a$ and $m$ are relatively prime, since it is
obvious that the more general case is readily reducible to this one.

The example
\begin{equation*}
x^2 \equiv 3 \mod 5
\end{equation*}
shows at once that the congruence (3) does not always have a
solution. First of all, then, it is necessary to find out in what
cases (3) has a solution. Before taking up the question it will be
convenient to introduce some definitions.

\smallskip\textsc{Definitions.} An integer $a$ is said to be a
quadratic residue modulo $m$ or a quadratic non-residue modulo $m$
according as the congruence
\begin{equation*}
x^2 = a \mod m
\end{equation*}
has or has not a solution. We shall confine our attention to the
case when $m > 2$.\index{Residue}

We shall now prove the following theorem:

\smallskip I.~\emph{If $a$ and $m$ are relatively prime integers, a
necessary condition that $a$ is a quadratic residue modulo $m$ is
that}
\begin{equation*}
a^{\frac{1}{2}\lambda(m)} \equiv 1 \mod m.
\end{equation*}

Suppose that the congruence $x^2 = a \mod m$ has the solution $x =
\alpha$. Then $a^2 \equiv a \mod m$. Hence
\begin{equation*}
a^{\lambda(m)} \equiv a^{\frac{1}{2}\lambda(m)} \mod m.
\end{equation*}
Since $a$ is prime to $m$ it is clear from $\alpha^2 \equiv a \mod
m$ that $a$ is prime to $m$. Hence $\alpha^{\lambda(m)} \equiv 1
\mod m$. Therefore we have
\begin{equation*}
1 \equiv a^{\frac{1}{2}\lambda(m)} \mod m.
\end{equation*}
That is, this is a necessary condition in order that $a$ shall be a
quadratic residue modulo $m$.

In a similar way one may prove the following theorem:

\smallskip II.~\emph{If $a$ and $m$ are relatively prime integers, a
necessary condition that $a$ is a quadratic residue modulo $m$ is
that}
\begin{equation*}
a^{\frac{1}{2}\phi(m)} \equiv 1 \mod m.
\end{equation*}

When $m$ is a prime number $p$ each of the above results takes the
following form: If $a$ is prime to $p$ and is a quadratic residue
modulo $p$, then
\begin{equation*}
a^{\frac{1}{2}(p-1)} \equiv 1 \mod p.
\end{equation*}
We shall now prove the following more complete theorem, without the
use of I or II.

\smallskip III.~\emph{If $p$ is an odd prime number and $a$ is an
integer not divisible by $p$, then $a$ is a quadratic residue or a
quadratic non-residue modulo $p$ according as}
\begin{equation*}
a^{\tfrac{1}{2}(p-1)} \equiv +1 \quad \text{or} \quad
a^{\tfrac{1}{2}(p-1)} \equiv -1 \bmod p.
\end{equation*}

This is called Euler's criterion.\index{Euler's!criterion}

Given a number $a$, not divisible by $p$, we have to determine
whether or not the congruence
\begin{gather}
x^2 \equiv a \bmod p \notag \\
\intertext{has a solution. Let $r$ be any number of the set}
1,\ 2,\ 3,\ \ldots,\ p-1 \tag{A} \\
\intertext{and consider the congruence}
rx \equiv a \bmod p.
\end{gather}
This has always one and just one solution $x$ equal to a number $s$
of the set (A). Two cases can arise: either for every $r$ of the set
(A) the corresponding $s$ is different from $r$ or for some $r$ of
the set (A) the corresponding $s$ is equal to $r$. The former is the
case when $a$ is a quadratic non-residue modulo $p$; the latter is
the case when $a$ is a quadratic residue modulo $p$. We consider the
two cases separately.

In the first case the numbers of the set (A) go in pairs such that
the product of the numbers in the pair is congruent to a modulo $p$.
Hence, taking the product of all $\tfrac{1}{2}(p - 1)$ pairs, we
have
\begin{align*}
1 \cdot 2 \cdot 3 \ldots \overline{p-1} &\equiv
      +a^{\tfrac{1}{2}(p-1)} \bmod p. \\
\intertext{But}
1 \cdot 2 \cdot 3 \ldots \overline{p-1} &= -1 \bmod p. \\
\intertext{Hence}
a^{\tfrac{1}{2}(p-1)} \equiv -1 \bmod p,
\end{align*}
whence the truth of one part of the theorem.

In the other case, namely that in which some $r$ and corresponding
$s$ are equal, we have for this $r$
\begin{gather*}
r^{2} \equiv a \bmod p \\
\intertext{and}
(p - r)^{2} \equiv a \bmod p.
\end{gather*}
Since $x^{2} \equiv a \bmod p$ has at most two solutions it follows
that all the integers in the set (A) except $r$ and $p - r$ fall in
pairs such that the product of the numbers in each pair is congruent
to a modulo $p$. Hence, taking the product of all these pairs, which
are $\frac{1}{2}(p - 1) - 1$ in number, and multiplying by $r(p-r)$
we have
\begin{align*}
1 \cdot 2 \cdot 3 \cdots \overline{p -1}
   &\equiv (p - r) r a^{\frac{1}{2}(p -1) - 1} \bmod p  \\
   &\equiv -r^{2} a^{\frac{1}{2}(p -1) - 1}    \bmod p  \\
   &\equiv -a a^{\frac{1}{2}(p -1) - 1}        \bmod p  \\
   &\equiv -a^{\frac{1}{2}(p -1)}              \bmod p. \\
\intertext{Since $1 \cdot 2 \cdot 3 \cdots \overline{p - 1} \equiv
\bmod p$ we have}
a^{\frac{1}{2}(p -1)} &\equiv + 1 \bmod p
\end{align*}
whence the truth of another part of the theorem.

Thus the proof of the entire theorem is complete.%
\index{Quadratic residues|)}\index{Wilson's theorem|)}

\chapter{PRIMITIVE ROOTS MODULO $m$.}

\section{Exponent of an Integer Modulo $m$}\label{s32}%
\index{Exponent of an integer|(}\index{Primitive roots|(}

Let
\begin{equation*}
a_{1},\ a_{2},\ \cdots,\ a_{\phi(m)} \tag{A}
\end{equation*}
be the set of $\phi(m)$ positive integers not greater than $m$ and
prime to $m$; and let $a$ denote any integer of the set (A). Now any
positive integral power of $a$ is prime to $m$ and hence is
congruent modulo $m$ to a number of the set (A). Hence, among all
the powers of a there must be two, say $a^{n}$ and $a^{\nu}$, $n >
\nu$, which, are congruent to the same integer of the set (A). These
two powers are then congruent to each other; that is,
\begin{equation*}
a^{n} \equiv a^{\nu} \bmod m
\end{equation*}
Since $a^{\nu}$ is prime to $m$ the members of this congruence may
be divided by $a^{\nu}$. Thus we have
\begin{equation*}
a^{n - \nu} \equiv 1 \bmod m.
\end{equation*}
That is, among the powers of $a$ there is one at least which is
congruent to $1$ modulo $m$.

\smallskip Now, in the set of all powers of $a$ which are congruent
to $1$ modulo $m$ there is one in which the exponent is less than in
any other of the set. Let the exponent of this power be $d$, so that
$a^{d}$ is the lowest power of $a$ such that
\begin{equation}
a^{d} \equiv 1 \bmod m. \tag{1}
\end{equation}

We shall now show that if $a^{\alpha} \equiv 1 \bmod m$, then
$\alpha$ is a multiple of $d$. Let us write
\begin{gather}
\alpha = d\delta + \beta, \quad 0 \leqq \beta < d. \notag \\
\intertext{Then}
a^{\alpha} \equiv 1 \bmod m,   \tag{2} \\
a^{d\delta} \equiv 1 \bmod m,  \tag{3} \\
\intertext{the last congruence being obtained by raising (1) to the
power $\delta$. From (3) we have}
a^{d\delta + \beta} \equiv a^{\beta} \bmod m; \notag \\
\intertext{or}
a^{\beta}\equiv 1 \bmod m. \notag
\end{gather}
Hence $\beta = 0$, for otherwise $d$ is not the exponent of the
lowest power of $a$ which is congruent to 1 modulo $m$. Hence $d$ is
a divisor of $\alpha$.

\smallskip These results may be stated as follows:

\smallskip I.~\emph{If $m$ is any integer and $a$ is any integer
prime to $m$, then there exists an integer $d$ such that}
\begin{gather*}
a^d\equiv 1 \bmod m \\
\intertext{\emph{while there is no integer $\beta$ less than $d$ for
which}}
a^\beta\equiv 1 \bmod m. \\
\intertext{\emph{Further, a necessary and sufficient condition
that}}
a^\nu \equiv 1 \bmod m
\end{gather*}
\emph{is that $\nu$ is a multiple of $d$.}

\smallskip \textsc{Definition.} The integer $d$ which is thus
uniquely determined when the two relatively prime integers $a$ and
$m$ are given is called the exponent of $a$ modulo $m$. Also, $d$ is
said to be the exponent to which $a$ belongs modulo $m$.

Now, in every case we have
\begin{equation*}
a^{\phi(m)} \equiv 1,\quad a^{\lambda(m)} \equiv 1 \bmod m,
\end{equation*}
if $a$ and $m$ are relatively prime. Hence from the preceding
theorem we have at once the following:

\smallskip II.~\textit{The exponent $d$ to which $a$ belongs modulo
$m$ is a divisor of both $\phi(m)$ and $\lambda(m)$.}%
\index{Exponent of an integer|)}

\section{Another Proof of Fermat's General Theorem}\label{s33}

In this section we shall give an independent proof of the theorem
that the exponent $d$ of $a$ modulo $m$ is a divisor of $\phi(m)$;
from this result we have obviously a new proof of Fermat's theorem
itself.

We retain the notation of the preceding section. We shall first
prove the following theorem:

\smallskip I.~\textit{The numbers}
\begin{equation}
1,\ a,\ a^2,\ \ldots,\ a^{a-1} \tag{A}
\end{equation}
\textit{are incongruent each to each modulo $m$.}

For, if $a^\alpha \equiv a^\beta \bmod m$, where $0 \leqq \alpha <
d$ and $0 \leqq \beta < d$, $\alpha > \beta$, we have
$a^{\alpha-\beta} \equiv 1 \bmod m$, so that $d$ is not the exponent
to which $a$ belongs modulo $m$, contrary to hypothesis.

\smallskip Now any number of the set (A) is congruent to some number
of the set
\begin{equation}
a_1,\ a_2,\ \ldots,\ a_{\phi(m)}. \tag{B}
\end{equation}
Let us undertake to separate the numbers (B) into classes after the
following manner: Let the first class consist of the numbers
\begin{equation}
\alpha_1,\ \alpha_2,\ \ldots,\ \alpha_{a-1}, \tag{I}
\end{equation}
where $\alpha_i$ is the number of the set (B) to which $a^i$ is
congruent modulo $m$.

If the class (I) does not contain all the numbers of the set (B),
let $a_i$ be any number of the set (B) not contained in (I) and form
the following set of numbers:
\begin{equation}
\alpha_0 a_i,\ \alpha_1 a_i,\ \alpha_2 a_i,\ \ldots,\
  \alpha_{d-1}a_i. \tag{II'}
\end{equation}
We shall now show that no number of this set is congruent to a
number of class (I). For, if so, we should have a congruence of the
form
\begin{gather*}
a_i a_j \equiv a_k \bmod m; \\
\intertext{hence}
a_i a^j \equiv a^k \bmod m, \\
\intertext{so that}
a_i a^d \equiv a^{k+d-j} \bmod m; \\
\intertext{or}
a_i \equiv a^{k+d-j} \bmod m,
\end{gather*}
so that $a_i$ would belong to the set (I) contrary to hypothesis.

Now the numbers of the set (II$'$) are all congruent to numbers of
the set (B); and no two are congruent to the same number of this
set. For, if so, we should have two numbers of (II') congruent; that
is, $\alpha_k a_i \equiv \alpha_j a_i \bmod m,$ or $\alpha_k \equiv
\alpha_j \bmod m;$ and this we have seen to be impossible.

Now let the numbers of the set (B) to which the numbers of the set
(II') are congruent be in order the following:
\begin{equation}
\beta_0,\ \beta_1,\ \beta_2,\ \ldots,\ \beta_{d-1}. \tag{II}
\end{equation}
These numbers constitute our class (II).

If classes (I) and (II) do not contain all the numbers of the set
(B), let $a_j$ be a number of the set ($B$) not contained in either
of the classes (I) and (II): and form the set of numbers
\begin{equation}
\alpha_0 a_j,\ \alpha_1 a_j,\ \alpha_2 a_j,\ \ldots,\
   \alpha_{d-1} a_j. \tag{III'}
\end{equation}
Just as in the preceding case it may be shown that no number of this
set is congruent to a number of class (I) and that the numbers of
(III') are incongruent each to each. We shall also show that no
number of (III') is congruent to a number of class (II). For, if so,
we should have $a_k a_j \equiv \beta_l \bmod m$. Hence $a^k a_j
\equiv a^l a_i \bmod m$; or $a_j \equiv a^{l+d-k} \bmod m$, from
which it follows that $a_j$ is of class (II), contrary to
hypothesis.

Now let the numbers of the set (B) to which the numbers of the set
(III') are congruent be in order the following:
\begin{equation}
\gamma_0,\ \gamma_1,\ \gamma_2,\ \ldots,\ \gamma_{d-1}. \tag{III}
\end{equation}
These numbers form our class (III).

It is now evident that the process may be continued until all the
numbers of the set (B) have been separated into classes, each class
containing $d$ integers, thus:
\begin{equation*}
\begin{matrix}
(\text{I})   & \alpha_0, & \alpha_1, & \alpha_2,
                                         & \ldots, & \alpha_{d-1}, \\
(\text{II})  & \beta_0,  & \beta_1,  & \beta_2,
                                         & \ldots, & \beta_{d-1},  \\
(\text{III}) & \gamma_0, & \gamma_1, & \gamma_2,
                                         & \ldots, & \gamma_{d-1}, \\
&\hdotsfor{5} \\
(\quad  )  & \lambda_0, & \lambda_1, & \lambda_2,
                                           & \ldots, & \lambda_{d-1}.
\end{matrix}
\end{equation*}
The set (B), which consists of $\phi(m)$ integers, has thus been
separated into classes, each class containing $d$ integers. Hence we
conclude that $d$ is a divisor of $\phi(m)$. Thus we have a second
proof of the theorem:

\smallskip II.~\emph{If $a$ and $m$ are any two relatively prime
integers and $d$ is the exponent to which $a$ belongs modulo $m$,
then $d$ is a divisor of $\phi(m)$.}

In our classification of the numbers (B) into the rectangular array
above we have proved much more than theorem II; in fact, theorem II
is to be regarded as one only of the consequences of the more
general result contained in the array.

If we raise each member of the congruence
\begin{equation*}
a^d \equiv 1 \bmod m
\end{equation*}
to the (integral) power $\phi(m)/d$, the preceding theorem leads
immediately to an independent proof of Fermat's general theorem.

\section{Definition of Primitive Roots}\label{s34}

\textsc{Definition.} Let $a$ and $m$ be two relatively prime
integers. If the exponent to which $a$ belongs modulo $m$ is
$\phi(m)$, $a$ is said to be a primitive root modulo $m$ (or a
primitive root of $m$).

In a previous chapter we saw that the congruence
\begin{equation*}
a^{\lambda(m)} \equiv 1 \bmod m
\end{equation*}
is verified by every pair of relatively prime integers $a$ and $m$.
Hence, primitive roots can exist only for such a modulus $m$ as
satisfies the equation
\begin{equation*}
\phi(m) = \lambda(m). \tag{1}
\end{equation*}
We shall show later that this is also sufficient for the existence
of primitive roots.

From the relation which exists in general between the
$\phi$-function and the $\lambda$-function in virtue of the
definition of the latter, it follows that (1) can be satisfied only
when $m$ is a prime power or is twice an odd prime power.

Suppose first that $m$ is a power of $2$, say $m = 2^\alpha$. Then
(1) is satisfied only if $\alpha = 0,\ 1,\ 2$. For $\alpha = 0$ or
$1$, $1$ itself is a primitive root. For $\alpha = 2$, $3$ is a
primitive root. We have therefore left to examine only the cases
\begin{equation*}
m = p^\alpha,\quad m = 2p^\alpha
\end{equation*}
where $p$ is an odd prime number. The detailed study of these cases
follows in the next sections.

\section{Primitive roots modulo $p$.}\label{s35}

We have seen that if $p$ is a prime number and $d$ is the exponent
to which $a$ belongs modulo $p$, then $d$ is a divisor of $\phi(p) =
p - 1$. Now, let
\begin{gather*}
d_1,\ d_2,\ d_3,\ \ldots,\ d_r \\
\intertext{be all the divisors of $p-1$ and let $\psi(d_i)$ denote
the number of integers of the set}
1,\ 2,\ 3,\ \ldots,\ p-1
\end{gather*}
which belong to the exponent $d_i$. If there is no integer of the
set belonging to this exponent, then $\psi(d_i) = 0$.

Evidently every integer of the set belongs to some one and only one
of the exponents $d_1, d_2, \ldots, d_r$. Hence we have the relation
\begin{gather}
\psi(d_1) + \psi(d_2) + \ldots + \psi(d_r) = p-1. \tag{1} \\
\intertext{But}
\phi(d_1) + \phi(d_2) + \ldots + \phi(d_r) = p-1. \tag{2} \\
\intertext{If then we can show that}
\psi(d_i) \leqq \phi(d_i) \tag{3} \\
\intertext{for $i = 1, 2, \ldots, r$, it will follow from a
comparison of (1) and (2) that}
\psi(d_i) = \phi(d_i). \notag
\end{gather}
Accordingly, we shall examine into the truth of (3).

Now the congruence
\begin{equation}
x^{d_i} \equiv 1 \mod p \tag{4}
\end{equation}
has not more than $d_i$ roots. If no root of this congruence belongs
to the exponent $d_i$, then if $\psi(d_i) = 0$ and therefore in this
case we have $\psi(d_i) < \phi(d_i)$. On the other hand if $a$ is a
root of (4) belonging to the exponent $d_i$, then
\begin{equation}
a, a^2, a^3, \ldots, a^{d_i} \tag{5}
\end{equation}
are a set of $d_i$ incongruent roots of (4); and hence they are the
complete set of roots of (4).

But it is easy to see that $a^k$ does or does not belong to the
exponent $d_i$ according as $k$ is or is not prime to $d_i$; for, if
$a^k$ belongs to the exponent $t$, then $t$ is the least integer
such that $kt$ is a multiple of $d_i$. Consequently the number of
roots in the set (5) belonging to the exponent $d_i$ is $\phi(d_i)$.
That is, in this case $\psi(d_i) = \phi(d_i)$. Hence in general
$\psi(d_i) \leqq \phi(d_i)$ Therefore from (1) and (2) we conclude
that
\begin{equation*}
\psi(d_i) = \phi(d_i), \quad i = 1,\ 2,\ \ldots,\ r.
\end{equation*}
The result thus obtained may be stated in the form of the following
theorem:

\smallskip I.~\emph{If $p$ is a prime number and $d$ is any divisor
of $p-1$, then the number of integers belonging to the exponent $d$
modulo $p$ is $\phi(d)$.}

In particular:

\smallskip II.~\emph{There exist primitive roots modulo $p$ and their
number is $\psi(p-i)$.}

\section{Primitive Roots Modulo $p^\alpha$, $p$ an Odd
Prime}\label{s36}

In proving that there exist primitive roots modulo $p^\alpha$, where
$p$ is an odd prime and $\alpha > 1$, we shall need the following
theorem:

I.~\emph{There always exists a primitive root $\gamma$ modulo $p$
for which $\gamma^{p-1}$ is not divisible by $p^2$.}

Let $g$ be any primitive root modulo $p$. If $g^{p-1}$ is not
divisible by $p^2$ our theorem is verified. Then suppose that
$g^{p-1}-1$ is divisible by $p^2$, so that we have
\begin{gather*}
g^{p-1}-1 = kp^2 \\
\intertext{where $k$ is an integer. Then put}
\gamma = g + xp \\
\intertext{where $x$ is an integer. Then $\gamma = g \mod p$, and
hence}
\gamma^h \equiv g^h \mod p;
\end{gather*}
whence we conclude that $\gamma$ is a primitive root modulo $p$. But
\begin{align*}
\gamma^{p-1}-1 &=
     g^{p-1} - 1 + \frac{p-1}{1!}g^{p-2}xp +
        \frac{(p-1)(p-2)}{2!}g^{p-3}x^2p^2 + \ldots \\
  &= p\left(kp + \frac{p-1}{1!}g^{p-2}x +
        \frac{(p-1)(p-2)}{2!}g^{p-3}x^2p + \ldots\right).
\end{align*}
Hence
\begin{equation*}
\gamma^{p-1}-1 \equiv p(-g^{p-2}x) \mod p^2.
\end{equation*}
Therefore it is evident that $x$ can be so chosen that
$\gamma^{p-1}-1$ is not divisible by $p^2$. Hence there exists a
primitive root $\gamma$ modulo $p$ such that $\gamma^{p-1}-1$ is not
divisible by $p^2$. Q.~E.~D.

\smallskip We shall now prove that this integer $\gamma$ is a
primitive root modulo $p^\alpha$, where $\alpha$ is any positive
integer.

If
\begin{equation*}
\gamma^k \equiv 1\mod p.
\end{equation*}
then $k$ is a multiple of $p-1$, since $\gamma$ is a primitive root
modulo $p$. Hence, if
\begin{equation*}
\gamma^k \equiv 1 \mod p^\alpha,
\end{equation*}
then $k$ is a multiple of $p-1$.

Now, write
\begin{equation*}
\gamma^{p-1} = 1 + hp.
\end{equation*}
Since $\gamma^{p-1}$ is not divisible by $p^2$, it follows that $h$
is prime to $p$. If we raise each member of this equation to the
power $\beta p^{\alpha-2}$, $\alpha \stackrel{=}{>}2$, we have
\begin{equation*}
\gamma^{\beta p^{\alpha-2}(p-1)} =
  1 + \beta p^{\alpha-1}h + p^\alpha I,
\end{equation*}
where $I$ is an integer. Then if
\begin{equation*}
\gamma^{\beta p^{\alpha-2}(p-1)} \equiv 1 \mod p^\alpha,
\end{equation*}
$\beta$ must be divisible by $p$. Therefore the exponent of the
lowest power of $\gamma$ which is congruent to $1$ modulo $p^\alpha$
is divisible by $p^{\alpha-1}$. But we have seen that this exponent
is also divisible by $p-1$. Hence the exponent of $\gamma$ modulo
$p^\alpha$ is $p^{\alpha-1}(p-1)$ since $\phi(p^\alpha) =
p^{\alpha-1}(p-1)$. That is, $\gamma$ is a primitive root modulo
$p^\alpha$.

It is easy to see that no two numbers of the set
\begin{equation}
\gamma, \gamma^2, \gamma^3, \ldots, \gamma^{p^{\alpha-1}(p-1)}
\tag{A}
\end{equation}
are congruent modulo $p^\alpha$; for, if so, $\gamma$ would belong
modulo $p^\alpha$ to an exponent less than $p^{\alpha-1}(p-1)$ and
would therefore not be a primitive root modulo $p^\alpha$. Now every
number in the set (A) is prime to $p^\alpha$; their number is
$\phi(p^\alpha) = p^{\alpha -1}(p-1)$. Hence the numbers of the set
(A) are congruent in some order to the numbers of the set (B):
\begin{equation}
a_1,\ a_2,\ a_3,\ \ldots ,\ a_{p^{\alpha-1}(p-1)}, \tag{B}
\end{equation}
where the integers (B) are the positive integers less than
$p^\alpha$ and prime to $p^\alpha$.

But any number of the set (B) is a solution of the congruence
\begin{equation}
x^{p^{\alpha-1} (p-1)} \equiv 1 \bmod p^\alpha. \tag{1}
\end{equation}
Further, every solution of this congruence is prime to $p^\alpha$.
Hence the integers (B) are a complete set of solutions of (1).
Therefore the integers (A) are a complete set of solutions of (1).
But it is easy to see that an integer $\gamma^k$ of the set (A) is
or is not a primitive root modulo $p^\alpha$ according as $k$ is or
is not prime to $p^{\alpha-1} (p-1)$. Hence the number of primitive
roots modulo $p^\alpha$ is $\phi \{p^{\alpha-1} (p-1) \}.$

The results thus obtained may be stated as follows:

\smallskip II.~\emph{If $p$ is any odd prime number and $\alpha$ is
any positive integer, then there exist primitive roots modulo
$p^\alpha$ and their number is $\phi \{ \phi(p^\alpha) \}$}.

\section{Primitive Roots Modulo $2p^\alpha$, $p$ an Odd
Prime}\label{s37}

In this section we shall prove the following theorem:

\emph{If $p$ is any odd prime number and $\alpha$ is any positive
integer, then there exist primitive roots modulo $2p^\alpha$ and
their number is $\phi \{\phi(2 p^{\alpha} )\}.$}

Since $2 p^\alpha$ is even it follows that every primitive root
modulo $2 p^\alpha$ is an odd number. Any odd primitive root modulo
$p^\alpha$ is obviously a primitive root modulo $2p^\alpha$. Again,
if $\gamma$ is an even primitive root modulo $p^\alpha$ then $\gamma
+ p^\alpha$ is a primitive root modulo $2 p^\alpha$. It is evident
that these two classes contain (without repetition) all the
primitive roots modulo $2 p^\alpha$. Hence the theorem follows as
stated above.

\section{Recapitulation}\label{s38}

The results which we have obtained in \S\S \ref{s34}--\ref{s37}
inclusive may be gathered into the following theorem:

\emph{In order that there shall exist primitive roots modulo $m$, it
is necessary and sufficient that $m$ shall have one of the values}
\begin{equation*}
m = 1, 2, 4, p^\alpha, 2p^\alpha
\end{equation*}
\emph{where $p$ is an odd prime and $\alpha$ is a positive integer.}

\emph{If $m$ has one of these values then the number of primitive
roots modulo $m$ is $\phi\{\phi(m)\}$.}

\section{Primitive $\lambda$-roots}\label{s39}%
\index{Primitive roots!$\lambda$-roots|(}

In the preceding sections of this chapter we have developed the
theory of primitive roots in the way in which it is usually
presented. But if one approaches the subject from a more general
point of view the results which may be obtained are more general and
at the same time more elegant. It is our purpose in this section to
develop the more general theory.

\smallskip We have seen that if $a$ and $m$ are any two relatively
prime positive integers, then
\begin{equation*}
a^{\lambda(m)} \equiv 1 \mod m.
\end{equation*}
Consequently there is no integer belonging modulo $m$ to an exponent
greater than $\lambda(m)$. It is natural to enquire if there are any
integers $a$ which belong to the exponent $\lambda(m)$. It turns out
that the question is to be answered in the affirmative, as we shall
show. Accordingly, we introduce the following definition:

\smallskip \textsc{Definition.} If $a^{\lambda(m)}$ is the lowest
power of $a$ which is congruent to $1$ modulo $m$, $a$ is said to be
a primitive $\lambda$-root modulo $m$. We shall also say that it is
a primitive $\lambda$-root of the congruence $x^{\lambda(m)} = 1
\mod m$. To distinguish we may speak of the usual primitive root as
a primitive $\phi$-root modulo $m$.%
\index{Primitive roots!$\phi$-roots}

From the theory of primitive $\phi$-roots already developed it
follows that primitive $\lambda$-roots always exist when $m$ is a
power of any odd prime, and also when $m = 1,\ 2,\ 4$; for, for such
values of $m$ we have $\lambda(m) = \phi(m)$.

We shall next show that primitive $\lambda$-roots exist when $m =
2^{\alpha}$, $a > 2$, by showing that 5 is such a root. It is
necessary and sufficient to prove that $5$ belongs modulo
$2^{\alpha}$ to the exponent $2^{\alpha-2} = \lambda(2^{\alpha})$.
Let $d$ be the exponent to which $5$ belongs modulo $2^{\alpha}$.
Then from theorem II of \S \ref{s32} it follows that $d$ is a
divisor of $2^{\alpha-2} = \lambda(2^{\alpha})$. Hence if $d$ is
different from $2^{\alpha-2}$ it is $2^{\alpha-3}$ or is a divisor
of $2^{\alpha-3}$. Hence if we can show that $5^{2^{\alpha-3}}$ is
not congruent to $1$ modulo $2^{\alpha}$ we will have proved that
$5$ belongs to the exponent $2^{\alpha-2}$. But, clearly,
\begin{gather*}
5^{2^{\alpha-3}} = (1+2^2)^{2^{\alpha-3}}
 = 1+2^{\alpha-1}+ I\cdot 2^{\alpha}, \\
\intertext{where $I$ is an integer. Hence}
5^{2^{\alpha-3}} \not\equiv 1 \bmod 2^{\alpha}.
\end{gather*}
Hence 5 belongs modulo $2^{\alpha}$ to the exponent
$\lambda(2^{\alpha})$.

By means of these special results we are now in position to prove
readily the following general theorem which includes them as special
cases:

\smallskip I.~\emph{For every congruence of the form}
\begin{gather*}
x^{\lambda(m)} \equiv 1 \bmod m
\end{gather*}
\emph{a solution $g$ exists which is a primitive $\lambda$-root, and
for any such solution $g$ there are $\phi\{\lambda(m)\}$ primitive
roots congruent to powers of $g$.}

If any primitive $\lambda$-root $g$ exists, $g^\nu$ is or is not a
primitive $\lambda$-root according as $\nu$ is or is not prime to
$\lambda(m)$; and therefore the number of primitive $\lambda$-roots
which are congruent to powers of any such root $g$ is
$\phi\{\lambda(m)\}$.

The existence of a primitive $\lambda$-root in every case may easily
be shown by induction. In case $m$ is a power of a prime the theorem
has already been established. We will suppose that it is true when
$m$ is the product of powers of $r$ different primes and show that
it is true when $m$ is the product of powers of $r+1$ different
primes; from this will follow the theorem in general.

Put $m = p_1^{\alpha_1} p_2^{\alpha_2} \ldots p_r^{\alpha_r}
p_{r+1}^{\alpha_{r+1}}, \quad n = p_1^{\alpha_1} p_2^{\alpha_2}
\ldots p_r^{\alpha_r}$, and let $h$ be a primitive $\lambda$-root of
\begin{gather}
x^{\lambda(n)} \equiv 1 \mod n. \tag{1} \\
\intertext{Then}
h + ny \notag
\end{gather}
is a form of the same root if $y$ is an integer.

Likewise, if $c$ is any primitive $\lambda$-root of
\begin{equation}
x^\lambda(p_{r+1}^{\alpha_{r+1}})
   \equiv 1 \mod p_{r+1}^{\alpha_{r+1}} \tag{2}
\end{equation}
a form of this root is
\begin{equation*}
c+p_{r+1}^{\alpha_{r+1}}z
\end{equation*}
where $z$ is any integer.

Now, if $y$ and $z$ can be chosen so that
\begin{equation*}
h+ny = c+p_{r+1}^{\alpha_{r+1}}z
\end{equation*}
the number in either member of this equation will be a common
primitive $\lambda$-root of congruences (1) and (2); that is, a
common primitive $\lambda$-root of the two congruences may always be
obtained provided that the equation
\begin{equation*}
p_1^{\alpha_1} \ldots p_r^{\alpha_r}y - p_{r+1}^{\alpha_{r+1}}z = c-h
\end{equation*}
has always a solution in which $y$ and $z$ are integers. That this
equation has such a solution follows readily from theorem III of \S
\ref{s9}; for, if $c-h$ is replaced by $1$, the new equation has a
solution $\bar{y}$, $\bar{z}$; and therefore for $y$ and $z$ we may
take $y = \bar{y}(c-h)$, $z = \bar{z}(c-h)$.

Now let $g$ be a common primitive $\lambda$-root of congruences (1)
and (2) and write
\begin{equation*}
g^\nu \equiv 1 \mod m,
\end{equation*}
where $\nu$ is to be the smallest exponent for which the congruence
is true. Since $g$ is a primitive $\lambda$-root of (1) $\nu$ is a
multiple of $\lambda(p_1^{\alpha_1} \ldots p_r^{\alpha_r})$. Since
$g$ is a primitive $\lambda$-root of (2) $\nu$ is a multiple of
$\lambda\left(p_{r+1}^{\alpha_{r+1}} \right)$. Hence it is a
multiple of $\lambda(m)$. But $g^{\lambda(m)} \equiv 1 \bmod m$;
therefore $\nu = \lambda(m)$. That is, $g$ is a primitive
$\lambda$-root modulo $m$.

The theorem as stated now follows at once by induction.

\smallskip There is nothing in the preceding argument to indicate
that the primitive $\lambda$-roots modulo $m$ are all in a single
set obtained by taking powers of some root $g$; in fact it is not in
general true when $m$ contains more than one prime factor.

By taking powers of a primitive $\lambda$-root $g$ modulo $m$ one
obtains $\phi\{\lambda(m)\}$ different primitive $\lambda$-roots
modulo $m$. It is evident that if $\gamma$ is any one of these
primitive $\lambda$-roots, then the same set is obtained again by
taking the powers of $\gamma$. We may say then that the set thus
obtained is the set belonging to $g$.

\smallskip II.~\emph{If $\lambda(m)>2$ the product of the
$\phi\{\lambda(m)\}$ primitive $\lambda$-roots in the set belonging
to any primitive $\lambda$-root $g$ is congruent to $1$ modulo $m$.}

These primitive $\lambda$-roots are
\begin{gather*}
g,\ g^{c_1},\ g^{c_2},\ \ldots,\ g^{c_\mu} \\
\intertext{where}
1,\ c_1,\ c_2,\ \ldots,\ c_\mu \\
\end{gather*}
are the integers less than $\lambda(m)$ and prime to $\lambda(m)$.
If any one of these is $c$ another is $\lambda(m)-c$, since
$\lambda(m) > 2$. Hence
\begin{gather*}
1 + c_1 + c_2 + \ldots + c_\mu \equiv 0 \bmod \lambda(m). \\
\intertext{Therefore}
g^{1 + c_1 + c_2 + \ldots + c_\mu} \equiv 1 \bmod m.
\end{gather*}
From this the theorem follows.

\smallskip \textsc{Corollary.}\emph{The product of all the
primitive $\lambda$-roots modulo $m$ is congruent to $1$ modulo $m$
when $\lambda(m) > 2$.}\index{Primitive roots!$\lambda$-roots|)}

\begin{center}
EXERCISES
\end{center}

\small\begin{enumerate}
\item[1.] If $x_1$ is the largest value of $x$ satisfying the equation
$\lambda(x) = a$, where $a$ is a given integer, then any solution
$x_2$ of the equation is a factor of $x_1$.

\item[2*.] Obtain an effective rule for solving the equation
$\lambda(x) = a$.

\item[3*.] Obtain an effective rule for solving the equation
$\phi(x) = a$.

\item[4.] A necessary and sufficient condition that $a^{P-1} \equiv 1
\mod P$ for every integer $a$ prime to $P$ is that $P \equiv 1 \mod
\lambda(P)$.

\item[5.] If $a^{P-1} \equiv 1\mod P$ for every a prime to $P$, then
(1) $P$ does not contain a square factor other than $1$, (2) $P$
either is prime or contains at least three different prime factors.

\item[6.] Let $p$ be a prime number. If $a$ is a root of the congruence
$x^q \equiv 1 \mod p$ and $\alpha$ is a root of the congruence
$x^\delta\equiv 1 \mod p$, then $a\alpha$ is a root of the
congruence $x^{d\delta}\equiv 1 \mod p$. If $a$ is a primitive root
of the first congruence and $\alpha$ of the second and if $d$ and
$\delta$ are relatively prime, then $a\alpha$ is a primitive root of
the congruence $x^{d\delta} \equiv 1\mod p$.
\end{enumerate} \normalsize\index{Primitive roots|)}

\chapter{OTHER TOPICS}

\section{Introduction}\label{s40}

The theory of numbers is a vast discipline and no single volume can
adequately treat of it in all of its phases. A short book can serve
only as an introduction; but where the field is so vast such an
introduction is much needed. That is the end which the present
volume is intended to serve; and it will best accomplish this end
if, in addition to the detailed theory already developed, some
account is given of the various directions in which the matter might
be carried further.

To do even this properly it is necessary to limit the number of
subjects considered. Consequently we shall at once lay aside many
topics of interest which would find a place in an exhaustive
treatise. We shall say nothing, for instance, about the vast domain
of algebraic numbers, even though this is one of the most
fascinating subjects in the whole field of
mathematics.\index{Algebraic numbers} Consequently, we shall not
refer to any of the extensive theory connected with the division of
the circle into equal parts.\index{Circle, Division of} Again, we
shall leave unmentioned many topics connected with the theory of
positive integers; such, for instance, is the frequency of prime
numbers in the ordered system of integers---a subject which contains
in itself an extensive and elegant theory.\index{Prime numbers}

In \S\S \ref{s41}--\ref{s44} we shall speak briefly of each of the
following topics: theory of quadratic residues, Galois imaginaries,
arithmetic forms, analytical theory of numbers. Each of these alone
would require a considerable volume for its proper development. All
that we can do is to indicate the nature of the problem in each case
and in some cases to give a few of the fundamental results.

In the remaining three sections we shall give a brief introduction
to the theory of Diophantine equations, developing some of the more
elementary properties of certain special cases. We shall carry this
far enough to indicate the nature of the problem connected with the
now famous Last Theorem of Fermat. The earlier sections of this
chapter are not required as a preliminary to reading this latter
part.

\section{Theory of Quadratic Residues}\label{s41}%
\index{Quadratic residues|(}

Let $a$ and $m$ be any two relatively prime integers. In \S
\ref{s31} we agreed to say that $a$ is a quadratic residue modulo
$m$ or a quadratic non-residue modulo $m$ according as the
congruence
\begin{equation*}
x^2 \equiv a \bmod m
\end{equation*}
has or has not a solution. We saw that if $m$ is chosen equal to an
odd prime number $p$, then $a$ is a quadratic residue modulo $p$ or
a quadratic non-residue modulo $p$ according as
\begin{equation*}
a^{\frac{1}{2} (p-1)} \equiv 1\quad \mathrm{or}\quad
  a^{\frac{1}{2} (p-1)} \equiv -1 \bmod p.
\end{equation*}
This is known as Euler's criterion.\index{Euler's!criterion}

It is convenient to employ the Legendre symbol
\begin{equation*}
\left( \frac{a}{p} \right )
\end{equation*}
to denote the quadratic character of $a$ with respect to $p$.%
\index{Legendre symbol} This symbol is to have the value $+1$ or the
value $-1$ according as $a$ is a quadratic residue modulo $p$ or a
quadratic non-residue modulo $p$. We shall now derive some of the
fundamental properties of this symbol, understanding always that the
numbers in the numerator and the denominator are relatively prime.

From the definition of quadratic residues and non-residues it is
obvious that
\begin{equation}
\left ( \frac{a}{p} \right ) = \left ( \frac{b}{p} \right )
   \quad \text{if}\quad a \equiv b \bmod p. \tag{1}
\end{equation}

It is easy to prove in general that
\begin{equation}
\left ( \frac{a}{p} \right ) \left ( \frac{b}{p} \right ) =
  \left (\frac {ab}{p} \right ). \tag{2}
\end{equation}
This comes readily from Euler's criterion. We have to consider the
three cases
\begin{align*}
\left( \frac{a}{p} \right )    &=+1,&
  \left( \frac{b}{p} \right )  &=+1; &
\left( \frac{a}{p} \right )    &=+1,&
  \left( \frac{b}{p} \right )  &=-1;  \\
&& \left( \frac{a}{p} \right ) &=-1,&
  \left( \frac{b}{p} \right )  &=-1.
\end{align*}
The method will be sufficiently illustrated by the treatment
of the last case. Here we have
\begin{gather*}
a^{\frac 12 (p-1)}\equiv -1 \bmod p,\quad
   b^{\frac 12 (p-1)}\equiv -1 \bmod p. \\
\intertext{Multiplying these two congruences together member by
member we have}
(ab)^{\frac 12 (p-1)} \equiv 1 \bmod p, \\
\intertext{whence}
\left( \frac {ab}{p} \right ) = 1 =
  \left( \frac ap \right ) \left( \frac bp \right ),
\end{gather*}
as was to be proved.

If $m$ is any number prime to $p$ and we write $m$ as the product of
factors
\begin{equation*}
m = \epsilon \cdot 2^\alpha \cdot q' q'' q''' \cdots
\end{equation*}
where $q',\ q'',\ q''',\ \ldots$ are odd primes, $\alpha$ is zero or
a positive integer and $\epsilon$ is $+1$ or $-1$ according as $m$
is positive or negative, we have
\begin{equation}
\left( \frac{m}{p} \right ) =
\left( \frac{\epsilon}{p} \right )
\left( \frac{2}{p} \right ) ^\alpha
\left( \frac{q'}{p} \right )
\left( \frac{q''}{p} \right )
\left( \frac{q'''}{p} \right ) \ldots, \tag{3}
\end{equation}
as one shows easily by repeated application of relation (2).
Obviously,
\begin{equation*}
\left( \frac{1}{p} \right ) = 1.
\end{equation*}
Hence, it follows from (3) that we can readily determine the
quadratic character of $m$ with respect to the odd prime $p$, that
is, the value of
\begin{equation*}
\left( \frac{m}{p} \right ),
\end{equation*}
provided that we know the value of each of the expressions
\begin{equation}
\left( \frac{-1}{p}  \right ),\quad
  \left( \frac{2}{p} \right ),\quad
  \left( \frac{q}{p} \right ),\tag{4}
\end{equation}
where $q$ is an odd prime.

The first of these can be evaluated at once by means of Euler's
criterion; for, we have
\begin{gather*}
\left( \frac{-1}{p} \right ) \equiv
  (-1)^{\frac{1}{2} (p-1)} \bmod p \\
\intertext{and hence}
\left( \frac{-1}{p} \right ) = (-1)^{\frac{1}{2} (p-1)}.
\end{gather*}
Thus we have the following result: The number $-1$ is a quadratic
residue of every prime number of the form $4k + 1$ and a quadratic
non-residue of every prime number of the form $4k + 3$.

The value of the second symbol in (4) is given by the formula
\begin{equation*}
\left( \frac{2}{p} \right ) = (-1)^{\frac{1}{8} (p^2 -1)}.
\end{equation*}
The theorem contained in this equation may be stated in the
following words: The number $2$ is a quadratic residue of every
prime number of either of the forms $8k + 1, 8k + 7$; it is a
quadratic non-residue of every prime number of either of the forms
$8k + 3, 8k + 5$.

The proof of this result is not so immediate as that of the
preceding one. To evaluate the third expression in (4) is still more
difficult. We shall omit the demonstration in both of these cases.
For the latter we have the very elegant relation
\begin{equation*}
\left( \frac{p}{q} \right ) \left( \frac{q}{p} \right ) =
  (-1)^{\frac{1}{4}(p-1)(q-1)}.
\end{equation*}
This equation states the law which connects the quadratic character
of $q$ with respect to $p$ with the quadratic character of $p$ with
respect to $q$. It is known as the Law of Quadratic Reciprocity.
About fifty proofs of it have been given. Its history has been a
very interesting one; see Bachmann's Niedere Zablentheorie, Teil I,
pp.\ 180--318, especially pp.\ 200--206.\index{Bachmann}%
\index{Law of quadratic reciprocity}\index{Quadratic reciprocity}

For a further account of this beautiful and interesting subject we
refer the reader to Bachmann, loc.\ cit., and to the memoirs to
which this author gives reference.\index{Quadratic residues|)}

\section{Galois Imaginaries}\label{s42}%
\index{Galois imaginaries}\index{Imaginaries of Galois}

If one is working in the domain of real numbers the equation
\begin{equation*}
x^2 + 1 = 0
\end{equation*}
has no solution; for there is no real number whose square is $-1$.
If, however, one enlarges the ``number system'' so as to include not
only all real numbers but all complex numbers as well, then it is
true that every algebraic equation has a root. It is on account of
the existence of this theorem for the enlarged domain that much of
the general theory of algebra takes the elegant form in which we
know it.

The question naturally arises as to whether we can make a similar
extension in the case of congruences. The congruence
\begin{equation*}
x^2 = 3 \bmod 5
\end{equation*}
has no solution, if we employ the term solution in the sense in
which we have so far used it. But we may if we choose introduce an
imaginary quantity, or mark, $j$ such that
\begin{equation*}
j^2 \equiv 3 \bmod 5,
\end{equation*}
just as in connection with the equation $x^2 + 1 = 0$ we would
introduce the symbol $i$ having the property expressed by the
equation
\begin{equation*}
i^2 = -1.
\end{equation*}

It is found to be possible to introduce in this way a general set of
imaginaries satisfying congruences with prime moduli; and the new
quantities or marks have the property of combining according to the
laws of algebra.

The quantities so introduced are called Galois imaginaries.

We cannot go into a development of the important theory which is
introduced in this way. We shall be content with indicating two
directions in which it leads.

In the first place there is the general Galois field theory which is
of fundamental importance in the study of certain finite groups. It
may be developed from the point of view indicated here. An excellent
exposition, along somewhat different lines, is to be found in
Dickson's \emph{Linear Groups with an Exposition of the Galois Field
Theory.}\index{Dickson}

Again, the whole matter may be looked upon from the geometric point
of view. In this way we are led to the general theory of finite
geometries, that is, geometries in which there is only a finite
number of points. For a development of the ideas which arise here
see Veblen and Young's \emph{Projective Geometry} and the memoir by
Veblen and Bussey in the Transactions of the American Mathematical
Society, vol.\ 7, pp.\ 241--259.\index{Bussey}\index{Veblen}%
\index{Young}

\section{Arithmetic Forms}\label{s43}%
\index{Arithmetic forms|(}\index{Forms|(}

The simplest arithmetic form is $ax + b$ where $a$ and $b$ are fixed
integers different from zero and $x$ is a variable integer. By
varying $x$ in this case we have the terms of an arithmetic
progression. We have already referred to Dirichlet's celebrated
theorem which asserts that the form $ax + b$ has an infinite number
of prime values if only $a$ and $b$ are relatively
prime.\index{Dirichlet} This is an illustration of one type of
theorem connected with arithmetic forms in general, namely, those in
which it is asserted that numbers of a given form have in addition a
given property.\index{Prime numbers}

Another type of theorem is illustrated by a result stated in \S
\ref{s41}, provided that we look at that result in the proper way.
We saw that the number $2$ is a quadratic residue of every prime of
either of the forms $8k + 1$ and $8k + 7$ and a quadratic
non-residue of every prime of either of the forms $8k + 3$ and $8k +
5$. We may state that result as follows: A given prime number of
either of the forms $8k + 1$ and $8k + 7$ is a divisor of some
number of the form $x^2 - 2$, where $x$ is an integer; no prime
number of either of the forms $8k + 3$ and $8k + 5$ is a divisor of
a number of the form $x^2 - 2$, where $x$ is an integer.

The result just stated is a theorem in a discipline of vast extent,
namely, the theory of quadratic forms. Here a large number of
questions arise among which are the following: What numbers can be
represented in a given form? What is the character of the divisors
of a given form? As a special case of the first we have the question
as to what numbers can be represented as the sum of three squares.
To this category belong also the following two theorems: Every
positive integer is the sum of four squares of integers; every prime
number of the form $4n + 1$ may be represented (and in only one way)
as the sum of two squares.\index{Prime numbers}

For an extended development of the theory of quadratic forms we
refer the reader to Bachmann's Arithmetik der Quadratischen Formen
of which the first part has appeared in a volume of nearly seven
hundred pages.\index{Bachmann}

It is clear that one may further extend the theory of arithmetic
forms by investigating the properties of those of the third and
higher degrees. Naturally the development of this subject has not
been carried so far as that of quadratic forms; but there is a
considerable number of memoirs devoted to various parts of this
extensive field, and especially to the consideration of various
special forms.

Probably the most interesting of these special forms are the
following:
\begin{equation*}
\alpha^n + \beta^n , \quad
  \frac{\alpha^n - \beta^n}{\alpha - \beta} =
    \alpha^{n-1} + \alpha^{n-2} \beta + \cdots + \beta^{n-1},
\end{equation*}
where $\alpha$ and $\beta$ are relatively prime integers, or, more
generally, where $\alpha$ and $\beta$ are the roots of the quadratic
equation $x^2 - ux + v = 0$ where $u$ and $v$ are relatively prime
integers. A development of the theory of these forms has been given
by the present author in a memoir published in 1913 in the Annals of
Mathematics, vol.\ 13, pp.\ 30--70.%
\index{Arithmetic forms|)}\index{Carmichael}\index{Forms|)}%
\index{Quadratic forms}

\section{Analytical theory of numbers}\label{s44}%
\index{Analytical theory of numbers|(}

Let us consider the function
\begin{equation*}
P(x) = \frac{1}{\prod_{k=0}^\infty (1-x^{2^k} )} , \quad
  |x|\leqq \rho < 1.
\end{equation*}
It is clear that we have
\begin{align*}
P(x) = \prod_{k=0}^\infty \frac{1}{(1-x^{2^k} )} &=
  \prod_{k=0}^\infty
       ( 1 + x^{2k} + x^{2\cdot 2^k} + x^{3\cdot 2^k} + \cdots ) \\
&= \sum_{s=0}^\infty G(s) x^s,
\end{align*}
where $G(0) = 1$ and $G(s)$ (for $s$ greater than $0$) is the number
of ways in which the positive integer $s$ may be separated into like
or distinct summands each of which is a power of $2$.

We have readily
\begin{equation*}
(1-x)\sum_{s=0}^\infty G(s) x^s = (1-x)P(x) = P(x^2) =
    \sum_{s=0}^\infty x^{2^s};
\end{equation*}
whence
\begin{equation}
G(2s + 1) = G(2s) = G(2s - 1) + G(s), \tag{A}
\end{equation}
as one readily verifies by equating coefficients of like powers of
$x$. From this we have in particular
\begin{gather*}
G(0) = 1, \quad G(1) = 1, \quad G(2) = 2, \quad G(3) = 2, \\
G(4) = 4, \quad G(5) = 4, \quad G(6) = 6, \quad G(7) = 6.
\end{gather*}
Thus in (A) we have recurrence relations by means of which we may
readily reckon out the values of the number theoretic function
$G(s)$. Thus we may determine the number of ways in which a given
positive integer $s$ may be represented as a sum of powers of $2$.

We have given this example as an elementary illustration of the
analytical theory of numbers, that is, of that part of the theory of
numbers in which one employs (as above) the theory of a continuous
variable or some analogous theory in order to derive properties of
sets of integers. This general subject has been developed in several
directions. For a systematic account of it the reader is referred to
Bachmann's Analytische Zahlentheorie.%
\index{Analytical theory of numbers|)}\index{Bachmann}

\section{Diophantine equations}\label{s45}%
\index{Diophantine equations}\index{Equations!Diophantine}

If $f(x, y, z, \ldots)$ is a polynomial in the variables $x, y, z,
\ldots$ with integral coefficients, then the equation
\begin{equation*}
f(x, y, z, \ldots) = 0
\end{equation*}
is called a Diophantine equation when we look at it from the point
of view of determining the integers (or the positive integers) $x,
y, z, \ldots$ which satisfy it. Similarly, if we have several such
functions $f_i(x, y, z, \ldots)$, in number less than the number of
variables $x, y, z, \ldots$, then the set of equations
\begin{equation*}
f_i(x, y, z, \ldots) = 0,\quad  i = i, 2, \ldots,
\end{equation*}
is said to be a Diophantine system of equations. Any set of integers
$x, y, z, \ldots$ which satisfies the equation [system] is said to
be a solution of the equation [system].

We may likewise define Diophantine inequalities by replacing the
sign of equality above by the sign of inequality. But little has
been done toward developing a theory of Diophantine inequalities.
Even for Diophantine equations the theory is in a rather fragmentary
state.

In the next two sections we shall illustrate the nature of the ideas
and the methods of the theory of Diophantine equations by developing
some of the results for two important special cases.

\section{Pythagorean triangles}\label{s46}%
\index{Pythagorean triangles|(}

\textsc{Definitions.} If three positive integers $x, y, z$ satisfy
the relation
\begin{equation}
x^2 + y^2 = z^2 \tag{1}
\end{equation}
they are said to form a Pythagorean triangle or a numerical right
triangle; $z$ is called the hypotenuse of the triangle and $x$ and
$y$ are called its legs. The area of the triangle is said to be
$\frac{1}{2} xy$.\index{Triangles, Numerical}

We shall determine the general form of the integers $x$, $y$, $z$,
such that equation (1) may be satisfied. Let us denote by $\nu$ the
greatest common divisor of $x$ and $y$ in a particular solution of
(1). Then $\nu$ is a divisor of $z$ and we may write
\begin{equation*}
x = \nu u, \quad y = \nu v,\quad  z = \nu w.
\end{equation*}
Substituting these values in (1) and reducing we have
\begin{equation}
u^2 + v^2 = w^2, \tag{2}
\end{equation}
where $u, v, w$ are obviously prime each to each, since $u$ and $v$
have the greatest common divisor $1$.

Now an odd square is of the form $4k + 1$. Hence the sum of two odd
squares is divisible by $2$ but not by $4$; and therefore the sum of
two odd squares cannot be a square. Hence one of the numbers $u$,
$v$ is even. Suppose that $u$ is even and write equation (2) in the
form
\begin{equation}
u^2 = (w - v)(w + v). \tag{3}
\end{equation}
Every common divisor of $w - v$ and $w + v$ is a divisor of their
difference $2v$. Therefore, since $w$ and $v$ are relatively prime,
it follows that $2$ is the greatest common divisor of $w - v$ and $w
+ v$. Then from (3) we see that each of these numbers is twice a
square, so that we may write
\begin{equation*}
w - v = 2b^2,\quad w + v = 2a^2
\end{equation*}
where $a$ and $b$ are relatively prime integers. From these two
equations and equation (3) we have
\begin{equation}
w = a^2 + b^2, \quad v = a^2 -b^2,\quad u = 2ab. \tag{4}
\end{equation}
Since $u$ and $v$ are relatively prime it is evident that one of the
numbers $a$, $b$ is even and the other odd.

The forms of $u$, $v$, $w$ given in (4) are necessary in order that
(2) may be satisfied. A direct substitution in (2) shows that this
equation is indeed satisfied by these values. Hence we have in (4)
the general solution of (2) where $u$ is restricted to be even. A
similar solution would be obtained if $v$ were restricted to be
even. Therefore \emph{the general solution of (1) is
\begin{gather*}
x = 2\nu ab,\quad y = \nu (a^2 - b^2),\quad z = \nu (a^2 + b^2)\\
\intertext{and}
x = 2\nu (a^2 - b^2 ),\quad y = 2\nu ab,\quad z = \nu (a^2 + b^2)
\end{gather*}
where $a$, $b$, $\nu$ are arbitrary integers except that $a$ and $b$
are relatively prime and one of them is even and the other odd.}

By means of this general solution of (1) we shall now prove the
following theorem:

\smallskip I.~\emph{There do not exist integers $m$, $n$, $p$, $q$,
all different from zero, such that}
\begin{equation}
q^2 + n^2 = m^2 , \quad m^2 + n^2 = p^2. \tag{5}
\end{equation}

It is obvious that an equivalent theorem is the following:

\smallskip II.~\emph{There do not exist integers $m$, $n$, $p$, $q$,
all different from zero such that}
\begin{equation}
p^2 + q^2 = 2m^2, \quad p^2 - q^2 = 2n^2. \tag{6}
\end{equation}

Obviously, we may without loss of generality take $m$, $n$, $p$, $q$
to be positive; and this we do.

The method of proof is to assume the existence of integers
satisfying equations (5) and (6) and to show that we are thus led to
a contradiction. The argument we give is an illustration of Fermat's
famous method of ``infinite descent.''%
\index{Descent, Infinite}\index{Fermat}\index{Infinite descent}

If any two of the numbers $p$, $q$, $m$, $n$ have a common prime
factor $t$, it follows at once from (5) and (6) that all four of
them have this factor. For, consider an equation in (5) or in (6) in
which these two numbers occur; this equation contains a third
number, and it is readily seen that this third number is divisible
by $t$. Then from one of the equations containing the fourth number
it follows that this fourth number is divisible by $t$. Now let us
divide each equation of system (6) through by $t^2$; the resulting
system is of the same form as (6). If any two numbers in this
resulting system have a common prime factor $t_1$, we may divide
through by $t_1^2$; and so on. Hence if a pair of simultaneous
equations (6) exists then there exists a pair of equations of the
same form in which no two of the numbers $m$, $n$, $p$, $q$ have a
common factor other than unity. Let this system of equations be
\begin{equation}
p_1^2 + q_1^2 = 2m_1^2, \quad p_1^2 - q_1^2 = 2n_1^2. \tag{7}
\end{equation}

From the first equation in (7) it follows that $p_1$ and $q_1$ are
both even or both odd; and, since they are relatively prime, it
follows that they are both odd. Evidently $p_1 > q_1$. Then we may
write
\begin{equation*}
p_1 = q_1 + 2\alpha,
\end{equation*}
where $\alpha$ is a positive integer. If we substitute this value of
$p_1$ in the first equation of (7), the result may readily be put in
the form
\begin{equation}
(q_1 + \alpha)^2 + a^2 = m_1^2. \tag{8}
\end{equation}
Since $q_1$ and $m_1$ have no common prime factor it is easy to see
from this equation that $\alpha$ is prime to both $q_1$ and $m_1$,
and hence that no two of the numbers $q_1 + \alpha, \alpha, m_1$
have a common factor.

Now we have seen that if $a$, $b$, $c$ are positive integers no two
of which have a common prime factor, while
\begin{equation*}
a^2 + b^2 = c^2,
\end{equation*}
then there exist relatively prime integers $r$ and $s$, $r > s$,
such that
\begin{gather}
c = r^2 + s^2,\quad a = 2rs,\quad b = r^2 - s^2 \notag \\
\intertext{or}
c = r^2 + s^2,\quad a = r^2 - s^2,\quad b = 2rs. \notag \\
\intertext{Hence from (8) we see that we may write}
q_1 + \alpha = 2rs,\quad \alpha = r^2 - s^2 \tag{9} \\
\intertext{or}
q_1 + \alpha = r^2 - s^2, \alpha = 2rs. \tag{10} \\
\intertext{In either case we have}
p_1^2 - q_1^2 = (p_1 - q_1)(p_1 + q_1) =
  2\alpha \cdot 2(q_1 + \alpha) = 8rs(r^2 - s^2). \notag \\
\intertext{If we substitute in the second equation of (7) and divide
by 2 we have} 4rs(r^2 - s^2) = n_1^2. \notag
\end{gather}

From this equation and the fact that $r$ and $s$ are relatively
prime it follows at once that $r$, $s$, $r^2 - s^2$ are all square
numbers; say,
\begin{gather}
r = u^2,\quad s = v^2,\quad r^2 - s^2 = w^2. \notag \\
\intertext{Now $r - s$ and $r + s$ can have no common factor other
than 1 or 2; hence from}
w^2 = (r^2-s^2) = (r-s)(r+s) = (u^2-v^2)(u^2+v^2) \notag \\
\intertext{we see that either}
u^2 + v^2 = 2w_1^2,\quad  u^2 - v^2 = 2w_2^2 \tag{11} \\
\intertext{or}
u^2 + v^2 = w_1^2,\quad u^2 - v^2 = w_2^2. \notag \\
\intertext{And if it is the latter case which arises, then}
w_1^2 + w_2^2 = 2u^2,\quad w_1^2 - w_2^2 = 2v^2. \tag{12}
\end{gather}
Hence, assuming equations of the form (6) we are led either to
equations (11) or to equations (12); that is, we are led to new
equations of the form with which we started. Let us write the
equations thus:
\begin{equation}
p_2^2 + q_2^2 = 2m_2^2,\quad p_2^2 - q_2^2 = 2n_2^2; \tag{13}
\end{equation}
that is, system (13) is identical with that one of systems (11),
(12) which actually arises.

Now from (9) and (10) and the relations $p_1 = q_1 + 2\alpha, r
> s$, we see that
\begin{gather*}
p_1 = 2rs + r^2 - s^2 > 2s^2 + r^2 - s^2 =
   r^2 + s^2 = u^4 + v^4. \\
\intertext{Hence $u < p_1$. Also,}
w_1^2 \leqq w^2 \leqq r+s < r^2 + s^2.
\end{gather*}
Hence $w_1 < p_1$. Since $u$ and $w_1$ are both less than $p_1$ it
follows that $p_2$ is less than $p_1$. Hence, obviously, $p_2 < p$.
Moreover, it is clear that all the numbers $p_2, q_2, m_2, n_2$ are
different from zero.

From these results we have the following conclusion: If we assume a
system of the form (6) we are led to a new system (13) of the same
form; and in the new system $p_2$ is less than $p$.

Now if we start with (13) and carry out a similar argument
we shall be led to a new system
\begin{gather*}
p_3^2 + q_3^2 = 2m_3^2,\quad p_3^2 - q_3^2 = 2n_3^2,
\end{gather*}
with the relation $p_3 < p_2$, starting from this last system we
shall be led to a new one of the same form, with a similar relation
of inequality; and so on \emph{ad infinitum.} But, since there is
only a finite number of positive integers less than the given
positive integer $p$ this is impossible. We are thus led to a
contradiction; whence we conclude at once to the truth of II and
likewise of I.

By means of theorems I and II we may readily prove the following
theorem:

\smallskip III.~\emph{The area of a numerical right triangle is
never a square number.}

Let the sides and hypotenuse of a numerical right triangle be $u, v,
w$, respectively. The area of this triangle is $\frac{1}{2} uv$. If
we assume this to be a square number $t^2$ we shall have the
following simultaneous Diophantine equations
\begin{equation}
u^2 + v^2 = w^2,\quad uv = 2t^2. \tag{14}
\end{equation}
We shall prove our theorem by showing that the assumption of such a
system leads to a contradiction.

If any two of the numbers $u, v, w$ have a common prime factor $p$
then the remaining one also has this factor, as one sees readily
from the first equation in (14). From the second equation in (14) it
follows that $t$ also has the same factor. Then if we put $u = pu_1,
v = pv_1, w = pw_1, t = pt_1$, we have
\begin{equation*}
u_1^2 + v_1^2 = w_1^2,\quad u_1 v_1 = 2t_1^2,
\end{equation*}
a system of the same form as (14). It is clear that we may start
with this new system and proceed in the same manner as before, and
so on, until we arrive at a system
\begin{equation}
\bar{u}^2 + \bar{v}^2 = \bar{w}^2,\quad
  \bar{u}\bar{v} = 2\bar{t}^2, \tag{15}
\end{equation}
where $\bar{u}$, $\bar{v}$, $\bar{w}$ are prime each to each.

Now the general solution of the first equation (15) may be written
in one of the forms
\begin{gather*}
\bar{u} = 2ab,\quad \bar{v} = a^2 - b^2,\quad \bar{w} = a^2 + b^2 \\
\bar{u} = a^2 b^2,\quad \bar{v} = 2ab, \quad \bar{w} = a^2 + b^2. \\
\intertext{Then from the second equation in (15) we have}
\bar{t}^2 = ab(a^2 - b^2 ) = ab(a-b)(a+b).
\end{gather*}
It is easy to see that no two of the numbers $a$, $b$, $a - b$, $a +
b$ in the last member of this equation have a common factor; for, if
so, $\bar{u}$ and $\bar{v}$ would have a common factor, contrary to
hypothesis. Hence each of these four numbers is a square. That is,
we have equations of the form
\begin{gather*}
a = m^2,\quad b = n^2,\quad a + b = p^2,\quad a - b = q^2; \\
\intertext{whence}
m^2 - n^2 = q^2,\quad m^2 + n^2 = p^2.
\end{gather*}
But, according to theorem I, no such system of equations can exist.
That is, the assumption of equations (14) leads to a contradiction.
Hence the theorem follows as stated above.%
\index{Pythagorean triangles|)}

\section{The Equation $x^n + y^n = z^n$.}\label{s47}%
\index{Equation $x^n + y^n = z^n$|(}\index{Fermat's!last theorem}

The following theorem, which is commonly known as Fermat's Last
Theorem, was stated without proof by Fermat in the seventeenth
century:

\smallskip\emph{If n is an integer greater than 2 there do not exist
integers x, y, z, all different from zero, such that}
\begin{equation}
x^n + y^n = z^n. \tag{1}
\end{equation}

No general proof of this theorem has yet been given. For various
special values of $n$ the proof has been found; in particular, for
every value of $n$ not greater than 100.

In the study of equation (1) it is convenient to make some
preliminary reductions. If there exists any particular solution of
(1) there exists also a solution in which $x$, $y$, $z$ are prime
each to each, as one may show readily by the method employed in the
first part of \S \ref{s46}. Hence in proving the impossibility of
equation (1) it is sufficient to treat only the case in which $x$,
$y$, $z$ are prime each to each.

Again, since $n$ is greater than 2 it must contain the factor
4 or an odd prime factor $p$. If $n$ contains the factor $p$ we write
$n = mp$, whence we have
\begin{gather*}
(x^m)^p + (y^m)^p = (z^m)^p). \\
\intertext{If $n$ contains the factor 4 we write $n = 4m$, whence we
have}
(x^m)^4 + (y^m)^4 = (z^m)^4.
\end{gather*}
From this we see that in order to prove the impossibility of (1) in
general it is sufficient to prove it for the special cases when $n$
is 4 and when $n$ is an odd prime $p$. For the latter case the proof
has not been found. For the former case we give a proof below. The
theorem may be stated as follows:

\smallskip I.~\emph{There are no integers $x, y, z$, all different
from zero, such that}
\begin{equation*}
x^4 + y^4 = z^4.
\end{equation*}

This is obviously a special case of the more general theorem:

\smallskip II.~\emph{There are no integers $p$, $q$, $\alpha$, all
different from zero, such that}
\begin{equation}
p^4 - q^4 = \alpha^2. \tag{2}
\end{equation}

The latter theorem is readily proved by means of theorem III of \S
\ref{s46}. For, if we assume an equation of the form (2), we have
\begin{gather}
(p^4 - q^4)p^2 q^2 = p^2 q^2 \alpha^2. \tag{3} \\
\intertext{But, obviously,}
(2p^2 q^2)^2 + (p^4 - q^4)^2 = (p^4 + q^4)^2. \tag{4}
\end{gather}
Now, from (3) we see that the numerical right triangle determined by
(4) has its area $p^2 q^2(p^4 - q^4)$ equal to the square number
$p^2 q^2 \alpha^2$. But this is impossible. Hence no equation of the
form (2) exists.

\begin{center}
EXERCISES
\end{center}

\begin{enumerate}
\item[1.] Show that the equation $\alpha^4 + 4\beta^4 = \gamma^2$ is
impossible in integers $\alpha$, $\beta$, $\gamma$ all of which are
different from zero.

\item[2.] Show that the system $p^2 - q^2 = km^2$, $p^2 + q^2 = kn^2$
impossible in integers $p$, $q$, $k$, $m$, $n$, all of which are
different from zero.

\item[3*.] Show that neither of the equations $m^4 - 4n^4 = \pm t^2$
is possible in integers $m$, $n$, $t$, all of which are different
from zero.

\item[4*.] Prove that the area of a numerical right triangle is not
twice a square number.

\item[5*.] Prove that the equation $m^4 + n^4 = \alpha^2$ is not
possible in integers $m$, $n$, $\alpha$ all of which are different
from zero.

\item[6*.] In the numerical right triangle $a^2 + b^2 = c^2$,
not more than one of the numbers $a$, $b$, $c$ is a square.

\item[7.] Prove that the equation $x^{2k} + y^{2k} = z^{2k}$ implies
an equation of the form $m^k + n^k = 2^{k-2} t^k$.

\item[8.] Find the general solution in integers of the equation
$x^2 + 2y^2 = t^2$.

\item[9.] Find the general solution in integers of the equation
$x^2 + y^2 = z^4$.

\item[10.] Obtain solutions of each of the following Diophantine
equations:
\begin{align*}
x^3 +  y^3 +  z^3 &= 2t^3, \\
x^3 + 2y^3 + 3z^3 &=  t^3, \\
x^4 +  y^4 + 4z^4 &=  t^4, \\
x^4 +  y^4 +  z^4 &= 2t^4.
\end{align*}
\end{enumerate}\index{Equation $x^n + y^n = z^n$|)}

\addcontentsline{toc}{chapter}{Index}
\printindex


\newpage
\chapter{PROJECT GUTENBERG "SMALL PRINT"}
\small
\pagenumbering{gobble}
\begin{verbatim}

End of Project Gutenberg's The Theory of Numbers, by Robert D. Carmichael

*** END OF THIS PROJECT GUTENBERG EBOOK THE THEORY OF NUMBERS ***

***** This file should be named 13693-pdf.pdf or 13693-pdf.zip *****
This and all associated files of various formats will be found in:
        http://www.gutenberg.org/1/3/6/9/13693/

Produced by David Starner, Joshua Hutchinson, John Hagerson,

Updated editions will replace the previous one--the old editions
will be renamed.

Creating the works from public domain print editions means that no
one owns a United States copyright in these works, so the Foundation
(and you!) can copy and distribute it in the United States without
permission and without paying copyright royalties.  Special rules,
set forth in the General Terms of Use part of this license, apply to
copying and distributing Project Gutenberg-tm electronic works to
protect the PROJECT GUTENBERG-tm concept and trademark.  Project
Gutenberg is a registered trademark, and may not be used if you
charge for the eBooks, unless you receive specific permission.  If you
do not charge anything for copies of this eBook, complying with the
rules is very easy.  You may use this eBook for nearly any purpose
such as creation of derivative works, reports, performances and
research.  They may be modified and printed and given away--you may do
practically ANYTHING with public domain eBooks.  Redistribution is
subject to the trademark license, especially commercial
redistribution.



*** START: FULL LICENSE ***

THE FULL PROJECT GUTENBERG LICENSE
PLEASE READ THIS BEFORE YOU DISTRIBUTE OR USE THIS WORK

To protect the Project Gutenberg-tm mission of promoting the free
distribution of electronic works, by using or distributing this work
(or any other work associated in any way with the phrase "Project
Gutenberg"), you agree to comply with all the terms of the Full Project
Gutenberg-tm License available with this file or online at
  www.gutenberg.org/license.


Section 1.  General Terms of Use and Redistributing Project Gutenberg-tm
electronic works

1.A.  By reading or using any part of this Project Gutenberg-tm
electronic work, you indicate that you have read, understand, agree to
and accept all the terms of this license and intellectual property
(trademark/copyright) agreement.  If you do not agree to abide by all
the terms of this agreement, you must cease using and return or destroy
all copies of Project Gutenberg-tm electronic works in your possession.
If you paid a fee for obtaining a copy of or access to a Project
Gutenberg-tm electronic work and you do not agree to be bound by the
terms of this agreement, you may obtain a refund from the person or
entity to whom you paid the fee as set forth in paragraph 1.E.8.

1.B.  "Project Gutenberg" is a registered trademark.  It may only be
used on or associated in any way with an electronic work by people who
agree to be bound by the terms of this agreement.  There are a few
things that you can do with most Project Gutenberg-tm electronic works
even without complying with the full terms of this agreement.  See
paragraph 1.C below.  There are a lot of things you can do with Project
Gutenberg-tm electronic works if you follow the terms of this agreement
and help preserve free future access to Project Gutenberg-tm electronic
works.  See paragraph 1.E below.

1.C.  The Project Gutenberg Literary Archive Foundation ("the Foundation"
or PGLAF), owns a compilation copyright in the collection of Project
Gutenberg-tm electronic works.  Nearly all the individual works in the
collection are in the public domain in the United States.  If an
individual work is in the public domain in the United States and you are
located in the United States, we do not claim a right to prevent you from
copying, distributing, performing, displaying or creating derivative
works based on the work as long as all references to Project Gutenberg
are removed.  Of course, we hope that you will support the Project
Gutenberg-tm mission of promoting free access to electronic works by
freely sharing Project Gutenberg-tm works in compliance with the terms of
this agreement for keeping the Project Gutenberg-tm name associated with
the work.  You can easily comply with the terms of this agreement by
keeping this work in the same format with its attached full Project
Gutenberg-tm License when you share it without charge with others.

1.D.  The copyright laws of the place where you are located also govern
what you can do with this work.  Copyright laws in most countries are in
a constant state of change.  If you are outside the United States, check
the laws of your country in addition to the terms of this agreement
before downloading, copying, displaying, performing, distributing or
creating derivative works based on this work or any other Project
Gutenberg-tm work.  The Foundation makes no representations concerning
the copyright status of any work in any country outside the United
States.

1.E.  Unless you have removed all references to Project Gutenberg:

1.E.1.  The following sentence, with active links to, or other immediate
access to, the full Project Gutenberg-tm License must appear prominently
whenever any copy of a Project Gutenberg-tm work (any work on which the
phrase "Project Gutenberg" appears, or with which the phrase "Project
Gutenberg" is associated) is accessed, displayed, performed, viewed,
copied or distributed:

This eBook is for the use of anyone anywhere at no cost and with
almost no restrictions whatsoever.  You may copy it, give it away or
re-use it under the terms of the Project Gutenberg License included
with this eBook or online at www.gutenberg.org

1.E.2.  If an individual Project Gutenberg-tm electronic work is derived
from the public domain (does not contain a notice indicating that it is
posted with permission of the copyright holder), the work can be copied
and distributed to anyone in the United States without paying any fees
or charges.  If you are redistributing or providing access to a work
with the phrase "Project Gutenberg" associated with or appearing on the
work, you must comply either with the requirements of paragraphs 1.E.1
through 1.E.7 or obtain permission for the use of the work and the
Project Gutenberg-tm trademark as set forth in paragraphs 1.E.8 or
1.E.9.

1.E.3.  If an individual Project Gutenberg-tm electronic work is posted
with the permission of the copyright holder, your use and distribution
must comply with both paragraphs 1.E.1 through 1.E.7 and any additional
terms imposed by the copyright holder.  Additional terms will be linked
to the Project Gutenberg-tm License for all works posted with the
permission of the copyright holder found at the beginning of this work.

1.E.4.  Do not unlink or detach or remove the full Project Gutenberg-tm
License terms from this work, or any files containing a part of this
work or any other work associated with Project Gutenberg-tm.

1.E.5.  Do not copy, display, perform, distribute or redistribute this
electronic work, or any part of this electronic work, without
prominently displaying the sentence set forth in paragraph 1.E.1 with
active links or immediate access to the full terms of the Project
Gutenberg-tm License.

1.E.6.  You may convert to and distribute this work in any binary,
compressed, marked up, nonproprietary or proprietary form, including any
word processing or hypertext form.  However, if you provide access to or
distribute copies of a Project Gutenberg-tm work in a format other than
"Plain Vanilla ASCII" or other format used in the official version
posted on the official Project Gutenberg-tm web site (www.gutenberg.org),
you must, at no additional cost, fee or expense to the user, provide a
copy, a means of exporting a copy, or a means of obtaining a copy upon
request, of the work in its original "Plain Vanilla ASCII" or other
form.  Any alternate format must include the full Project Gutenberg-tm
License as specified in paragraph 1.E.1.

1.E.7.  Do not charge a fee for access to, viewing, displaying,
performing, copying or distributing any Project Gutenberg-tm works
unless you comply with paragraph 1.E.8 or 1.E.9.

1.E.8.  You may charge a reasonable fee for copies of or providing
access to or distributing Project Gutenberg-tm electronic works provided
that

- You pay a royalty fee of 20% of the gross profits you derive from
     the use of Project Gutenberg-tm works calculated using the method
     you already use to calculate your applicable taxes.  The fee is
     owed to the owner of the Project Gutenberg-tm trademark, but he
     has agreed to donate royalties under this paragraph to the
     Project Gutenberg Literary Archive Foundation.  Royalty payments
     must be paid within 60 days following each date on which you
     prepare (or are legally required to prepare) your periodic tax
     returns.  Royalty payments should be clearly marked as such and
     sent to the Project Gutenberg Literary Archive Foundation at the
     address specified in Section 4, "Information about donations to
     the Project Gutenberg Literary Archive Foundation."

- You provide a full refund of any money paid by a user who notifies
     you in writing (or by e-mail) within 30 days of receipt that s/he
     does not agree to the terms of the full Project Gutenberg-tm
     License.  You must require such a user to return or
     destroy all copies of the works possessed in a physical medium
     and discontinue all use of and all access to other copies of
     Project Gutenberg-tm works.

- You provide, in accordance with paragraph 1.F.3, a full refund of any
     money paid for a work or a replacement copy, if a defect in the
     electronic work is discovered and reported to you within 90 days
     of receipt of the work.

- You comply with all other terms of this agreement for free
     distribution of Project Gutenberg-tm works.

1.E.9.  If you wish to charge a fee or distribute a Project Gutenberg-tm
electronic work or group of works on different terms than are set
forth in this agreement, you must obtain permission in writing from
both the Project Gutenberg Literary Archive Foundation and Michael
Hart, the owner of the Project Gutenberg-tm trademark.  Contact the
Foundation as set forth in Section 3 below.

1.F.

1.F.1.  Project Gutenberg volunteers and employees expend considerable
effort to identify, do copyright research on, transcribe and proofread
public domain works in creating the Project Gutenberg-tm
collection.  Despite these efforts, Project Gutenberg-tm electronic
works, and the medium on which they may be stored, may contain
"Defects," such as, but not limited to, incomplete, inaccurate or
corrupt data, transcription errors, a copyright or other intellectual
property infringement, a defective or damaged disk or other medium, a
computer virus, or computer codes that damage or cannot be read by
your equipment.

1.F.2.  LIMITED WARRANTY, DISCLAIMER OF DAMAGES - Except for the "Right
of Replacement or Refund" described in paragraph 1.F.3, the Project
Gutenberg Literary Archive Foundation, the owner of the Project
Gutenberg-tm trademark, and any other party distributing a Project
Gutenberg-tm electronic work under this agreement, disclaim all
liability to you for damages, costs and expenses, including legal
fees.  YOU AGREE THAT YOU HAVE NO REMEDIES FOR NEGLIGENCE, STRICT
LIABILITY, BREACH OF WARRANTY OR BREACH OF CONTRACT EXCEPT THOSE
PROVIDED IN PARAGRAPH 1.F.3.  YOU AGREE THAT THE FOUNDATION, THE
TRADEMARK OWNER, AND ANY DISTRIBUTOR UNDER THIS AGREEMENT WILL NOT BE
LIABLE TO YOU FOR ACTUAL, DIRECT, INDIRECT, CONSEQUENTIAL, PUNITIVE OR
INCIDENTAL DAMAGES EVEN IF YOU GIVE NOTICE OF THE POSSIBILITY OF SUCH
DAMAGE.

1.F.3.  LIMITED RIGHT OF REPLACEMENT OR REFUND - If you discover a
defect in this electronic work within 90 days of receiving it, you can
receive a refund of the money (if any) you paid for it by sending a
written explanation to the person you received the work from.  If you
received the work on a physical medium, you must return the medium with
your written explanation.  The person or entity that provided you with
the defective work may elect to provide a replacement copy in lieu of a
refund.  If you received the work electronically, the person or entity
providing it to you may choose to give you a second opportunity to
receive the work electronically in lieu of a refund.  If the second copy
is also defective, you may demand a refund in writing without further
opportunities to fix the problem.

1.F.4.  Except for the limited right of replacement or refund set forth
in paragraph 1.F.3, this work is provided to you 'AS-IS', WITH NO OTHER
WARRANTIES OF ANY KIND, EXPRESS OR IMPLIED, INCLUDING BUT NOT LIMITED TO
WARRANTIES OF MERCHANTABILITY OR FITNESS FOR ANY PURPOSE.

1.F.5.  Some states do not allow disclaimers of certain implied
warranties or the exclusion or limitation of certain types of damages.
If any disclaimer or limitation set forth in this agreement violates the
law of the state applicable to this agreement, the agreement shall be
interpreted to make the maximum disclaimer or limitation permitted by
the applicable state law.  The invalidity or unenforceability of any
provision of this agreement shall not void the remaining provisions.

1.F.6.  INDEMNITY - You agree to indemnify and hold the Foundation, the
trademark owner, any agent or employee of the Foundation, anyone
providing copies of Project Gutenberg-tm electronic works in accordance
with this agreement, and any volunteers associated with the production,
promotion and distribution of Project Gutenberg-tm electronic works,
harmless from all liability, costs and expenses, including legal fees,
that arise directly or indirectly from any of the following which you do
or cause to occur: (a) distribution of this or any Project Gutenberg-tm
work, (b) alteration, modification, or additions or deletions to any
Project Gutenberg-tm work, and (c) any Defect you cause.


Section  2.  Information about the Mission of Project Gutenberg-tm

Project Gutenberg-tm is synonymous with the free distribution of
electronic works in formats readable by the widest variety of computers
including obsolete, old, middle-aged and new computers.  It exists
because of the efforts of hundreds of volunteers and donations from
people in all walks of life.

Volunteers and financial support to provide volunteers with the
assistance they need are critical to reaching Project Gutenberg-tm's
goals and ensuring that the Project Gutenberg-tm collection will
remain freely available for generations to come.  In 2001, the Project
Gutenberg Literary Archive Foundation was created to provide a secure
and permanent future for Project Gutenberg-tm and future generations.
To learn more about the Project Gutenberg Literary Archive Foundation
and how your efforts and donations can help, see Sections 3 and 4
and the Foundation information page at www.gutenberg.org


Section 3.  Information about the Project Gutenberg Literary Archive
Foundation

The Project Gutenberg Literary Archive Foundation is a non profit
501(c)(3) educational corporation organized under the laws of the
state of Mississippi and granted tax exempt status by the Internal
Revenue Service.  The Foundation's EIN or federal tax identification
number is 64-6221541.  Contributions to the Project Gutenberg
Literary Archive Foundation are tax deductible to the full extent
permitted by U.S. federal laws and your state's laws.

The Foundation's principal office is located at 4557 Melan Dr. S.
Fairbanks, AK, 99712., but its volunteers and employees are scattered
throughout numerous locations.  Its business office is located at 809
North 1500 West, Salt Lake City, UT 84116, (801) 596-1887.  Email
contact links and up to date contact information can be found at the
Foundation's web site and official page at www.gutenberg.org/contact

For additional contact information:
     Dr. Gregory B. Newby
     Chief Executive and Director
     gbnewby@pglaf.org

Section 4.  Information about Donations to the Project Gutenberg
Literary Archive Foundation

Project Gutenberg-tm depends upon and cannot survive without wide
spread public support and donations to carry out its mission of
increasing the number of public domain and licensed works that can be
freely distributed in machine readable form accessible by the widest
array of equipment including outdated equipment.  Many small donations
($1 to $5,000) are particularly important to maintaining tax exempt
status with the IRS.

The Foundation is committed to complying with the laws regulating
charities and charitable donations in all 50 states of the United
States.  Compliance requirements are not uniform and it takes a
considerable effort, much paperwork and many fees to meet and keep up
with these requirements.  We do not solicit donations in locations
where we have not received written confirmation of compliance.  To
SEND DONATIONS or determine the status of compliance for any
particular state visit www.gutenberg.org/donate

While we cannot and do not solicit contributions from states where we
have not met the solicitation requirements, we know of no prohibition
against accepting unsolicited donations from donors in such states who
approach us with offers to donate.

International donations are gratefully accepted, but we cannot make
any statements concerning tax treatment of donations received from
outside the United States.  U.S. laws alone swamp our small staff.

Please check the Project Gutenberg Web pages for current donation
methods and addresses.  Donations are accepted in a number of other
ways including checks, online payments and credit card donations.
To donate, please visit:  www.gutenberg.org/donate


Section 5.  General Information About Project Gutenberg-tm electronic
works.

Professor Michael S. Hart was the originator of the Project Gutenberg-tm
concept of a library of electronic works that could be freely shared
with anyone.  For forty years, he produced and distributed Project
Gutenberg-tm eBooks with only a loose network of volunteer support.

Project Gutenberg-tm eBooks are often created from several printed
editions, all of which are confirmed as Public Domain in the U.S.
unless a copyright notice is included.  Thus, we do not necessarily
keep eBooks in compliance with any particular paper edition.

Most people start at our Web site which has the main PG search facility:

     www.gutenberg.org

This Web site includes information about Project Gutenberg-tm,
including how to make donations to the Project Gutenberg Literary
Archive Foundation, how to help produce our new eBooks, and how to
subscribe to our email newsletter to hear about new eBooks.

\end{verbatim}
\normalsize

\iffalse
-----File: 092.png---\ggna\JohnHa0\----------------------------------------
INDEX

Algebraic numbers, 76.

Analytical theory of numbers, 83--84.

Arithmetic forms, 81--83.

Arithmetic progression, 13.



Bachmann, 80, 82, 84.

Bussey, 81.



Carmichael, 83.

Circle, Division of, 76.

Common divisors, 9, 18--20, 21.
  multiples, 9, 20--21.

Composite numbers, 10.

Congruences, 37--46.
  Linear, 43--46, 56.
  Solution by trial, 39--40.
  with prime modulus, 41--43.



Descent, Infinite, 86.

Dickson, 81.

Diophantine equations, 84.

Dirichlet, 81.

Divisibility, 8.

Divisors of a numbers, 16, 17.



Equations, Diophantine, 84.

Equation $x^n + y^n =z^n$, 91--92.

Eratosthenes, 11.

Euclid, Theorem of, 13.

Euclidian algorithm, 18.

Euler, 28, 48.

Euler's criterion, 59, 77.
  $\phi$-function, 30.

Exponent of an integer, 61--63.



Factorization theorem, 14.

Factors, 14, 16, 17, 18.

Fermat, 28, 48, 86.

Fermat's general theorem, 47, 63.
  last theorem, 91.
  simple theorem, 48, 55.
  theorem extended, 52--54.

Forms, 81--83.

Fundamental notions, 7.



Galois imaginaries, 80.

Gauss, 37.

Greatest common factor, 18--20, 21.



Highest power of $p$ in $n!$, 24--28.



Imaginaries of Galois, 80.

Indicator, 30--36.
  of any integer, 32--34.
  of a prime power, 30.
  of a product, 30--32.

Infinite descent, 86.



$\lambda(m), 53.

Law of quadratic reciprocity, 80.

Least common multiple, 20--21.

Legendre symbol, 77.



$\phi(m), 30.

Prime each to each, 9.

Prime numbers, 10, 12, 13, 28--29, 51, 76, 81, 82.

Primitive roots, 61--75.
  $\lambda$-roots, 71--74.
  $\psi$-roots, 71.

Pythagorean triangles, 85--90.
-----File: 093.png---\ggna\JohnHa0\----------------------------------------

Quadratic forms, 82.

Quadratic reciprocity, 80.

Quadratic residues, 57--60, 77--80.



Relatively prime, 10.

Residue, 37, 58.



Scales of notation, 22--24.

Sieve of Eratosthenes, 10.



Totient, 30.

Triangles, Numerical, 85.



Unit, 8.



Veblen, 81.



Wilson's theorem, 49--81.



Young, 81.
\fi
% %%%%%%%%%%%%%%%%%%%%%%%%%%%%%%%%%%%%%%%%%%%%%%%%%%%%%%%%%%%%%%%%%%%%%%% %
%                                                                         %
% End of Project Gutenberg's The Theory of Numbers, by Robert D. Carmichael
%                                                                         %
% *** END OF THIS PROJECT GUTENBERG EBOOK THE THEORY OF NUMBERS ***       %
%                                                                         %
% ***** This file should be named 13693-t.tex or 13693-t.zip *****        %
% This and all associated files of various formats will be found in:      %
%         http://www.gutenberg.org/1/3/6/9/13693/                         %
%                                                                         %
% %%%%%%%%%%%%%%%%%%%%%%%%%%%%%%%%%%%%%%%%%%%%%%%%%%%%%%%%%%%%%%%%%%%%%%% %

\end{document}

This is pdfTeX, Version 3.1415926-1.40.10 (TeX Live 2009/Debian) (format=pdflatex 2012.9.24)  8 APR 2013 03:28
entering extended mode
 %&-line parsing enabled.
**13693-t.tex
(./13693-t.tex
LaTeX2e <2009/09/24>
Babel <v3.8l> and hyphenation patterns for english, usenglishmax, dumylang, noh
yphenation, farsi, arabic, croatian, bulgarian, ukrainian, russian, czech, slov
ak, danish, dutch, finnish, french, basque, ngerman, german, german-x-2009-06-1
9, ngerman-x-2009-06-19, ibycus, monogreek, greek, ancientgreek, hungarian, san
skrit, italian, latin, latvian, lithuanian, mongolian2a, mongolian, bokmal, nyn
orsk, romanian, irish, coptic, serbian, turkish, welsh, esperanto, uppersorbian
, estonian, indonesian, interlingua, icelandic, kurmanji, slovenian, polish, po
rtuguese, spanish, galician, catalan, swedish, ukenglish, pinyin, loaded.
(/usr/share/texmf-texlive/tex/latex/base/book.cls
Document Class: book 2007/10/19 v1.4h Standard LaTeX document class
(/usr/share/texmf-texlive/tex/latex/base/bk10.clo
File: bk10.clo 2007/10/19 v1.4h Standard LaTeX file (size option)
)
\c@part=\count79
\c@chapter=\count80
\c@section=\count81
\c@subsection=\count82
\c@subsubsection=\count83
\c@paragraph=\count84
\c@subparagraph=\count85
\c@figure=\count86
\c@table=\count87
\abovecaptionskip=\skip41
\belowcaptionskip=\skip42
\bibindent=\dimen102
) (/usr/share/texmf-texlive/tex/latex/base/inputenc.sty
Package: inputenc 2008/03/30 v1.1d Input encoding file
\inpenc@prehook=\toks14
\inpenc@posthook=\toks15
(/usr/share/texmf-texlive/tex/latex/base/latin1.def
File: latin1.def 2008/03/30 v1.1d Input encoding file
)) (/usr/share/texmf-texlive/tex/latex/amsmath/amsmath.sty
Package: amsmath 2000/07/18 v2.13 AMS math features
\@mathmargin=\skip43
For additional information on amsmath, use the `?' option.
(/usr/share/texmf-texlive/tex/latex/amsmath/amstext.sty
Package: amstext 2000/06/29 v2.01
(/usr/share/texmf-texlive/tex/latex/amsmath/amsgen.sty
File: amsgen.sty 1999/11/30 v2.0
\@emptytoks=\toks16
\ex@=\dimen103
)) (/usr/share/texmf-texlive/tex/latex/amsmath/amsbsy.sty
Package: amsbsy 1999/11/29 v1.2d
\pmbraise@=\dimen104
) (/usr/share/texmf-texlive/tex/latex/amsmath/amsopn.sty
Package: amsopn 1999/12/14 v2.01 operator names
)
\inf@bad=\count88
LaTeX Info: Redefining \frac on input line 211.
\uproot@=\count89
\leftroot@=\count90
LaTeX Info: Redefining \overline on input line 307.
\classnum@=\count91
\DOTSCASE@=\count92
LaTeX Info: Redefining \ldots on input line 379.
LaTeX Info: Redefining \dots on input line 382.
LaTeX Info: Redefining \cdots on input line 467.
\Mathstrutbox@=\box26
\strutbox@=\box27
\big@size=\dimen105
LaTeX Font Info:    Redeclaring font encoding OML on input line 567.
LaTeX Font Info:    Redeclaring font encoding OMS on input line 568.
\macc@depth=\count93
\c@MaxMatrixCols=\count94
\dotsspace@=\muskip10
\c@parentequation=\count95
\dspbrk@lvl=\count96
\tag@help=\toks17
\row@=\count97
\column@=\count98
\maxfields@=\count99
\andhelp@=\toks18
\eqnshift@=\dimen106
\alignsep@=\dimen107
\tagshift@=\dimen108
\tagwidth@=\dimen109
\totwidth@=\dimen110
\lineht@=\dimen111
\@envbody=\toks19
\multlinegap=\skip44
\multlinetaggap=\skip45
\mathdisplay@stack=\toks20
LaTeX Info: Redefining \[ on input line 2666.
LaTeX Info: Redefining \] on input line 2667.
) (/usr/share/texmf-texlive/tex/latex/amsfonts/amssymb.sty
Package: amssymb 2009/06/22 v3.00
(/usr/share/texmf-texlive/tex/latex/amsfonts/amsfonts.sty
Package: amsfonts 2009/06/22 v3.00 Basic AMSFonts support
\symAMSa=\mathgroup4
\symAMSb=\mathgroup5
LaTeX Font Info:    Overwriting math alphabet `\mathfrak' in version `bold'
(Font)                  U/euf/m/n --> U/euf/b/n on input line 96.
)) (/usr/share/texmf-texlive/tex/latex/base/makeidx.sty
Package: makeidx 2000/03/29 v1.0m Standard LaTeX package
)
\@indexfile=\write3
\openout3 = `13693-t.idx'.

Writing index file 13693-t.idx
(./13693-t.aux)
\openout1 = `13693-t.aux'.

LaTeX Font Info:    Checking defaults for OML/cmm/m/it on input line 67.
LaTeX Font Info:    ... okay on input line 67.
LaTeX Font Info:    Checking defaults for T1/cmr/m/n on input line 67.
LaTeX Font Info:    ... okay on input line 67.
LaTeX Font Info:    Checking defaults for OT1/cmr/m/n on input line 67.
LaTeX Font Info:    ... okay on input line 67.
LaTeX Font Info:    Checking defaults for OMS/cmsy/m/n on input line 67.
LaTeX Font Info:    ... okay on input line 67.
LaTeX Font Info:    Checking defaults for OMX/cmex/m/n on input line 67.
LaTeX Font Info:    ... okay on input line 67.
LaTeX Font Info:    Checking defaults for U/cmr/m/n on input line 67.
LaTeX Font Info:    ... okay on input line 67.

Overfull \hbox (18.82077pt too wide) in paragraph at lines 93--93
[]\OT1/cmtt/m/n/9 The Project Gutenberg EBook of The Theory of Numbers, by Robe
rt D. Carmichael[] 
 []

[1

{/var/lib/texmf/fonts/map/pdftex/updmap/pdftex.map}] [1

]
LaTeX Font Info:    Try loading font information for U+msa on input line 186.
(/usr/share/texmf-texlive/tex/latex/amsfonts/umsa.fd
File: umsa.fd 2009/06/22 v3.00 AMS symbols A
)
LaTeX Font Info:    Try loading font information for U+msb on input line 186.
(/usr/share/texmf-texlive/tex/latex/amsfonts/umsb.fd
File: umsb.fd 2009/06/22 v3.00 AMS symbols B
) [2] [3

] [4

] (./13693-t.toc [5

])
\tf@toc=\write4
\openout4 = `13693-t.toc'.

[6]
Chapter 1.
[1


]
LaTeX Font Info:    Try loading font information for OMS+cmr on input line 401.

(/usr/share/texmf-texlive/tex/latex/base/omscmr.fd
File: omscmr.fd 1999/05/25 v2.5h Standard LaTeX font definitions
)
LaTeX Font Info:    Font shape `OMS/cmr/m/n' in size <10> not available
(Font)              Font shape `OMS/cmsy/m/n' tried instead on input line 401.
[2] [3] [4] [5] [6] [7] [8] [9] [10] [11] [12] [13] [14] [15] [16] [17] [18] [1
9]
Chapter 2.
[20

] [21]
Overfull \hbox (15.17194pt too wide) in paragraph at lines 1607--1608
[]\OT1/cmr/m/it/10 If $\OML/cmm/m/it/10 m \OT1/cmr/m/n/10 = \OML/cmm/m/it/10 p[
]p[] [] p[]$ \OT1/cmr/m/it/10 where $\OML/cmm/m/it/10 p[]; p[]; [] ; p[]$ \OT1/
cmr/m/it/10 are dif-fer-ent primes and $\OML/cmm/m/it/10 []; []; [] ; []$
 []

[22] [23] [24] [25]
Chapter 3.
[26

] [27] [28] [29] [30] [31] [32] [33]
Chapter 4.
[34

] [35] [36] [37] [38] [39] [40] [41] [42] [43] [44] [45]
Chapter 5.
[46

] [47] [48] [49] [50] [51] [52]
Overfull \hbox (2.6285pt too wide) in paragraph at lines 3312--3312
[]\OT1/cmr/bx/n/14.4 Primitive Roots Mod-ulo $\OT1/cmr/m/n/14.4 2\OML/cmm/m/it/
14.4 p[]$\OT1/cmr/bx/n/14.4 , $\OML/cmm/m/it/14.4 p$ \OT1/cmr/bx/n/14.4 an Odd 
Prime 
 []

[53] [54] [55] [56] [57]
Chapter 6.
[58

] [59] [60] [61] [62] [63] [64] [65] [66] [67] [68] [69] [70] (./13693-t.ind [7
1] [72

] [73

])
Chapter 7.
[1


] [2] [3] [4] [5] [6] [7] [8] (./13693-t.aux)

 *File List*
    book.cls    2007/10/19 v1.4h Standard LaTeX document class
    bk10.clo    2007/10/19 v1.4h Standard LaTeX file (size option)
inputenc.sty    2008/03/30 v1.1d Input encoding file
  latin1.def    2008/03/30 v1.1d Input encoding file
 amsmath.sty    2000/07/18 v2.13 AMS math features
 amstext.sty    2000/06/29 v2.01
  amsgen.sty    1999/11/30 v2.0
  amsbsy.sty    1999/11/29 v1.2d
  amsopn.sty    1999/12/14 v2.01 operator names
 amssymb.sty    2009/06/22 v3.00
amsfonts.sty    2009/06/22 v3.00 Basic AMSFonts support
 makeidx.sty    2000/03/29 v1.0m Standard LaTeX package
    umsa.fd    2009/06/22 v3.00 AMS symbols A
    umsb.fd    2009/06/22 v3.00 AMS symbols B
  omscmr.fd    1999/05/25 v2.5h Standard LaTeX font definitions
 13693-t.ind
 ***********

 ) 
Here is how much of TeX's memory you used:
 1323 strings out of 493848
 14666 string characters out of 1152823
 81402 words of memory out of 3000000
 4605 multiletter control sequences out of 15000+50000
 21383 words of font info for 81 fonts, out of 3000000 for 9000
 714 hyphenation exceptions out of 8191
 27i,13n,24p,223b,282s stack positions out of 5000i,500n,10000p,200000b,50000s
</usr/share/texmf-texlive/fonts/type1/public/amsfonts/cm/cmbx10.pfb></usr/sha
re/texmf-texlive/fonts/type1/public/amsfonts/cm/cmbx12.pfb></usr/share/texmf-te
xlive/fonts/type1/public/amsfonts/cm/cmbx8.pfb></usr/share/texmf-texlive/fonts/
type1/public/amsfonts/cm/cmbx9.pfb></usr/share/texmf-texlive/fonts/type1/public
/amsfonts/cm/cmcsc10.pfb></usr/share/texmf-texlive/fonts/type1/public/amsfonts/
cm/cmex10.pfb></usr/share/texmf-texlive/fonts/type1/public/amsfonts/cmextra/cme
x8.pfb></usr/share/texmf-texlive/fonts/type1/public/amsfonts/cmextra/cmex9.pfb>
</usr/share/texmf-texlive/fonts/type1/public/amsfonts/cm/cmmi10.pfb></usr/share
/texmf-texlive/fonts/type1/public/amsfonts/cm/cmmi12.pfb></usr/share/texmf-texl
ive/fonts/type1/public/amsfonts/cm/cmmi5.pfb></usr/share/texmf-texlive/fonts/ty
pe1/public/amsfonts/cm/cmmi6.pfb></usr/share/texmf-texlive/fonts/type1/public/a
msfonts/cm/cmmi7.pfb></usr/share/texmf-texlive/fonts/type1/public/amsfonts/cm/c
mmi8.pfb></usr/share/texmf-texlive/fonts/type1/public/amsfonts/cm/cmmi9.pfb></u
sr/share/texmf-texlive/fonts/type1/public/amsfonts/cm/cmr10.pfb></usr/share/tex
mf-texlive/fonts/type1/public/amsfonts/cm/cmr12.pfb></usr/share/texmf-texlive/f
onts/type1/public/amsfonts/cm/cmr17.pfb></usr/share/texmf-texlive/fonts/type1/p
ublic/amsfonts/cm/cmr5.pfb></usr/share/texmf-texlive/fonts/type1/public/amsfont
s/cm/cmr6.pfb></usr/share/texmf-texlive/fonts/type1/public/amsfonts/cm/cmr7.pfb
></usr/share/texmf-texlive/fonts/type1/public/amsfonts/cm/cmr8.pfb></usr/share/
texmf-texlive/fonts/type1/public/amsfonts/cm/cmr9.pfb></usr/share/texmf-texlive
/fonts/type1/public/amsfonts/cm/cmsl10.pfb></usr/share/texmf-texlive/fonts/type
1/public/amsfonts/cm/cmsy10.pfb></usr/share/texmf-texlive/fonts/type1/public/am
sfonts/cm/cmsy5.pfb></usr/share/texmf-texlive/fonts/type1/public/amsfonts/cm/cm
sy6.pfb></usr/share/texmf-texlive/fonts/type1/public/amsfonts/cm/cmsy7.pfb></us
r/share/texmf-texlive/fonts/type1/public/amsfonts/cm/cmsy8.pfb></usr/share/texm
f-texlive/fonts/type1/public/amsfonts/cm/cmsy9.pfb></usr/share/texmf-texlive/fo
nts/type1/public/amsfonts/cm/cmti10.pfb></usr/share/texmf-texlive/fonts/type1/p
ublic/amsfonts/cm/cmti7.pfb></usr/share/texmf-texlive/fonts/type1/public/amsfon
ts/cm/cmti8.pfb></usr/share/texmf-texlive/fonts/type1/public/amsfonts/cm/cmtt9.
pfb></usr/share/texmf-texlive/fonts/type1/public/amsfonts/symbols/msam10.pfb>
Output written on 13693-t.pdf (88 pages, 690196 bytes).
PDF statistics:
 425 PDF objects out of 1000 (max. 8388607)
 0 named destinations out of 1000 (max. 500000)
 1 words of extra memory for PDF output out of 10000 (max. 10000000)

